\documentclass{lmcs}
\pdfoutput=1

\usepackage{enumerate}
\usepackage[colorlinks=true]{hyperref}
\usepackage{amssymb}
\usepackage{xcolor,latexsym,amsmath,extarrows,alltt}
\usepackage{xspace}
\usepackage{booktabs}
\usepackage{mathtools}
\usepackage{enumitem}
\usepackage{stmaryrd}
\usepackage{microtype}

\theoremstyle{theorem}\newtheorem{theorem}{Theorem}
\theoremstyle{theorem}\newtheorem{lemma}[theorem]{Lemma}
\theoremstyle{theorem}\newtheorem{corollary}[theorem]{Corollary}
\theoremstyle{definition}\newtheorem{definition}[theorem]{Definition}
\theoremstyle{definition}\newtheorem{example}[theorem]{Example}

\newcommand{\N}{\mathbb{N}}
\newcommand{\F}{\mathcal{F}}
\newcommand{\V}{\mathcal{V}}
\newcommand{\M}{\mathcal{M}}
\newcommand{\Vfree}{\mathcal{V}_{\mathit{nonb}}}
\newcommand{\Vbound}{\mathcal{V}_{\mathit{binder}}}
\newcommand{\Sorts}{\mathcal{S}}
\newcommand{\Types}{\mathcal{Y}}
\newcommand{\Terms}{\mathcal{T}}
\newcommand{\ATerms}{\mathcal{T}_{\mathcal{A}}}
\newcommand{\FOTerms}{\mathcal{T}_{\mathcal{FO}}}
\newcommand{\Rules}{\mathcal{R}}
\newcommand{\FV}{\mathit{FV}}
\newcommand{\FMV}{\mathit{MV}}
\newcommand{\Positions}{\mathit{Positions}}
\newcommand{\SubPositions}{\mathit{SubPositions}}
\newcommand{\HeadPositions}{\mathit{HeadPositions}}
\newcommand{\Pairs}{\mathit{Pairs}}

\newcommand{\domain}{\mathtt{dom}}
\newcommand{\order}{\mathit{order}}

\newcommand{\asort}{\iota}
\newcommand{\bsort}{\kappa}
\newcommand{\atype}{\sigma}
\newcommand{\btype}{\tau}
\newcommand{\ctype}{\pi}
\newcommand{\dtype}{\alpha}
\newcommand{\identifier}[1]{\mathtt{#1}}
\newcommand{\afun}{\identifier{f}}
\newcommand{\bfun}{\identifier{g}}
\newcommand{\cfun}{\identifier{h}}
\newcommand{\avar}{x}
\newcommand{\bvar}{y}
\newcommand{\cvar}{z}
\newcommand{\Avar}{X}
\newcommand{\Bvar}{Y}
\newcommand{\Cvar}{Z}
\newcommand{\AFvar}{F}
\newcommand{\BFvar}{G}
\newcommand{\CFvar}{H}
\newcommand{\ameta}{F}
\newcommand{\bmeta}{G}
\newcommand{\cmeta}{Z}

\newcommand{\clause}[1]{\textbf{#1}}

\newcommand{\abs}[2]{\lambda #1.#2}
\newcommand{\meta}[2]{#1\langle#2\rangle}
\newcommand{\superapply}{\mathtt{superapply}}
\newcommand{\tuple}[2]{\llparenthesis #1,\dots,#2 \rrparenthesis}
\newcommand{\product}[2]{\llparenthesis #1,\dots,#2 \rrparenthesis}
\newcommand{\pair}[2]{\llparenthesis #1,#2 \rrparenthesis}
\newcommand{\triple}[3]{\llparenthesis #1,#2,#3 \rrparenthesis}

\newcommand{\arity}{\mathit{arity}}
\newcommand{\head}{\mathsf{head}}
\newcommand{\arrtype}{\Rightarrow}
\newcommand{\arrz}{\Rightarrow}
\newcommand{\arr}[1]{\arrz_{#1}}
\newcommand{\arrr}[1]{\arr{#1}^*}
\newcommand{\subtermeq}{\unlhd}
\newcommand{\headsubtermeq}{\unlhd_{\bullet}}
\newcommand{\supterm}{\rhd}
\newcommand{\suptermeq}{\unrhd}

\newcommand{\symb}[1]{\mathtt{#1}}

\newcommand{\nul}{\symb{0}}
\newcommand{\one}{\symb{1}}
\newcommand{\nil}{\symb{nil}}
\newcommand{\cons}{\symb{cons}}
\newcommand{\strue}{\symb{true}}
\newcommand{\sfalse}{\symb{false}}
\newcommand{\suc}{\symb{s}}
\newcommand{\map}{\symb{map}}
\newcommand{\bool}{\symb{bool}}
\newcommand{\nat}{\symb{nat}}
\newcommand{\lijst}{\symb{list}}
\newcommand{\unitsort}{\mathtt{o}}

\newcommand{\cora}{\textsf{CORA}\xspace}
\newcommand{\charlie}{\textsf{Charlie}\xspace}

\newcommand{\secshort}{\S}
\newcommand{\myparagraph}[1]{\paragraph{\textbf{#1}}}

\setlength{\parindent}{0pt}
\setlength{\parskip}{\bigskipamount}
\setlist[itemize]{topsep=-\bigskipamount}

\newcommand{\mysubsection}[1]{\vspace{-12pt}\subsubsection{#1}}

\begin{document}

\title{Constrained Higher-order Analysis Rewriting LIbrary: formalism}
\author{Cynthia Kop}
\address{Department of Software Science, Radboud University Nijmegen}
\email{C.Kop@cs.ru.nl}

\maketitle

\begin{abstract}
\cora\ is a tool meant to analyse constrained term rewriting systems, both
first-order and higher-order.
The cora code \charlie\ provides functionality to represent higher-order term
rewriting systems (with constraints).
This document explains the underlying formalism.
\end{abstract}

\section{Types}

We fix a set $\Sorts$ of \emph{sorts} and define the set $\Types$ of
\emph{types} inductively:
\begin{itemize}
\item all elements of $\Sorts$ are types (also called \emph{base types});
\item if $\atype,\btype \in \Types$ then $\atype \arrtype \btype$ is also a type
  (called an \emph{arrow type});
\item if $\atype_1,\dots,\atype_n \in \Types$ with $n \geq 2$ then
  $\product{\atype_1}{\atype_n}$ is also a type (called a \emph{product type}).
\end{itemize}
The arrow operator $\arrtype$ is right-associative, so all types can be denoted
in a form $\atype_1 \arrtype \dots \arrtype \atype_m \arrtype \asort$ with
$\asort$ a base type or product type; we say the \emph{arity} of this type is
$m$, and the \emph{output type} is $\asort$.

The \emph{order} of a type is recursively defined as follows:
\begin{itemize}
\item for $\asort \in \Sorts$: $\order(\asort) = 0$;
\item for arrow types: $\order(\atype \arrtype \btype) = \max(\order(\atype) +
  1,\order(\btype))$.
\item for product types: $\order(\product{\atype_1}{\atype_n}) =
  \max(\order(\atype_1),\dots,\order(\atype_n))$
\end{itemize}

\bigskip
Type equality is literal equality (i.e., $\atype_1 \arrtype \btype_1$ is equal
to $\atype_2 \arrtype \btype_2$ iff $\atype_1 = \atype_2$ and $\btype_1 =
\btype_2$, and $\product{\atype_1}{\atype_n}$ is equal to
$\product{\btype_1}{\btype_m}$ if $n = m$ and each $\atype_i = \btype_i$).
We do \emph{not} flatten product types; e.g.,
$\pair{\atype}{\pair{\btype}{\ctype}} \neq \triple{\atype}{\btype}{\ctype}$.

\subsection*{Remarks}

We do not impose limitations on the set of sorts.  In traditional, unsorted term
rewriting, there is only one sort (e.g., $\Sorts = \{ \unitsort \}$). However,
we may also have a larger finite or even infinite sort set.
In the future, we may consider a shallow form of polymorphic types, but for the
moment we will limit interest to these simple types.

\subsection*{In \charlie}

\begin{itemize}
\item The class \texttt{charlie.types.Type} represents types.
\item Types can be constructed using \texttt{charlie.types.TypeFactory}.
\item Types can be printed using \texttt{charlie.types.TypePrinter} or any class
  inheriting it.
  (The in-built \texttt{toString()} is meant for debugging and unit testing, and
  should typically not be used when printing to users.)
\end{itemize}

\section{Unconstrained Applicative Meta-variable Systems (AMSs)}

Before explaining the full formalism of \cora, let us start by explaining
higher-order term rewriting systems without constraints.  Most of the notions
will be directly relevant to constrained systems as well.

\subsection{Terms}
Terms are \emph{well-typed} expressions built over given sets of \emph{function
symbols}, \emph{variables} and \emph{meta-variables}. The full definition is
presented below.

\mysubsection{Symbols and (meta-)variables}

We fix a set $\F$ of \emph{function symbols}, also called the \emph{alphabet};
each function symbol is a \emph{typed constant}. Notation: $\afun \in \F$ or
$(\afun :: \atype) \in \F$ if we wish to explicitly refer to the type (but the
type should be considered implicit in the symbol).
Function symbols will generally be referred to as $\afun,\bfun,\cfun$ or using
more suggestive notation.

We also fix a set $\V$ of \emph{variables}, which are typed constants in the
same way.  $\V$ should be disjoint from $\F$, and we assume that $\V = \Vfree
\uplus \Vbound$, where both $\Vbound$ and $\Vfree$ contain infinitely many
variables of each type.
Variables will generally be referred to as $\avar,\bvar,\cvar,\Avar,\Bvar,\Cvar,
\AFvar,\BFvar,\CFvar$ or using more suggestive notation.

Finally, we fix a set $\M$ of \emph{meta-variables}, which are constants
equipped with both a type $\atype$ and an arity $k$, where $k$ is an integer
such that $0 \leq k \leq \arity(\atype)$.  Notation:
$X :: (\atype_1 \arrtype \dots \arrtype \atype_k \arrtype \btype,k) \in \M$ or
$X :: [\atype_1,\dots,\atype_k] \arrtype \btype \in \M$, for a meta-variable
with arity $k$ and type $\atype_1 \arrtype \dots \arrtype \atype_k \arrtype
\btype$.  We require that $\Vfree = \{ (x :: \atype) \mid x :: (\atype,0) \in
\M \}$; so the non-binder variables are exactly the meta-variables with arity
$0$.  We also require that the symbols in $\F$ and $\Vbound$ are disjoint from
those in $\M$.

\mysubsection{Term formation}\label{subsec:form}

Terms are those expressions $s$ such that $s :: \atype$ can be derived for some
$\atype \in \Types$ using the following clauses:

\begin{description}
\item[constant] if $(\afun :: \atype) \in \F$ then $\afun :: \atype$
\item[variable] if $(\avar :: \atype) \in \Vfree \cup \Vbound$ then $\avar ::
  \atype$
\item[application] if $h :: \atype_1 \arrtype \dots \arrtype \atype_n \arrtype
  \btype$ with $n > 0$ and $h$ is a constant, variable, abstraction or
  meta-application, and if $s_1 :: \atype_1,\dots,s_n :: \atype_n$,
  then $h(s_1,\dots,s_n) :: \btype$
\item[tuple] if $s_1 :: \atype_1,\dots,s_n :: \atype_n$, then
  $\tuple{s_1}{s_n} :: \product{\atype_1}{\atype_n}$
\item[abstraction] if $(\avar :: \atype) \in \Vbound$ and $t :: \btype$ then
  $\abs{\avar}{t} :: \atype \arrtype \btype$
\item[meta-application] if $(\ameta :: (\atype_1 \arrtype \dots \arrtype
  \atype_k \arrtype \btype,k)) \in \M$ for some $k \geq 1$ and $t_1 :: \atype_1,
  \dots,t_k :: \atype_n$ then $\meta{\ameta}{t_1,\dots,t_k} :: \btype$
\end{description}

We will identify $h() = h$.  This allows us to write any term in a form $h(s_1,
\dots,s_n)$ with $n \geq 0$, where the \emph{head} $h$ is either a constant,
variable, abstraction, tuple or meta-application.  We will often use this notation.

Moreover, for $\avar \in \Vfree$, we identify $\meta{\avar}{} = \avar$.  We will
refer to any occurrence of $\meta{\ameta}{t_1,\dots,t_k}$ with $k \geq 0$ (so
including variables in $\Vfree$) as a meta-application, and $\ameta$ is its
meta-variable.

Using this notation (so $n \geq 0, k \geq 0$), we define:
\begin{itemize}
\item A term of the form $\afun(s_1,\dots,s_n)$ is called a \emph{functional
  term} and $\afun$ is its root.
\item A term of the form $\avar(s_1,\dots,s_n)$ is called a \emph{var term}, and
  $\avar$ is its variable.
\item The \emph{variable of} an abstraction $\abs{\avar}{t}$ or a var term
  $\avar(s_1,\dots,s_n)$ is $\avar$.
\item The \emph{meta-variable of} a meta-application $\meta{\ameta}{t_1,\dots,
  t_k}$ is $\ameta$.
\item An application of the form $(\abs{\avar}{t})(s_0,\dots,s_n)$ is called a
  \emph{$\beta$-redex} and $\avar$ is its variable.
\item A term of the form $\meta{x}{t_1,\dots,t_k}(s_{k+1},\dots,s_n)$ with
  $n > k \geq 0$ is a meta-application-application, and $\avar$ is its variable.
  However, while we permit the formation of meta-application-applications to
  avoid many exceptions in for instance termination techniques, in practice we
  will typically only consider terms of this form with $k = 0$.
\end{itemize}

If $s :: \atype$ then we say that $\atype$ is the type of $s$; it is clear from
the definitions above that each term has a unique type.

Note that in the \clause{application} and \clause{meta-application} clauses, $n$
and $k$ are not required to be maximal; for example, if $\symb{greater} ::
\mathtt{int} \arrtype \mathtt{int} \arrtype \mathtt{bool}$, then each of
$\symb{greater}(),\symb{greater}(\avar)$ and $\symb{greater}(\avar,\bvar)$ are
terms (with distinct types); each having $\symb{greater}$ as its head.
Note also that a variable $\avar$ is also considered a var term, and a constant
$\afun$ is a functional term, but a plain abstraction is \emph{not} a
$\beta$-redex.

\mysubsection{$\alpha$-equality}
We let $=_\alpha$ be the usual $\alpha$-renaming equivalence relation as used in
the $\lambda$-calculus. This relation can be formally defined as follows:
\begin{itemize}
\item Let $\mu_0,\nu_0 : \Vbound \rightarrow \N$ be defined as follows:
  $\mu_0(\avar) = \nu_0(\avar) = 0$ for all $\avar \in \Vbound$.
\item Let $s =_\alpha t$ iff $s =_\alpha^{\mu_0,\nu_0,1} t$.
\item For $\mu,\nu : \Vbound \rightarrow \N$ and $k \in \N$, let
  $s =_\alpha^{\mu,\nu,k} t$ if and only if this can be defined by the following
  clauses:
  \begin{itemize}
  \item $\afun =_\alpha^{\mu,\nu,k} \afun$ for all $\afun \in \F$
  \item $a(s_1,\dots,s_n) =_\alpha^{\mu,\nu,k} b(t_1,\dots,t_n)$ with $n > 0$ if
    $a =_\alpha^{\mu,\nu,k} b$ and $s_i =_\alpha^{\mu,\nu,k} t_i$ for all $i \in
    \{1,\dots,n\}$;
  \item $\tuple{s_1}{s_n} =_\alpha^{\mu,\nu,k} \tuple{t_1}{t_n}$ if
    $s_i =_\alpha^{\mu,\nu,k+1} t_i$ for all $i \in \{1,\dots,k\}$.
  \item $\meta{\ameta}{s_1,\dots,s_k} =_\alpha^{\mu,\nu,k} \meta{\ameta}{t_1,
    \dots,t_k}$ if $s_i =_\alpha^{\mu,\nu,k+1} t_i$ for all $i \in \{1,\dots,
    k\}$. \\
    (Note that this includes the case where $k = 0$, so $\avar =_\alpha^{\mu,
    \nu,k} \avar$ for all $\avar \in \Vfree$.)
  \item $\avar =_\alpha^{\mu,\nu,k} \bvar$ for $\avar,\bvar \in \Vbound$ if
    either  $\avar = \bvar$ and $\mu(\avar) = \nu(\bvar) = 0$, or $\mu(\avar) =
    \mu(\bvar) > 0$;
  \item $\abs{\avar}{s} =_\alpha^{\mu,\nu,k} \abs{\bvar}{t}$ iff $s
    =_\alpha^{\mu[\avar:=k],\mu[\bvar:=k],k+1} t$; \\
    (Here, $\mu[\avar:=k]$ is the function that maps $\avar$ to $k$ and all
    other $\cvar$ to $\mu(\cvar)$; similar for $\nu[\bvar:=k]$.)
  \end{itemize}
\end{itemize}
That is, we progressively descend into the term and keep track of where
variables are bound; the structure of the two terms has to be exactly the same,
and function symbols and unbound variable should occur at the same positions in
both terms. However, when encountering a bound variable, we only require that
this variable was bound by the same $\lambda$ in both terms.  We can
straightforwardly prove that $=_\alpha$ is an equivalence relation (Corollary
\ref{corr:alphaequiv}).

\mysubsection{Restricted terms}\label{subsec:termsets}

The definition of terms is deliberately broad, to support a rather liberal kind
of higher-order rewriting; in some other works, what we call ``terms'' would be
referred to as \emph{metaterms}, with the word \emph{terms} referring to what we
will call ``true terms''.  We use this terminology simply because in an analysis
tool, the vast majority of reasoning happens on metaterms, so these are the
primary objects we wish to consider.

However, in practice we do often consider limitations: specific kinds of terms
which are built using subsets and restrictions of the clauses in Section
\ref{subsec:form}.  We consider the most important ones:

\begin{itemize}
\item A \emph{true term} is a term without meta-applications; that is, a term
  whose type can be derived using only the clauses \clause{constant},
  \clause{variable}, \clause{application}, \clause{tuple} and
  \clause{abstraction}.

\item A term $s$ is a \emph{semi-pattern} if the arguments to a meta-application
  must all be distinct bound variables.  That is, a term whose type can be
  derived using the clauses \clause{constant}, \clause{variable},
  \clause{application}, \clause{tuple}, \clause{abstraction} and:
  \begin{description}
  \item[meta-pattern] if $(\ameta :: (\atype_1 \arrtype \dots \arrtype \atype_k
    \arrtype \btype,k)) \in \M$ for some $k > 0$ and $(\bvar_1 :: \atype_1),
    \dots,(\bvar_k :: \atype_k) \in \Vbound$ are all distinct, then
    $\meta{\ameta}{\bvar_1,\dots,\bvar_k} :: \btype$
  \end{description}
  (Note that \clause{meta-pattern} is a restriction of
  \clause{meta-application} so all semi-patterns are indeed meta-terms.  Any
  true term is necessarily a semi-pattern.)

\item A term $s$ is a \emph{pattern} if it is a semi-pattern, and moreover
  abstractions, meta-variables and non-binder variables do not occur at the head
  of an application.
  That is, a pattern is a term whose type can be derived using the clauses
  \clause{tuple}, \clause{abstraction}, \clause{meta-pattern} and:
  \begin{description}
  \item[func] if $(\afun :: \atype_1 \arrtype \dots \arrtype \atype_n
    \arrtype \btype) \in \F$ and $s_1 :: \atype_1,\dots,s_n :: \atype_n$ with
    $n \geq 0$ then $\afun(s_1,\dots,s_n) :: \btype$
  \item[bvarterm] if $(\avar :: \atype_1 \arrtype \dots \arrtype \atype_n
    \arrtype \btype) \in \Vbound$ and $s_1 :: \atype_1,\dots,s_n :: \atype_n$
    with $n \geq 0$ then $\avar(s_1,\dots,s_n) :: \btype$
  \item[freevar] if $(\avar :: \atype) \in \Vfree$ then $\avar :: \atype$
  \end{description}
  (Note that \clause{func} is a combination of \clause{constant} and
  \clause{application}; \clause{bvarterm} combines a restriction of
  \clause{variable} and \clause{application}; and \clause{freevar} is a
  restriction of \clause{variable}
  So, any expression typed using these expressions is indeed a term.)

\item Hence, a true term is a pattern if it can be typed using only
  \clause{func}, \clause{bvarterm}, \clause{freevar}, \clause{tuple} and
  \clause{abstraction}.

\item A term $s$ is \emph{applicative} if it does not use meta-applications or
  variables in $\Vbound$.  Note that this limitation excludes abstractions.
  Hence, a term is applicative if it can be typed using just the clauses
  \clause{constant}, \clause{freevar}, \clause{application} and \clause{tuple}.
  Equivalently, an applicative term can be typed using just \clause{func},
  \clause{tuple} and:
  \begin{description}
  \item[fvarterm] if $(\avar :: \atype_1 \arrtype \dots \arrtype \atype_n
    \arrtype \btype) \in \Vfree$ and $s_1 :: \atype_1,\dots,s_n :: \atype_n$
    then $\avar(s_1,\dots,s_n) :: \btype$
  \end{description}

\item Hence, an \emph{applicative pattern} can be typed using just the clauses
  \clause{func}, \clause{tuple} and \clause{freevar}.
\item A term is \emph{first-order} if it is an applicative pattern with only
  base-type subterms; that is, a term whose type can be derived using only
  \clause{tuple} and:
  \begin{description}
  \item[(fofunc)] if $(\afun :: \atype_1 \arrtype \dots \arrtype \atype_n
    \arrtype \asort) \in \F$ with $\asort \in \Sorts$ and $s_1 :: \atype_1,
    \dots,s_n :: \atype_n$ then $\afun(s_1,\dots,s_n) :: \asort$
  \item[(fovar)] if $(\avar :: \asort) \in \Vfree$ with $\asort \in \Sorts$ then
    $\avar :: \asort$.
  \end{description}
\end{itemize}

We denote $\Terms(\F,\V)$ for the set of all true terms $s$, modulo $=_\alpha$.
In practice, we will reason with true terms rather than equivalence classes, but
always consider equality modulo $=_\alpha$.

The set of applicative terms is denoted $\ATerms(\F,\V)$.  Note that $=_\alpha$
is the identity on applicative terms, and that every applicative term is a true
term, so $\ATerms(\F,\V) \subseteq \Terms(\F,\V)$.

The set of first-order terms is denoted $\FOTerms(\F,\V)$.  Since every
first-order term is also an applicative pattern, $\FOTerms(\F,\V) \subseteq
\ATerms(\F,\V) \subseteq \Terms(\F,\V)$.

\subsection{Free variables}
The set of \emph{free variables} of a term is inductively defined as follows:
\begin{itemize}
\item $\FV(\afun) = \emptyset$;
\item $\FV(\avar) = \{ \avar \}$;
\item $\FV(h(s_1,\dots,s_n)) = \FV(h) \cup \FV(s_1) \cup \dots \cup \FV(s_n)$ if
  $n > 0$;
\item $\FV(\tuple{s_1}{s_n}) = \FV(s_1) \cup \dots \cup \FV(s_n)$.
\item $\FV(\abs{\avar}{t}) = \FV(t) \setminus \{ \avar \}$;
\item $\FV(\meta{\Avar}{t_1,\dots,t_k}) = \FV(t_1) \cup \dots \cup \FV(t_k)$.
\end{itemize}
That is, $\FV(s)$ contains all variables in $s$ except for those bound by a $\lambda$.
For applicative and first-order terms $s$, this is the set of \emph{all} variables occurring in
$s$.
Note that it is possible for $\FV(s)$ to contain variables in $\Vbound$.

The set of \emph{meta-variables} of a term is inductively defined as follows:
\begin{itemize}
\item $\FMV(\afun) = \emptyset$;
\item $\FMV(\avar) = \{ \avar \}$ if $\avar \in \Vfree$
\item $\FMV(\avar) = \emptyset$ if $\avar \in \Vbound$
\item $\FMV(h(s_1,\dots,s_n)) = \FMV(h) \cup \FMV(s_1) \cup \dots \cup
  \FMV(s_n)$ if $n > 0$;
\item $\FMV(\tuple{s_1}{s_n}) = \FMV(s_1) \cup \dots \cup \FMV(s_n)$.
\item $\FMV(\abs{\avar}{t}) = \FMV(t)$
\item $\FMV(\meta{\Avar}{t_1,\dots,t_k}) = \{ \Avar \} \cup \FMV(t_1) \cup
  \dots \cup \FMV(t_k)$.
\end{itemize}

Note that $\FV(s)$ and $\FMV(s)$ overlap on the variables in $\Vfree$ that occur
in $s$.

A term $s$ is \emph{closed} if $\FV(s) \subseteq \Vfree$.
A term $s$ is \emph{ground} if $\FV(s) = \FMV(s) = \emptyset$.

Since $\FV(s) = \FV(t)$ and $\FMV(s) = \FMV(t)$ whenever $s =_\alpha t$
(Corollary \ref{corr:alphafreevar}), $\FV$ also defines a function on
equivalence classes of terms.

\subsection{Application}\label{subsec:application}
We have defined application as a clause, but it will sometimes be convenient to
have it as an operation.  This is defined as follows:

\begin{definition}
For any term $h(s_1,\dots,s_n) :: \atype \arrtype \btype$ (with $n \geq 0$),
and $t :: \atype$, we let $h(s_1,\dots,s_n) \cdot t$ be the term
$h(s_1,\dots,s_n,t) :: \btype$.
The application operator $\cdot$ is left-associative, so $h(s_1,\dots,s_n)
\cdot t_1 \cdots t_m$ denotes $((h(s_1,\dots,s_n) \cdot t_1) \cdot t_2) \cdots
t_m = h(s_1,\dots,s_n,t_1,\dots,t_m)$.
\end{definition}

We easily obtain the following result;

\begin{lemma}\label{lem:applicative_notation}
The set $\ATerms(\F,\V)$ is the smallest set such that:
\begin{itemize}
\item $\F \cup \V \subseteq \ATerms(\F,\V)$;
\item if $s_1,\dots,s_n \in \ATerms(\F,\V)$ with $n \geq 2$ then
  $\tuple{s_1}{s_n} \in \ATerms(\F,\V)$;
\item if $s,t \in \ATerms(\F,\V)$ and $s : \atype \arrtype \btype$ and $t : \atype$ then
  $s \cdot t \in \ATerms(\F,\V)$.
\end{itemize}
\end{lemma}

\begin{proof}
Trivial.
\end{proof}

Lemma~\ref{lem:applicative_notation} shows that our applicative terms are the
same as applicative terms constructed in the traditional way; however, for
convenience we denote them in a functional notation.
We can similarly see easily that the set of all terms is exactly the smallest
set which includes $\F,\V$ and is closed under abstraction, meta-application,
tuple and this application operator.

Application interacts with $\alpha$-equality as you would expect: $u \cdot s_1
\cdots s_n =_\alpha v \cdot t_1 \cdots t_n$ if and only if $u =_\alpha v$ and
each $s_i =_\alpha t_i$ (Lemma \ref{lem:alphaappl}).

\subsection{Substitution}

A \emph{substitution} is a partial function $\gamma$ that maps:
\begin{itemize}
\item \emph{variables} $\avar :: \atype$ to terms of type $\atype$
\item \emph{meta-variables} $\Avar :: (\atype,k)$ to terms $\abs{\bvar_1 \dots
  \bvar_k}{t}$ of type $\atype$
\end{itemize}
The \emph{domain} $\domain(\gamma)$ of a substitution $\gamma$ is the set of
variables and meta-variables for which $\gamma$ is defined; note that (in
contrast to some of the literature) this includes those $\avar$ with
$\gamma(\avar) = \avar$.
We denote $[x_1:=s_1,\dots,x_n:=s_n]$ for the substitution $\gamma$ with
domain $\{x_1,\dots,x_n\{$ that maps each $x_i$ to $s_i$; finite substitutions
can always be denoted in such a form.
Applying a substitution $\gamma$ to a term $s$, notation $s\gamma$, yields a
new term of the same type, by the following clauses:

\begin{itemize}
\item $\afun\gamma = \afun$;
\item $\avar\gamma = \gamma(\avar)$ if $\avar \in \domain(\gamma)$;
\item $\avar\gamma = \avar$ if $\avar \notin \domain(\gamma)$;
\item $h(s_1,\dots,s_n)\gamma = (h\gamma) \cdot (s_1\gamma) \cdots (s_n\gamma)$
  if $n > 0$;
\item $\tuple{s_1}{s_n}\gamma = \tuple{s_1\gamma}{s_n\gamma}$;
\item $(\abs{\avar}{t})\gamma = \abs{\cvar}{(t ([\avar:=\cvar] \cup [\bvar:=
  \gamma(\bvar) \mid \bvar \in \domain(\gamma) \setminus \{\avar\}]))}$ \\
  for $\cvar$ a \emph{fresh}** variable in $\Vbound$ with the same type as
  $\avar$;
\item $\meta{\Avar}{s_1,\dots,s_k}\gamma = \meta{\Avar}{s_1\gamma,\dots,s_k
  \gamma}$ if $k > 0$ and $\Avar \notin \domain(\gamma)$;
\item $\meta{\Avar}{s_1,\dots,s_k}\gamma = t[\bvar_1:=s_1\gamma,\dots,\bvar_k:=
  s_k\gamma]$ if $k > 0$ and $\Avar \in \domain(\gamma)$ and $\Avar =
  \abs{\bvar_1 \dots \bvar_k}{t}$.
  (Here, if some $\bvar_i = \bvar_j$ for $i < j$, the substitution $[\bvar_1:
  =u_1,\dots,\bvar_n:=u_n]$ maps $\bvar_i$ to $u_j$.)
\end{itemize}
** A \emph{fresh} variable $\cvar$ is one that does not occur in
$\FV(\bvar\gamma)$ for any $\bvar \in \FV(s)$.

Due to the case for meta-application, it is not immediately obvious that these
definitions are well-founded: it is not the case that each step is defined in
terms of an obviously smaller substitution.  However, we can see this in two
steps: substitutions are well-defined when $\domain(\gamma) \subseteq \Vbound$
(since then the meta-application case does not occur); and we can use this as a
prerequisite to see that all substitutions are well-defined (Lemma
\ref{lem:substdefined}).

===================

More critically, these definitions technically do not define functions on terms:
the substitution of an abstraction may lead to any fresh variable being chosen.
Hence, we should perhaps think of these notions as defining \emph{relations}
between terms.  However, they do define a function on \emph{equivalence classes}
(by Corollary \ref{cor:substitutionalpha}):
\begin{itemize}
\item if $s_1 =_\alpha s_2$ and $s_1\gamma = t_1$ and $s_2\gamma' = t_2$ and
  $\gamma(x) =_\alpha \gamma'(x)$ for $x \in \FV(s_1)$, then $t_1 =_\alpha t_2$;
\item if $s_1 =_\alpha s_2$ and $s_1\gamma = t_1$ and $s_2\gamma = t_2$ then
  $t_1 =_\alpha t_2$ (this is a special case of the previous item, but it is
  highlighted since it shows that substitution by a specific $\gamma$ defines a
  function);
\end{itemize}
Hence, the difference is not significant, and we can safely think of
substitution as defining a function.

For two substitutions $\gamma$ and $\delta$, we let $\gamma\delta$ denote the substitution
$[\avar := \gamma(\avar)\delta \mid \avar \in \domain(\gamma)] \cup
[\avar := \delta(\avar) \mid \avar \in \domain(\delta) \setminus \domain(\gamma)]$.
Essentially, applying a substitution $\gamma$ to a term corresponds with replacing each variable
$\avar$ by $\avar\gamma$ (and evaluating the applications of abstractions that are created as a
result), and applying $\gamma\delta$ corresponds to replacing $\avar$ by $(\avar\gamma)\delta$.
We can prove (Lemma \ref{lem:combinesubst}) that $s(\gamma\delta)$ is exactly $(s\gamma)\delta$.

A \emph{renaming} is a substitution $[x_1:=y_1,\dots,x_n:=y_n]$ with $x_1,\dots,x_n$ pairwise
distinct, and $y_1,\dots,y_n$ pairwise distinct.

\subsection{Positions}

The \emph{head positions} of a given term are the paths to specific subterms, defined as follows:

\begin{itemize}
\item $\HeadPositions(h(s_1,\dots,s_n)) = \{ \epsilon, \star 0, \dots, \star (n-1) \} \cup
  \{ i \cdot p \mid \exists i \in \{1,\dots,n\}. p \in \HeadPositions(s_i) \} \cup
  \SubPositions(h)$
\item $\SubPositions(\afun) = \SubPositions(\avar) = \emptyset$
\item $\SubPositions(\abs{\avar}{t}) = \{ 0 \cdot p \mid p \in \HeadPositions(t) \}$
\item $\SubPositions(\meta{\avar}{t_1,\dots,t_k}) = \{ !i \cdot p \mid p \in
  \HeadPositions(t) \}$
\end{itemize}

The \emph{positions} of a given term are all the head positions of a form $p_1 \cdots p_n \epsilon$,
so those that do not contain $\star$.
Note that (head) positions are associated to a term; thus, not every sequence of natural numbers
and exclamation marks is a position.

For a term $s$ and a position $p \in \HeadPositions(s)$, the \emph{subterm of $s$ at position $p$},
denoted $s|_p$, is defined as follows:
\begin{itemize}
\item $s|_\epsilon = s$;
\item $h(s_1,\dots,s_n)|_{\star i} = h(s_1,\dots,s_i)$ (so just $h$ if $i = 0$);
\item $h(s_1,\dots,s_n)|_{i \cdot p} = s_i|_p$ if $1 \leq i \leq n$;
\item $(\abs{\avar}{t})(s_1,\dots,s_n)|_{0 \cdot p} = t|_p$;
\item $\meta{\avar}{t_1,\dots,t_k}(s_1,\dots,s_n)|_{!i \cdot p} = t_i|_p$.
\end{itemize}

If $s|_p$ has the same type as some term $u$, then $s[u]_p$ denotes $s$ with the subterm at position
$p$ replaced by $u$.  Formally, $s[u]_p$ is obtained as follows:
\begin{itemize}
\item $s[u]_\epsilon = u$;
\item $h(s_1,\dots,s_n)[u]_{\star i} = u \cdot s_{i+1} \cdots s_n$;
\item $h(s_1,\dots,s_n)[u]_{i \cdot p} = h(s_1,\dots,s_{i-1},s_i[u]_p,s_{i+1},\dots,s_n)$;
\item $(\abs{\avar}{t})(s_1,\dots,s_n)[u]_{0 \cdot p} = (\abs{\avar}{t[u]_p})(s_1,\dots,s_n)$;
\item $\meta{x}{t_1,\dots,t_k}(s_1,\dots,s_n)[u]_{!i \cdot p} = \meta{x}{t_1,\dots,t_{i-1},t_i[u]_p,t_{i+1},\dots,t_k}(s_1,\dots,s_n)$;
\end{itemize}
Thus, we can find and replace the subterm at a given position.
Note that this is \emph{not} an operation on equivalence classes of terms; for example,
$\abs{x}{x} =_\alpha \abs{y}{y}$, but $(\abs{x}{x})[x]_0 = \abs{x}{x} \not =_\alpha
\abs{y}{x} = (\abs{y}{y})[y]_0$.  Hence, when subterms are considered this should usually be
accompanied by some result that guarantees the operation handles bound variables properly.

Note that we do \emph{not} have the property that if $p$ is a position of $s$, then $p$ is a
position of $s\gamma$ for any substitution $\gamma$; for example, $x(\afun(y))$ has a position
$1\ 1\ \epsilon$, with $x(\afun(y))|_{1\ 1\ \epsilon} = y$.  However,
$x(\afun(y))[x:=\bfun(\identifier{a})] = \bfun(\identifier{a},\afun(y))$, which does not have this
position.  If we ever need it, we could define alternative positions with the property that
if $s|_p = t$ then $(s\gamma)|_p = t\gamma$ for $s$ a proper term.  However, we cannot have both
this, and that positions correspond to the usual definition of positions in the first-order setting.

\subsection{Subterms}

We say that \emph{$t$ is a subterm of $s$}, notation $t \subtermeq s$, if there is some position
$p \in \Positions(s)$ with $t = s|_p$.  This could equivalently be formulated as follows:

\begin{lemma}
$u \subtermeq s$ if and only if one of the following holds:
\begin{itemize}
\item $s = u$;
\item $s = h(s_1,\dots,s_n)$ and $u \subtermeq s_i$ for some $i$;
\item $s = (\abs{x}{t})(s_1,\dots,s_n)$ and $u \subtermeq t$;
\item $s = \meta{x}{t_1,\dots,t_k}(s_1,\dots,s_n)$ and $u \subtermeq t_i$ for some $i$.
\end{itemize}
\end{lemma}

We also observe that $\subtermeq$ is transitive:

\begin{lemma}
If $s \subtermeq t$ and $t \subtermeq q$ then $s \subtermeq q$.
\end{lemma}

This is obvious because if $t = q|_p$ and $s = t|_{p'}$ then $s = q|_{p \cdot p'}$.

It should be noted that in contrast to most definitions of higher-order rewriting, we do \emph{not}
consider, for example, $\afun(x)$ to be a subterm of $\afun(x,y)$.  Instead, we define the
following: \emph{$t$ is a head-subterm of $s$}, notation $t \headsubtermeq s$ if $s|_p = t$ for
$p \in \HeadPositions(s)$.

It should also be noted that if $s =_\alpha t$, it does not follow that $s$ and $t$ have the same
subterms: $\abs{x}{x}$ has a subterm $x$, while $\abs{y}{y}$ does not.  For applicative and
first-order terms, this is not an issue.

Regarding different kinds of terms: the subterms and positions of a first-order term by these
definitions are exactly the subterms and positions as they are usually considered in first-order
term rewriting; however, head-subterms are generally not considered.  For applicative terms,
both subterms and head-subterms are usually referred to as just ``subterms''; we distinguish them
here because doing so is practical for analysis.

\subsection{Rules and rewriting}

A rule $\rho$ is a pair $\ell \arrz r$ of two closed terms with the same type.

For a given set of rules $\Rules$ and $T \subseteq \Terms(\F,\V)$, the reduction relation
$\arr{\Rules,T}$ is given by:
\begin{itemize}
\item if there is some $\ell \arrz r \in \Rules$, a substitution $\gamma$ and a position $p$ of
  $s$ such that $s|_p = \ell\gamma$, then $s \arr{\Rules} s[r\gamma]_p$
\end{itemize}

\medskip
Note that this explicitly includes applications of rules at the head of a subterm.
We will denote $\arr{\Rules}$ for the relation $\arr{\Rules},\Terms(\F,\V)$.

A rule is a \emph{pattern rule} if the left-hand side $\ell$ is a pattern.

We can see that $\arr{\Rules}$ really does define a reduction on the set of terms:

\begin{lemma}
If $s =_\alpha s'$ and $s \arr{\Rules} t$ then exists $t'$ with $t =_\alpha t'$ such that
$s' \arr{\Rules} t'$.
\end{lemma}

\subsection{HOTRSs}

We now have all the ingredients to define a \emph{higher-order term rewriting system (HOTRS)}.

\mysubsection{Abstract Rewriting Systems}

An abstract rewriting system is a pair $(\mathcal{A},\arrz)$ where $\mathcal{A}$ is a set and
$\arrz$ a binary relation on that set.  Properties such as termination and confluence can be
expressed in terms of abstract rewriting systems.

\mysubsection{HOTRSs}

A higher-order term rewriting system (HOTRS) is an abstract rewriting system of the form
$(T,\arr{Rules,T})$ where $T \subseteq \Terms(\F,\V)$ and for all rules $\ell \arrz r \in \Rules$:
both $\ell$ and $r$ are in $T$.

\mysubsection{Examples of HOTRSs}
Many standard forms of term rewriting systems can be expressed as HOTRSs.

A \emph{many-sorted term rewriting system} (MTRS) is a HOTRS with $T = \FOTerms(\F,\V)$ and
$\F,\Rules$ with the following properties:
\begin{itemize}
\item for all $(\afun : \atype) \in \F$: $\order(\atype) \leq 1$;
\item for all $\ell \arrz r \in \Rules$: $\ell$ is not a variable, and $\FV(r) \subseteq \FV(\ell)$.
\end{itemize}
Moreover, in a many-sorted term rewriting system, the reduction relation is exactly $\arr{\Rules}$
(so we do not need to consider $\arr{\Rules,\FOTerms(\F,\V)}$): this is the case because, for
$s \in \FOTerms(\F,V)$ and $\ell \arrz r$ a rule with $\ell,r \in \FOTerms$, we have
$s \arr{\Rules} t$ if and only if there exist $p \in \Positions(s)$ and a substitution $\gamma$
mapping $\Vfree$ to $\FOTerms(\F,V)$ such that $s|_p = \ell\gamma$ and $t = s[r\gamma]_p$.

An \emph{unsorted first-order term rewriting system} (TRS) is a many-sorted term rewriting system
with $\Sorts = \{ \unitsort \}$.

An \emph{applicative term rewriting system} (ATRS) is a HOTRS with $T = \ATerms(\F,\V)$.  Here,
there actually \emph{is} a difference between $\arr{\Rules}$ and $\arr{\Rules,T}$, because in the
former case, a rule $\afun(\avar(\nul)) \arrz \afun(\avar(\symb{1}))$ would reduce
$\afun(\symb{2})$ to itself (through the substitution $\gamma = [\avar:=\abs{\bvar}{\symb{2}}]$),
while in the latter case this would not happen.  However, if all rules in $\Rules$ are
\emph{pattern rules}, then here too $\arr{\Rules}$ and $\arr{\Rules,T}$ are the same.

A \emph{higher-order rewriting system} (HRS) is a HOTRS with $T = \{ s \in \Terms(\F,\V) \mid s$
is in $\eta$-long form$\}$.

A \emph{pattern higher-order rewriting system} (PRS) is a HRS where moreover all elements of
$\Rules$ are pattern rules.

An \emph{algebraic functional system} (AFS) is a HOTRS with the following properties:
\begin{itemize}
\item $\F \supseteq \{ @_{\sigma,\tau} : (\sigma \arrtype \tau) \arrtype \sigma \arrtype \tau
  \mid \sigma,\tau \in \Types \}$;
\item $T = \{ s \in \Terms(\F,\V) \mid \forall t \subtermeq s: t$ does not have the form
  $\avar(t_1,\dots,t_n)\}$;
\item $\Rules \supseteq \Rules_\beta := \{ @_{\sigma,\tau}(\abs{\avar}{s},t) \arrz s[\avar:=t] \mid
  \sigma,\tau \in \Types \wedge s,t \in T \wedge (\avar : \sigma) \in \V \wedge s : \tau \wedge
  t : \sigma \}$.
\end{itemize}
Note that we would have the same relation if $\Rules_\beta$ were replaced by \{
$@_{\sigma,\tau}(\avar,\bvar) \arrz \avar(\bvar) \mid \sigma,\tau \in \Types \}$, but this would not
satisfy the property that rules use only terms in $T$.

\newpage\appendix

\section{Correctness of the unconstrained formalism}

\subsection{$\alpha$-equality}

We first see that $=_\alpha$ is an equivalence relation (on terms, so also on true terms), so that
we can reason modulo it!

\begin{lemma}\label{lem:alphaequiv}
For all terms $s,t,q$, functions $\mu,\xi,\chi : \V \to \N$ and $k \in \N$:
\begin{enumerate}
\item\label{lem:alphaequiv:reflexive}
  $s =_\alpha^{\mu,\mu,k} s$;
\item\label{lem:alphaequiv:symmetric}
  if $s =_\alpha^{\mu,\xi,k} t$ then $t =_\alpha^{\xi,\mu,k} s$;
\item\label{lem:alphaequiv:transitive}
  if $s =_\alpha^{\mu,\xi,k} t$ and $t =_\alpha^{\xi,\chi,k} q$ then $s =_\alpha^{\mu,\chi,k} q$.
\end{enumerate}
\end{lemma}

\begin{proof}
All follow by a straightforward induction on the size of $s$. [Outcommented]
    % 
    % (\ref{lem:alphaequiv:reflexive})
    % If $s = h(s_1,\dots,s_n)$ with $n > 0$ then by IH both $h =_\alpha^{\mu,\mu,k}
    % h$ and $s_i  =_\alpha^{\mu,\mu,k} s_i$ for all $i$, which immediately gives $s
    % =_\alpha^{\mu,\mu,k} s$.
    % We similarly complete by the IH if $s = \meta{\avar}{t_1,\dots,t_k}$, and if
    % $s = \tuple{s_1}{s_n}$.
    % If $s = \afun \in \F$ or $\avar \in \Vfree$ the result is immediate.
    % If $s = \avar \in \Vbound$, then whether $\mu(\avar) = 0$ or $\mu(\avar) =
    % \mu(\avar) > 0$, $\avar =_\alpha^{\mu,\nu,k} \avar$ follows.
    % If $s = \abs{x}{t}$ then by IH $t =_\alpha^{\mu[x:=k],\mu[x:=k],k+1} t$; here also
    % $s =_\alpha^{\mu,\mu,k} s$ follows directly.
    %  
    % (\ref{lem:alphaequiv:symmetric})
    % If $s = a(s_1,\dots,s_n)$ with $n > 0$ then $t = b(t_1,\dots,t_n)$ with $a
    % =_\alpha^{\mu,\xi,k} b$  and $s_i =_\alpha^{\mu,\xi,k} t_i$ for all $i$; by IH
    % then $b =_\alpha^{\xi,\mu,k} a$ and $t_i =_\alpha^{\xi,\mu,k} s_i$ for all $i$,
    % so $t =_\alpha^{\xi,\mu,k} s$.
    % We similarly complete by IH if $s = \meta{\avar}{s_1,\dots,s_k}$ and
    % $t = \meta{\avar}{t_1,\dots,t_k}$, or if $s = \tuple{s_1}{s_n}$ and $t =
    % \tuple{t_1}{t_n}$.
    % If $s \in \F \cup \Vfree$ then $t = s$ and clearly also $t =_\alpha^{\xi,\mu,k}
    % s$.
    % If $s = \avar \in \Vbound$ then $t = \bvar \in \Vbound$ and $\mu(\avar) =
    % \xi(\bvar)$; either $\mu(\avar) = 0$ and $\bvar = \avar$ or $\mu(\avar) > 0$,
    % but in both cases $t =_\alpha^{\xi,\mu,k}$ holds.
    % If $s = \abs{x}{s'}$ then $t = \abs{y}{t'}$ and $s' =_\alpha^{\mu[x:=k],
    % \xi[y:=k],k+1} t'$. By IH also $t' =_\alpha^{\xi[y:=k],[x:=k],k+1} s'$.
    %  
    % (\ref{lem:alphaequiv:transitive})
    % If $s = a(s_1,\dots,s_n)$ with $n > 0$ then $t = b(t_1,\dots,t_n)$ and $q =
    % c(q_1,\dots,q_n)$ with $a =_\alpha^{\mu,\xi,k} b =_\alpha^{\xi,\chi,k} c$ and
    % $s_i =_\alpha^{\mu,\xi,k} t_i =_\alpha^{\xi,\chi,k} q_i$ for all $i$.  By IH
    % therefore $a =_\alpha^{\mu,\chi,k} c$ and $s_i =_\alpha^{\mu,\chi,k} q_i$ for
    % all $i$, allowing for the conclusion $s =_\alpha^{\mu,\chi,k} q$.
    % The cases $s = \meta{\avar}{s_1,\dots,s_k}$ and $s = \tuple{s_1}{s_n}$ follow
    % from IH in the same way.
    % If $s \in \F \cup \Vfree$ then $t = s$ and $q = s$, and $s =_\alpha^{\mu,\chi,k}
    % q$ follows directly.
    % If $s = \avar \in \Vbound$ and $\mu(\avar) = 0$ then $t = \avar$ and $\xi(\avar)
    % = 0$, and therefore $q = \avar$ and $\chi(\avar) = 0$; hence also $s
    % =_\alpha^{\mu,\chi,k} q$.
    % If $s = \avar \in \Vbound$ and $\mu(\avar) > 0$ then $t = \bvar \in \Vbound$ and
    % $\xi(\bvar) = \mu(\avar) > 0$, and therefore $q = \cvar \in \Vbound$ and
    % $\chi(\cvar) = \xi(\bvar) = \mu(\avar) > 0$.
    % Finally, if $s = \abs{\avar}{s'}$ then $t = \abs{\bvar}{t'}$ and $q =
    % \abs{\cvar}{q'}$, with $s' =_\alpha^{\mu[\avar:=k],\xi[\bvar:=k],k+1} t'
    % =_\alpha^{\xi[\bvar:=k],\chi[\cvar:=k],k+1} q'$.
    % By IH then $s' =_\alpha^{\mu[\avar:=k],\chi[cvar:=k],k+1} q'$.
\end{proof}

Choosing $k = 1$ and for $\mu,\xi,\chi$ the function mapping everything to $0$,
we obtain:

\begin{corollary}\label{corr:alphaequiv}
$=_\alpha$ is an equivalence relation.
\end{corollary}

As noted in the text, $\FV$ defines a function on equivalence classes:

\begin{lemma}\label{lem:alphafreevar}
If $s =_\alpha^{\mu,\xi,k} t$ then $\FV(s) \setminus \{ x \mid \mu(x) \neq 0 \} = \FV(t) \setminus \{ x \mid \xi(x) \neq 0 \}$.
\end{lemma}

\begin{proof}
By induction on the size of $s$.
All cases are straightforward. [Outcommented]
    % 
    % For brevity, we denote $M := \{ x \mid \mu(x) \neq 0 \}$ and $X :=  \{ x \mid
    % \xi(x) \neq 0 \}$.
    % \begin{itemize}
    % \item If $s = a(s_1,\dots,s_n)$ then $t = b(t_1,\dots,t_n)$ and both $a
    %   =_\alpha^{\mu,\xi,k} b$ and each $s_i =_\alpha^{\mu,\xi,k} t_i$.
    %   By the induction hypothesis, $\FV(a) \setminus M = \FV(b) \setminus X$ and
    %   $\FV(s_i) \setminus M = \FV(t_i) \setminus X$ for all $i$.
    %   Since $\FV(s) \setminus M = (\FV(a) \setminus M) \cup (\FV(s_1) \setminus M)
    %   \cup \dots \cup (\FV(s_n) \setminus M)$ for any set $M$, and similar for
    %   $\FV(t) \setminus X$, we are done.
    % \item If $s = \afun$ then $t = \afun$ and both sets are empty.
    % \item If $s = \avar \in \Vfree$ then $t = \avar$ and both sets are $\{\avar\}$.
    % \item If $s = \avar \in \Vbound$ and $\mu(\avar) = 0$ then $t = \avar$ and
    %   $\xi(\avar) = 0$ as well.  Therefore $\avar \notin M$ and $\avar \notin X$.
    %   Hence $\FV(s) \setminus M = \{ x \} = \FV(t) \setminus X$.
    % \item If $s = \avar \in \Vbound$ and $\mu(\avar) > 0$ then $t = \bvar \in
    %   \Vbound$ and $\xi(\bvar) = mu(\avar) > 0$ as well.  Therefore $\avar \in M$
    %   and $\bvar \in X$.
    %   Hence $\FV(s) \setminus M = \emptyset = \FV(t) \setminus X$.
    % \item If $s = \abs{x}{s'}$ then $t = \abs{y}{t'}$ and $s' =_\alpha^{\mu[x:=k],
    %   \xi[y:=k],k+1} t'$.  We have $\FV(s) \setminus M = (\FV(s') \setminus \{ x \})
    %   \setminus M = \FV(s') \setminus \{ z \mid \mu[x:=k](z) \neq 0 \}$.  Moreover,
    %   $\FV(t) \setminus X = \FV(t') \setminus \{ z \mid \xi[y:=k](z) \neq 0 \}$.
    %   We complete again by the induction hypothesis.
    % \item If $s = \meta{\Bvar}{s_1,\dots,s_n}$ and $t = \meta{\Bvar}{t_1,\dots,t_n}$,
    %   of if $s = \tuple{s_1}{s_n}$ and $t = \tuple{t_1}{t_n}$, then
    %   then $\FV(s) \setminus M = (\FV(s_1) \cup \dots \FV(s_k)) \setminus M =
    %   (\FV(s_1) \setminus M) \cup \dots \cup (\FV(s_n) \setminus M)$, which by the
    %   induction hypothesis is equal to $(\FV(t_1) \setminus X) \cup \dots \cup
    %   (\FV(t_n) \setminus X) = \FV(t) \setminus X$.
    %   \qedhere
    % \end{itemize}
\end{proof}

And similarly, $\FMV$ defines a function on equivalence classes as well.

\begin{lemma}\label{lem:alphamvar}
If $s =_\alpha^{\mu,\xi,k} t$ then $\FMV(s) = \FMV(t)$.
\end{lemma}

\begin{proof}
By induction on the size of $s$.
All cases are straightforward. [Outcommented]
    % 
    % \begin{itemize}
    % \item If $s = a(s_1,\dots,s_n)$ then $t = b(t_1,\dots,t_n)$ and both $a
    %   =_\alpha^{\mu,\xi,k} b$ and each $s_i =_\alpha^{\mu,\xi,k} t_i$.
    %   By the induction hypothesis, $\FMV(a) = \FMV(b)$ and $\FMV(s_i) = \FMV(t_i)$
    %   for all $i$.
    %   Since $\FMV(s) = \FMV(a) \cup \FMV(s_1) \cup \dots \cup \FMV(s_n)$, and
    %   similar for $\FMV(t)$, we are done.
    % \item We are similarly done quickly if $s = \tuple{s_1}{s_n}$ and $t =
    %   \tuple{t_1}{t_n}$.
    % \item If $s = \afun$ then $t = \afun$ and both sets are empty.
    % \item If $s = \avar \in \Vfree$ then $t = \avar$ and both sets are $\{\avar\}$.
    % \item If $s = \avar \in \Vbound$ then also $t \in \Vbound$, and both sets are
    %   $\emptyset$.
    % \item If $s = \abs{x}{s'}$ then $t = \abs{y}{t'}$ and $s' =_\alpha^{\mu[x:=k],
    %   \xi[y:=k],k+1} t'$.  We have $\FMV(s) = \FMV(s')$ and $\FMV(t) = \FMV(t')$,
    %   and by the induction hypothesis these are equal.
    % \item If $s = \meta{\Bvar}{s_1,\dots,s_n}$ and $t = \meta{\Bvar}{t_1,\dots,t_n}$,
    %   then $\FMV(s) = \{\Bvar\} \cup \FMV(s_1) \cup \dots \cup \FMV(s_n)$ and
    %   $\FMV(t) = \{\Bvar\} \cup \FMV(t_1) \cup \dots \cup \FMV(t_n)$.  We complete
    %   once more by induction on each $s_i$.
    %   \qedhere
    % \end{itemize}
\end{proof}

Choosing $k = 1$ and $\mu,\xi$ the constant $0$-functions again, we obtain:

\begin{corollary}\label{corr:alphafreevar}
If $s =_\alpha t$ then $\FV(s) = \FV(t)$ and $\FMV(s) = \FMV(t)$.
\end{corollary}

Next, we define some helper results that have little meaning on their own, but
will prove useful when reasoning about $\alpha$-equivalence (especially in
combination with substitution).

\begin{lemma}\label{lem:alphaincrease}
Let $\mu,\xi$ be such that $\mu(\avar) < k$ and $\xi(\avar) < k$ for all
$\avar \in \Vbound$. \\
Then if $s =_\alpha^{\mu,\xi,k} t$ we have $s =_\alpha^{\mu,\xi,k+1} t$.
\end{lemma}

\begin{proof}
For an integer $0 < n \leq k$, let:
\[\mathit{up}_n(i) = \left\{
\begin{array}{ll}
i & \text{if}\ i \leq n\ \text{or}\ i \geq k+1 \\
i+1 & \text{if}\ n < i \leq k \\
\end{array}
\right.
\]
Let $\mu_n(x) = \mathit{up}_n(\mu(x))$ and $\xi_n(x) = \mathit{up}_n(x)$.
We will prove by induction on $s$ that for fixed $n > 0$, all $k \geq n$:
if $s =_\alpha^{\mu,\xi,k} t$ then $s =_\alpha^{\mu_n,\xi_n,k+1} t$.
The result then follows by choosing $n = k$, since $\mathit{up}_k$ is the
identity and therefore $\mu_k = \mu$ and $\xi_k = \xi$.
The proof is straightforward. We only show the most difficult case.
[The rest is outcommented]

\begin{itemize}
    % \item If $s = a(s_1,\dots,s_m)$ then $t = b(t_1,\dots,b_m)$ with
    %   $a =_\alpha^{\mu,\xi,k} b$ and each $s_i =_\alpha^{\mu,\xi,k} t_i$.
    %   By the IH, $a =_\alpha^{\mu_n,\xi_n,k+1} b$ and each
    %   $s_i =_\alpha^{\mu_n,\xi_n,k+1} t_i$; hence, the result follows.
    % \item If $s = \meta{\Avar}{s_1,\dots,s_m}$ or $s = \tuple{s_1}{s_m}$, the result
    %   follows similarly easily.
    % \item If $s = \afun$ or $s = \avar \in \Vfree$, then $t = s$ and the result
    %   immediately follows.
    % \item If $s = \avar \in \Vbound$ and $\mu(\avar) = 0$ then $t = \avar$ and
    %   $\xi(\avar) = 0$; then also $\mu_n(\avar) = \xi_n(\avar) = 0$, and we are
    %   done.
    % \item If $s = \avar \in \Vbound$ and $\mu(\avar) > 0$ then $t = \bvar \in
    %   \Vbound$ and $\xi(\avar) = \mu(\avar) > 0$.  But then, whatever the value of
    %   $\mu(\avar)$, we have that $\mu_n(\avar) = \mathit{up}_n(\mu(\avar)) =
    %   \mathit{up}_n(\xi(\bvar)) = \xi_n(\avar)$ as well, giving the required result.
\item If $s = \abs{x}{s'}$ then $t = \abs{y}{t'}$ and $s' =_\alpha^{\mu[x:=k],
  \xi[y:=k],k+1} t'$.
  Write $\mu' := \mu[x:=k]$ and $\xi' := \xi[y:=k]$. Then we have:
  \begin{itemize}
  \item for all variables $z$: $\mu'(z) < k+1$ (since either $\mu'(z) = \mu(z)
    < k$, or $\mu'(z) = k$ if $z = x$);
  \item for all variables $z$: $\xi'(z) < k+$ (by the same reasoning);
  \item $k+1 > k \geq n$. \\
  \end{itemize}
  Hence, we can apply the induction hypothesis, to deduce: $s' =_\alpha^{\mu'_n,
  \xi'_n,k+2} t'$.

  But now observe that $\mu'_n = \mu_n[x:=k+1]$:
  \begin{itemize}
  \item for $z \neq x$: $\mu'_n(z) = \mathit{up}_n(\mu'(z)) =
    \mathit{up}_n(\mu(z)) = \mu_n(z) = \mu_n(z)[x:=k+1]$
  \item for $z = x$: $\mu'_n(z) = \mathit{up}_n(\mu'(z)) = \mathit{up}_n(k) =
    k+1 = \mu_n[x:=k+1](x)$. \\
  \end{itemize}
  And similarly, $\xi'_n = \xi_n[y:=k+1]$.

  So we have: $s' =_\alpha^{\mu_n[x:=k+1],\xi_n[y:=k+1],k+2} t'$, which
  exactly gives $s =_\alpha^{\mu_n,\xi_n,k+1} t$.
  \qedhere
\end{itemize}
\end{proof}

\begin{lemma}\label{lem:alphaunusedvar}
Suppose $s =_\alpha^{\mu,\xi,k} t$ and $\mu'(x) = \mu(x)$ for all $x \in \FV(s)$, and $\xi'(y) = \xi(y)$ for all $y \in \FV(t)$.
Then $s =_\alpha^{\mu',\xi',k} t$.
\end{lemma}

\begin{proof}
By induction on the size of $s$.
All cases are straightforward. [Outcommented]
    % \begin{itemize}
    % \item If $s = a(s_1,\dots,s_n)$ then $t = b(t_1,\dots,t_n)$ and
    %   $a =_\alpha^{\mu,\xi,k} b$ and each $s_i =_\alpha^{\mu,\xi,k} t_i$.
    %   By the induction hypothesis, $a =_\alpha^{\mu',\xi',k} b$ and each $s_i
    %   =_\alpha^{\mu',\xi',k} t_i$.
    %   This immediately gives $s =_\alpha^{\mu',\xi',k} t$ as required.
    % \item We similarly conclude immedately if $s = \meta{\ameta}{t_1,\dots,t_k}$ or
    %   $s = \tuple{s_1}{s_n}$.
    % \item If $s = \afun$ or $s = x \in \Vfree$ then $t = s$ and the result
    %   immediately follows.
    % \item If $s = \avar \in \Vbound$ and $\mu(\avar) = 0$ then $t = \avar$ and 
    %   $\xi(\avar) = 0$.  Since clearly $\avar \in \FV(s)$ and $\avar \in \FV(t)$ we
    %   have $\mu'(x) = \xi'(y) = 0$ as well.
    %   Hence indeed $s =_\alpha^{\mu',\xi',k} t$.
    % \item If $s = \avar \in \Vbound$ and $\mu(\avar) > 0$ then $t = \bvar \in
    %   \Vbound$ and $\xi(\bvar) > 0$.  Since clearly $\avar \in \FV(s)$ and $\bvar
    %   \in \FV(t)$ we have $\mu'(\avar) = \mu(\avar) = \xi(\bvar) = \xi'(\bvar) > 0$
    %   as well.
    % \item If $s = \abs{x}{s'}$ then $t = \abs{y}{t'}$ and $s' =_\alpha^{\mu[x:=k],\
    %   xi[y:=k],k+1} t'$.  Now, for all $z \in \FV(s')$: either $z = x$ and
    %   $\mu'[x:=k](z) = k = \mu[x:=k](z)$, or $z \in \FV(s)$ and $\mu[x:=k](z) =
    %   \mu(z) = \mu'(z) = \mu'[x:=k](z)$ by assumption.
    %   Similarly, for all $z \in \FV(t')$ we have $\xi'[y:=k](z) = \xi[y:=k](z)$.
    %   Hence we can apply the induction hypothesis on $s'$ to obtain
    %   $s' =_\alpha^{\mu'[x:=k],\xi'[y:=k],k+1} t'$.
    %   This immediately implies $s =_\alpha^{\mu',\xi',k} t$.
    %   \qedhere
    % \end{itemize}
\end{proof}

We also observe that $\alpha$-equality interacts well with application:

\begin{lemma}\label{lem:alphaappl}
Let $s = u \cdot s_1 \cdots s_n$ and $t = v \cdot t_1 \cdots t_n$.
Then $s =_\alpha^{\mu,\xi,k} t$ if and only if $u =_\alpha^{\mu,\xi,k} v$ and
$s_i =_\alpha^{\mu,\xi,k} t_i$ for $1 \leq i \leq n$.
\end{lemma}

\begin{proof}
This is easily seen by inspecting the definitions.
    % 
    % First suppose $u =_\alpha^{\mu,\xi,k} v$ and each $s_i =_\alpha^{\mu,\xi,k}
    % t_i$.  Consider the form of $u$; we can always write $u = a(u_1,\dots,u_m)$
    % with $m \geq 0$, and since $u =_\alpha^{\mu,\xi,k} v$, necessarily $v = b(v_1,
    % \dots,v_m)$ with $a =_\alpha^{\mu,\xi,k} b$ and each $u_i =_\alpha^{\mu,\xi,k}
    % v_i$.  But then $s = a(u_1,\dots,u_m,s_1,\dots,s_n)$ and $t = b(v_1,\dots,v_m,
    % t_1,\dots,t_n)$; it immediately follows that $s =_\alpha^{\mu,\xi,k} t$.
    % 
    % Alternatively, suppose that $s =_\alpha^{\mu,\xi,k} t$.  Again denote $u =
    % a(u_1,\dots,u_m)$.  Then necessarily $s = a(u_1,\dots,u_m,s_1,\dots,s_n)$.  Also
    % write $v = b(v_1,\dots,v_k)$; then $t = b(v_1,\dots,v_k,t_1,\dots,t_n)$.  But by
    % definition of $s =_\alpha^{\mu,\xi,k} t$ necessarily $m + n = k + n$; that is,
    % $m = k$.  Moreover, necessarily $a =_\alpha^{\mu,\xi,k} b$, each $u_i
    % =_\alpha^{\mu,\xi,k} b$, and each $s_i =_\alpha^{\mu,\xi,k} t_i$.  But then
    % also $u = a(u_1,\dots,u_m) =_\alpha^{\mu,\xi,k} b(v_1,\dots,v_m) = v$.
\end{proof}

\subsection{Well-definedness of substitution}

\newcommand{\subrel}[1]{\mathsf{subst}(#1)}

To avoid the suggestive equality notation, we will reformulate the notion of
substitution as a relation.  First, let us abuse notation and write
$\gamma(x) = x$ for all variables $x \notin \domain(\gamma)$.  We now let
substitution be the smallest relation generated by the following rules:

\begin{enumerate}
\item\label{subst:func} $\subrel{\afun,\gamma,\afun}$
\item\label{subst:var} $\subrel{\avar,\gamma,\gamma(\avar)}$ (which is now
  defined whether $\avar \in \domain(\gamma)$ or not)
\item\label{subst:appl} $\subrel{h(s_1,\dots,s_n), \gamma, u \cdot t_1 \cdots
  t_n)}$ if $\subrel{h,\gamma,u}$ and $\subrel{s_i,\gamma,t_i}$ for
  $1 \leq i \leq n$;
\item\label{subst:tuple} $\subrel{\tuple{s_1}{s_n},\gamma,\tuple{t_1}{t_n}}$ if
  $\subrel{s_i,\gamma,t_i}$ for $1 \leq i \leq n$;
\item\label{subst:abs} $\subrel{\abs{\avar}{s},\gamma,\abs{\cvar}{q}}$ if
  $\cvar$ has the same type as $\avar$ and:
  \begin{itemize}
  \item $\cvar \notin \FV(\gamma(\bvar))$ for any $\bvar \in
    \FV(\abs{\avar}{s}) \cup (\FMV(\abs{\avar}{s}) \cap \domain(\gamma))$
  \item $\subrel{s,[\avar:=\cvar] \cup [\bvar := \gamma(\bvar) \mid \bvar \in
    \domain(\gamma) \setminus \{\avar\}],q}$.
  \end{itemize}
\item\label{subst:boringmeta} $\subrel{\meta{\Avar}{s_1,\dots,s_n},\gamma,
  \meta{\Avar}{t_1,\dots,t_n}}$ if $\Avar \notin \domain(\gamma)$ and
  $\subrel{s_1,\gamma,t_i}$ for $1 \leq i \leq n$
\item\label{subst:meta} $\subrel{\meta{\Avar}{s_1,\dots,s_n},\gamma,q}$ if
  $\Avar \in \domain(\gamma)$ and $\gamma(\Avar) = \abs{\bvar_1 \dots \bvar_n}{
  t}$ and there exist $t_1,\dots,t_n$ such that:
  \begin{itemize}
  \item $\subrel{s_i,\gamma,t_i}$ for $q \leq i \leq n$
  \item $\subrel{t,[\bvar_1:=t_1,\dots,\bvar_n:=t_n],q}$
  \end{itemize}
\end{enumerate}
It is easy to see that this defines the same relation as substitution in the
main text.  The ``smallest'' relation with these properties is well-defined
because the definition can be seen as a fixpoint: any additional relation
$\subrel{s,\gamma,t}$ that is discovered cannot change the relations that were
previously discovered.
Beyond that, we have:

\begin{lemma}\label{lem:substdefined}
For every term $s$ and substitution $\gamma$ there exists $q$ with
$\subrel{s,\gamma,q}$.
\end{lemma}

\begin{proof}
We first prove that the lemma holds when limited to $\gamma$ where
$\domain(\gamma) \subseteq \Vbound$; this follows by a straightforward induction
on the size of $s$ because in the meta-application case $\meta{\Avar}{s_1,\dots,
s_k}$, the meta-variable $\Avar$ is not in $\domain(\gamma)$.
Having this property, the full lemma follows, again by induction on the size of
$s$: the final case where $s = \meta{\Avar}{s_1,\dots,s_k}$ follows by
observing that we have already proved the existence of some $q$ with
$\subrel{t,[\bvar_1:=t_1,\dots,\bvar_n:=t_n],q}$ in the first step.

We first consider the lemma statement limited to $\gamma$ where
$\domain(\gamma) \subseteq \Vbound$.  This we prove by induction on the size of
$s$.  Consider the form of $s$:
\begin{itemize}
\item If $s = \afun$ we can choose $q := \afun$.
\item If $s = \avar$ we can choose $q := \gamma(\avar)$.
\item If $s = h(s_1,\dots,s_n)$ with $n > 0$ then by the induction hypothesis
  there exist $u$ such that $\subrel{h,\gamma,u}$ and $t_1,\dots,t_n$ such that
  $\subrel{s_i,\gamma,t_i}$ for all $i$.  As application (the $\cdot$ operator)
  is unambiguously defined, we can choose $q := u \cdot t_1 \cdots t_n$.
\item If $s = \tuple{s_1}{s_n}$ then by the induction hypothesis there exist
  $u_1,\dots,u_n$ such that $\subrel{s_i,\gamma,u_i}$ for all $i$; we can
  choose $q := \tuple{u_1}{u_n}$.
\item If $s = \abs{\avar}{s'}$ then first note that a suitable $\cvar$ can
  always be found, as $\Vbound$ has infinitely many variables of all types and
  $\bigcup_{\bvar \in \FV(s)} \FV(\gamma(\bvar))$ is finite.  Then, writing
  $\delta := [\avar:=\cvar] \cup [\bvar:=\gamma(\bvar) \mid \bvar \in
  \domain(\gamma) \setminus \{\avar\}]$, there exists $q'$ such that
  $\subrel{s',\delta,q'}$ by the induction hypothesis on $s'$.  We complete with
  $q := \abs{\cvar}{q}$.
\item If $s = \meta{\Avar}{s_1,\dots,s_k}$, then by the induction hypothesis we
  find $t_1,\dots,t_k$ with $\subrel{s_i,\gamma,t_i}$ for all $i$.  Since, by
  assumption, $\Avar \notin \domain(\gamma)$, we can choose $q :=
  \meta{\Avar}{t_1,\dots,t_k}$.
\end{itemize}
Next, we prove the full lemma statement. The cases where $s$ is a constant,
variable, application, tuple or abstraction are exactly the same as above, as is
the case when $s = \meta{\Avar}{s_1,\dots,s_k}$ with $\Avar \notin \domain(
\Avar)$.  So assume $s = \meta{\Avar}{s_1,\dots,s_k}$ with $\Avar \in
\domain(\Avar)$; then by definition of substitution we can write
$\gamma(\Avar) = \abs{\bvar_1 \dots \bvar_k}{t}$.  By the induction hypothesis
there exist $t_1,\dots,t_k$ such that $\subrel{s_i,\gamma,t_i}$; let $\delta :=
[\bvar_1:=t_1,\dots,\bvar_k:=t_k]$.  The domain of $\delta$ only contains
variables in $\Vbound$, so by the limited lemma statement, there exists $q$
such that $\subrel{t,\delta,q}$.  But then $\subrel{s,\gamma,q}$ holds.
\end{proof}

Towards the second result for well-definedness, that substitution defines a
function on equivalence classes, we specify the following lemma.  This is
formulated more generally than we need to obtain an easier induction.

\begin{lemma}\label{lem:substitutionalpha}
Let $k$ be an integer, and $\nu,\chi : \Vbound \mapsto \{0,\dots,k\}$.
Assume given terms $s,s'$, an integer $p$, mappings $\mu,\xi : \Vbound \mapsto
\{0,\dots,p\}$ and substitutions $\gamma,\gamma'$ such that:
\begin{itemize}
\item for all $x \in \FV(s)$ with $\mu(x) = 0$ we have:
  $\gamma(x) =_\alpha^{\nu,\chi,k+1} \gamma'(x)$
\item for all $x \in \FV(s)$, $y \in \FV(s')$ such that $\mu(x) = \xi(y) > 0$
  we have: $\gamma(x) =_\alpha^{\nu,\chi,k+1} \gamma'(y)$
\item $\domain(\gamma) \cap \M = \domain(\gamma') \cap \M$ and
  $\gamma(\Avar) =_\alpha^{\nu,\chi,k+1} \gamma'(\Avar)$ for all
  $\Avar \in \domain(\gamma) \cap \M$
\item $s =_\alpha^{\mu,\xi,p+1} s'$
\end{itemize}
Assume given $t,t'$ such that $\subrel{s,\gamma,t}$ and $\subrel{s',\gamma',
t'}$.  Then we have $t =_\alpha^{\nu,\chi,k+1} t'$.
\end{lemma}

\begin{proof}
We prove the statement by induction on the definition of $\subrel$ (so, on the
proof of Lemma \ref{lem:substdefined}).
Although the proof is not overly complex, it is quite long, and requires
multiple applications of both Lemmas \ref{lem:alphafreevar} and
\ref{lem:alphaappl}, as well as an application of both Lemmas
\ref{lem:alphaincrease} and \ref{lem:alphaunusedvar}.
    % 
    % Consider the shape of $s$.
    % \begin{itemize}
    % \item Suppose $s = \afun$.  Then $s' = t = t' = \afun$ and indeed
    %   $t =_\alpha^{\nu,\chi,k+1} t'$.
    % \item Suppose $s = \avar$ with $\mu(\avar) = 0$.  Then $s' = \avar$ and $t =
    %   \gamma(x) =_\alpha^{\nu,\chi,k+1} \gamma'(x) = t'$ by assumption.
    % \item Suppose $s = \avar$ with $\mu(\avar) > 0$.  Then $s' = \bvar$ with
    %   $\xi(\bvar) = \mu(\avar)$, so also by assumption $t = \gamma(x)
    %   =_\alpha^{\nu,\chi,k+1} \gamma'(y) = t'$.
    % \item Suppose $s = h(s_1,\dots,s_n)$ with $n > 0$.
    %   Then $s' = h'(s_1',\dots,s_n')$ with $h =_\alpha^{\mu,\xi,p+1} h'$ and $s_i
    %   =_\alpha^{\mu,\xi,p+1} s_i'$ for all $i$. \\
    %   From $\subrel{s,\gamma,t}$ we obtain: $t = u \cdot t_1 \cdots t_n$ with
    %   $\subrel{h,\gamma,u}$ and $\subrel{s_i,\gamma,t_i}$ for all $i$.
    %   From $\subrel{s',\gamma',t'}$ we obtain: $t' = u' \cdot t_1' \cdots t_n'$ with
    %   $\subrel{h',\gamma',u'}$ and $\subrel{s_i',\gamma',t_i'}$ for all $i$. \\
    %   Hence, by the induction hypothesis on $\subrel{h,\gamma,u}$ and $\subrel{h',
    %   \gamma',u'}$ we have: $u =_\alpha^{\nu,\chi,k+1} u'$.
    %   Similarly, by the IH on each $\subrel{s_i,\gamma,t_i}$ and $\subrel{s_i',
    %   \gamma',t_i'}$ we have: $t_i =_\alpha^{\nu,\chi,k+1} t_i'$. \\
    %   By Lemma \ref{lem:alphaappl} we obtain the required conclusion
    %   $t =_\alpha^{\nu,\chi,k+1} t'$.
    % \item Suppose $s = \tuple{s_1}{s_n}$.  Then we complete in the same way.
    % \item Suppose $s = \meta{\Avar}{s_1}{s_k}$ with $\Avar \notin \domain(\gamma)$.
    %   Then also $\Avar \notin \domain(\gamma')$ since $\domain(\gamma) \cap \M =
    %   \domain(\gamma') \cap \M$ by assumption.  Hence, we again complete as above.
    % \item Suppose $s = \abs{x}{u}$. Then $s' = \abs{x'}{u'}$ and
    %     $t = \abs{z}{q}$ and $t' = \abs{z'}{q'}$, where:
    %   \begin{itemize}
    %   \item $u =_\alpha^{\mu[x:=p+1],\xi[x':=p+1],p+2} u'$;
    %   \item $z \notin \FV(\gamma(y))$ for any $y \in \FV(s)$ and
    %     $z' \notin \FV(\gamma'(y))$ for any $y \in \FV(s')$;
    %   \item $\subrel{u,[x:=z] \cup [y:=\gamma(y) \mid y \in \domain(\gamma)
    %     \setminus \{x\},q}$ and \\
    %     $\subrel{u',[x':=z'] \cup [y:=\gamma'(y) \mid y \in \domain(\gamma')
    %     \setminus \{x'\},q'}$. \\
    %   \end{itemize}
    %   Now, let $\delta := [x:=z] \cup [y:=\gamma(y) \mid y \in \domain(\gamma)
    %   \setminus \{ x \}]$ and $\delta' := [x':=z'] \cup [y:=\gamma'(y) \mid y \in
    %   \domain(\gamma') \setminus \{ x' \}]$, so we have
    %   $\subrel{u,\delta,q}$ and $\subrel{u',\delta',q'}$.
    %   Let $\mu_b := \mu[x:=p+1]$ and $\xi_b := \xi[x':=p+1]$ and $\nu_b :=
    %   \nu[z:=k+1]$ and $\chi_b := \chi[z':=k+1]$.
    %   We apply the induction hypothesis on $\subrel{u,\delta,q}$ and $\subrel{u',
    %   \delta',q'}$. (**)
    %   This gives us that $q =_\alpha^{\nu_b,\chi_b,k+2} q'$, and therefore $t =
    %   \abs{z}{q} =_\alpha^{\nu,\chi,k+1} \abs{z'}{q'} = t'$ as required.
    % 
    %   (**) To see that we may apply the induction hypothesis to obtain this
    %   conclusion, we observe that $\nu_b,\chi_b$ are functions in $\Vbound \to
    %   \{0,\dots,k+1\}$, that $\mu_b,\xi_b$ are functions in $\Vbound \to \{0,\dots,
    %   p+1\}$, and that $u =_\alpha^{\mu_b,\xi_b,p+2} u'$ as observed above.
    %   For the substitutions, we must show that for all $y \in \FV(u)$:
    %   \begin{itemize}
    %   \item if $\mu_b(y) = 0$ then $\delta(y) =_\alpha^{\nu_b,\chi_b,k+2}
    %     \delta'(y)$;
    %   \item if $\mu_b(y) > 0$ and $y' \in \FV(u')$ is such that $\mu_b(y) =
    %     \xi_b(y')$ then $\delta(y) =_\alpha^{\nu_b,\chi_b,k+2} \delta'(y')$. \\
    %   \end{itemize}
    %   So let $y \in \FV(u)$; if $\mu_b(y) = 0$ then let $y' := y$, otherwise let $y'$ be such that $\mu_b(y) = \xi_b(y')$.
    % 
    %   If $y = x$, then $\mu_b(y) = p + 1$, so we should show the second case.
    %   Since $\xi$ maps to $\{0,\dots,p\}$ and $\xi_b(x') = p + 1$, necessarily $y' =
    %   x'$.
    %   Hence we must show: $z = \delta(x) =_\alpha^{\nu_b,\chi_b,k+2} \delta'(x') = z'$.
    %   Since $\nu_b(z) = k + 1 = \chi_b(z')$ this clearly holds.
    % 
    %   Alternatively, if $y \neq x$, then $\mu_b(y) = \mu(y) \leq p$.
    %   \begin{itemize}
    %   \item If $\mu_b(y) > 0$ and $y' \in \FV(u')$ has $\mu_b(y) = \xi_b(y')$, then
    %     note that $y' \neq x'$, as $\xi_b(x') = p+1 > \mu_b(y)$.
    %     Hence, $\xi_b(y') = \xi(y')$.
    %   \item Otherwise, $y' = y$ by definition and since $y \in \FV(u) \setminus
    %     \{ x \} = \FV(s)$ we can apply Lemma \ref{lem:alphafreevar} to obtain $y \in
    %     \FV(s')$ and $\xi(y) = 0$; so $y \in \FV(u')$ and $y \neq x'$, hence
    %     $\xi_b(y) = \xi(y) = 0$.\\
    %   \end{itemize}
    %   Hence, either way, $y' \neq x'$ and $\xi_b(y') = \xi(y')$.
    %   Then also $\delta(y) = \gamma(y)$ and $\delta'(y') = \gamma'(y')$, and by the
    %   assumptions on $\gamma,\gamma'$ we have:
    %   $\delta(y) =_\alpha^{\nu,\chi,k+1} \delta'(y)$.
    %   Hence by Lemma \ref{lem:alphaincrease}, $\delta(y) =_\alpha^{\nu,\chi,k+2}
    %   \delta'(y)$ (*A). \\
    %   Now, since $y \in \FV(u)$ and $y \neq x$, we have $y \in \FV(s)$.  Similarly,
    %   $y' \in \FV(s')$.
    %   By the freshness condition on $z,z'$ we have: $z \notin \FV(\gamma(y)) =
    %   \FV(\delta(y))$, and $z' \notin \FV(\delta'(y'))$.
    %   But then by (*A) and Lemma \ref{lem:alphaunusedvar},
    %   $\delta(w) =_\alpha^{\nu_b,\chi_b,k+2} \delta'(w')$ as required.
    % \item Finally, suppose $s = \meta{\Avar}{s_1,\dots,s_m}$ with $\Avar \in
    %   \domain(\gamma)$.  Then:
    %   \begin{itemize}
    %   \item $s' = \meta{\Avar}{s_1',\dots,s_m'}$ with $s_i =_\alpha^{\mu,\xi,p+1}
    %     s_i'$ for all $i$;
    %   \item $\Avar \in \domain(\gamma')$ and $\gamma(\Avar) =_\alpha^{\nu,
    %     \chi,k+1} \gamma'(\Avar)$;
    %   \item we can write $\gamma(\Avar) = \abs{\avar_1 \dots \avar_m}{u}$ and
    %     $\gamma'(\Avar) = \abs{\bvar_1 \dots \bvar_m}{w}$
    %   \item there exist $t_1,\dots,t_m$ such that $\subrel{s_i,\gamma,t_i}$ for
    %     all $i$ and $\subrel{u,\delta,t}$ for $\delta = [\avar_1:=t_1,\dots,
    %     \avar_m:=t_m]$
    %   \item there exist $t_1',\dots,t_m'$ such that $\subrel{s_i',\gamma,t_i'}$ for
    %     all $i$ and $\subrel{w,\delta',t}$ for $\delta' = [\bvar_1:=t_1',\dots,
    %     \bvar_m:=t_m']$ \\
    %   \end{itemize}
    %   We conclude $t =_\alpha^{\nu,\chi,k+1} t'$ by the induction hypothesis on
    %   $\subrel{u,\delta,t}$ and $\subrel{w,\delta',t'}$.  To see that we are indeed
    %   allowed to apply the induction hypothesis, we observe:
    %   \begin{itemize}
    %   \item Denote $\mu' = \nu[\avar_1:=k+1,\dots,\avar_m:=k+m]$ (where, if
    %     $\avar_i = \avar_j$ for $i < j$ then $\mu'(\avar_i) = k+j$), and similarly
    %     $\xi' = \chi[\bvar_1:=k+1,\dots,\bvar_m:=k+m]$; then from $\gamma(\Avar)
    %     =_\alpha^{\nu,\chi,k+1} \gamma'(\Avar)$ we obtain:
    %     $u =_\alpha^{\mu',\xi',k+m+1} w$.
    %   \item For all $\cvar \in \FV(u)$ with $\mu'(\cvar) = 0$ we clearly do not have
    %     $\cvar = \avar_i$ for any $i$.  So necessarily $\cvar \in
    %     \FV(\gamma(\Avar))$ and $\nu(\cvar) = 0$ too; by Lemma
    %     \ref{lem:alphafreevar} therefore $\cvar \in \FV(\gamma'(\Avar))$ and
    %     $\chi(\cvar) = 0$. But if some $\bvar_i = \cvar$, then $\cvar \in
    %     \FV(\gamma'(\Avar))$ would not hold; hence, this is not the case, so
    %     $\delta(\cvar) = \delta'(\cvar) = \cvar$.  We indeed have $\cvar
    %     =_\alpha^{\nu,\chi,k+1} \cvar$.
    %   \item For all $\cvar \in \FV(u),\cvar' \in \FV(v)$ with $0 < \mu'(\cvar) =
    %     \xi'(\cvar') \leq k$ we clearly do not have $\cvar = \avar_i$ or $\cvar' =
    %     \bvar_i$ for any $i$.  So necessarily $\cvar \in \FV(\gamma(\Avar))$ and
    %     $\cvar' \in \FV(\gamma'(\Avar))$, and $\nu(\cvar) = \mu'(\cvar) =
    %     \xi'(\cvar') = \chi(\cvar) > 0$.  Hence, $\cvar\delta = \cvar =_\alpha^{
    %     \nu,\chi,k+1} \cvar' = \cvar'\delta'$ .
    %   \item For all $\cvar \in \FV(u),\cvar' \in \FV(v)$ with $k < \mu'(\cvar) =
    %     \xi'(\cvar') =: j$, we must have $\cvar = \avar_{j-k}$ and $\cvar' =
    %     \bvar_{j-k}$.  Moreover, by definition $\avar_l \neq \avar_{j-k}$ for any
    %     $l > j-k$; and $\bvar_l \neq \bvar_{j-k}$ for any $l > j-k$  Hence,
    %     $\delta(\cvar) = t_{j-k}$ and $\delta'(\cvar') = t'_{j-k}$.
    %     By the induction hypothesis for each $\subrel{s_i,\gamma,t_i}$ we have
    %     $\delta(\cvar) = t_{j-k} =_\alpha^{\nu,\chi,k+1} t'_{j-k}$.
    %   \item $\domain(\delta) \cap \M = \domain(\delta') \cap \M = \emptyset$
    %   \end{itemize}
    % \end{itemize}
\end{proof}

\begin{corollary}\label{cor:substitutionalpha}
If $s =_\alpha s'$ and $\subrel{s,\gamma,t}$ and $\subrel{s',\gamma',t'}$ and
  $\gamma(x) =_\alpha \gamma'(x)$ for all $x \in \FV(s)$, then $t =_\alpha t'$.
\end{corollary}

\subsection{Some results on $\alpha$-equivalence and substitution}

We will now prove several useful results regarding substitution that will often
be silently used in proofs.

\begin{lemma}\label{lem:appsubstitute}
We have:
\begin{enumerate}
\item
  If $\subrel{s_i,\gamma,t_i}$ for $0 \leq i \leq n$, then $\subrel{s_0 \cdot
  s_1 \cdots s_n,\gamma,t_0 \cdot t_1 \cdots t_n}$.
\item
  If $\subrel{s_0 \cdot s_1 \cdots s_n,\gamma,t}$, then we can write $t = t_0 \cdot t_1 \cdots t_n$
  with $\subrel{s_i,\gamma,t_i}$ for $0 \leq i \leq n$.
\end{enumerate}
\end{lemma}

\begin{proof}
Both sides are immediate by definition.
    % 
    % In the following, let $s = s_0 \cdot s_1 \cdots s_n$, and write $s_0 = h(u_1,
    % \dots,u_k)$ with $k \geq 0$ (this is always possible), so $s = h(u_1,\dots,u_k,
    % s_1,\dots,s_n)$.
    % 
    % First suppose $\subrel{s_i,\gamma,t_i}$ for $0 \leq i \leq n$.
    % Then (by definition of $\subrel{}$) necessarily $t_0 = w \cdot v_1 \cdots v_k$,
    % where $\subrel{h,\gamma,w}$ and $\subrel{u_i,\gamma,v_i}$ for $1 \leq i \leq k$.
    % Hence, writing $t = t_0 \cdot t_1 \cdots t_n$ we have $t = (w \cdot v_1 \cdots
    % v_k) \cdot t_1 \cdots t_n$.  Since $\cdot$ is left-associative, this is exactly
    % $w \cdot v_1 \cdots v_k \cdot t_1 \cdots t_n$, and $\subrel{h(u_1,\dots,u_k,s_1,
    % \dots,s_n),\gamma,t}$ follows immediately.
    % 
    % Next suppose $\subrel{s,\gamma,t}$.  Then $t = w \cdot v_1 \cdots v_k \cdot t_1
    % \cdots t_n$ with $\subrel{h,\gamma,w}$ and $\subrel{u_i,\gamma,v_i}$ for $1
    % \leq i \leq k$ and $\subrel{s_i,\gamma,t_i}$ for $1 \leq i \leq n$.  Write $t_0
    % := w \cdot v_1 \cdots v_k$; then also $t = t_0 \cdot t_1 \cdots t_n$, and
    % $\subrel{s_0,\gamma,t_0}$ follows immediately.
\end{proof}

Hence, viewing substitution as a function rather than a relation, we have shown
that $(s_0 \cdot s_1 \cdots s_n)\gamma = (s_0\gamma) \cdot (s_1\gamma) \cdots
(s_n\gamma)$ as claimed in the text.

\begin{lemma}\label{lem:substitutionrefl}
For all terms $s$ and substitutions $\gamma$ such that (1) $\domain(\gamma) \cap
\FMV(s) = \emptyset$, and (2) $\gamma(x) = x$ for all $x \in \FV(s)$ we have:
$\subrel{s,\gamma,s}$.
\end{lemma}

\begin{proof}
By a straightforward induction on the size of $s$.  For the case $s =
\abs{x}{s'}$ we can choose $x$ as the ``fresh'' variable.
    % 
    % \begin{itemize}
    % \item If $s = \afun$ then clearly $\subrel{s,\gamma,s}$
    % \item If $s = \avar$ then by assumption $\gamma(\avar) = \avar$, so $\subrel{s,
    %   \gamma,s}$.
    % \item If $s = h(s_1,\dots,s_n)$ with $n > 0$ then by the induction hypothesis,
    %   $\subrel{h,\gamma,h}$ and $\subrel{s_i,\gamma,s_i}$ for all $i$; hence
    %   $\subrel{s,\gamma,h \cdot s_1 \cdots s_n}$, and $h \cdot s_1 \cdots s_n$ is
    %   exactly $s$.
    % \item If $s = \tuple{s_1}{s_n}$ then by the induction hypothesis
    %   $\subrel{s_i,\gamma,s_i}$ for all $i$, so $\subrel{s,\gamma,\tuple{s_1}{s_n}}$
    %   follows.
    % \item If $s = \meta{\Avar}{s_1,\dots,s_k}$, then by the induction hypothesis,
    %   $\subrel{s_i,\gamma,s_i}$ for $1 \leq i \leq k$.  Since $\Avar \notin
    %   \domain(\gamma)$, we indeed have $\subrel{s,\gamma,\meta{\Avar}{s_1,\dots,
    %   s_k}}$.
    % \item If $s = \abs{\avar}{s'}$ then note that $\gamma(\bvar) = \bvar$ for all
    %   $\bvar \in \FV(s)$, and that $\avar \notin \FV(s)$ and therefore $\avar \neq
    %   \bvar$ for any $\bvar \in \FV(s)$, so clearly
    %   $\avar \notin \FV(\gamma(\bvar))$ holds for all $\bvar \in \FV(s)$.
    %   Also note that $\gamma' := [\avar:=\avar] \cup [\bvar:=\gamma(\bvar) \mid
    %   \bvar \in \domain(\gamma) \setminus \{\avar\}]$ satisfies the property that
    %   $\gamma'(\bvar) = \bvar$ for any $\bvar \in \FV(s')$. Since
    %   $\FMV(s) = \FMV(s')$, we also have $\domain(\gamma') \cap \FMV(s') =
    %   \domain(\gamma) \cap \FMV(s) = \emptyset$.
    %   Hence, we can apply the induction hypothesis to obtain $\subrel{s',\gamma',
    %   s'}$, and therefore $\subrel{s,\gamma,\abs{\avar}{s'}}$.
    % \qedhere
    % \end{itemize}
\end{proof}

Hence, applying an irrelevant substitution can be done without effect; even
without any variables being renamed.

\begin{lemma}\label{lem:substextend}
Suppose $\gamma(x) = \delta(x)$ for all $x \in \FV(s)$, and $\domain(\gamma)
\cap \FMV(s) = \domain(\delta) \cap \FMV(s)$ and $\gamma(\Avar) = \delta(\Avar)$
for $\Avar \in \FMV(s) \cap \domain(\gamma)$.  Then $\subrel{s,\gamma,t}$
implies $\subrel{s,\delta,t}$.
\end{lemma}

\begin{proof}
We prove the lemma by a straightforward induction on the size of $s$.
For the abstraction case, we observe that the definition of ``fresh'' allows us
to choose the same variable for $\delta$ as for $\gamma$, since $\gamma(\bvar) =
\delta(\bvar)$ for all $\bvar \in \in \FV(s)$.
%    % 
%    % \begin{itemize}
%    % \item
%    %   If $s = \afun$ then $t = \afun$ and indeed $\subrel{s,\delta,t}$.
%    % \item
%    %   If $s = \avar$ then $t = \gamma(\avar) = \delta(\avar)$ by assumption (as
%    %   $\avar \in \FV(s)$).
%    % \item 
%    %   If $s = h(s_1,\dots,s_n)$ with $n > 0$ then $t = w \cdot t_1 \cdots t_n$ with
%    %   $\subrel{h,\gamma,w}$ and $\subrel{s_i,\gamma,t_i}$ for all $i$.
%    %   By the induction hypothesis, also $\subrel{h,\delta,w}$ and $\subrel{s_i,
%    %   \delta,t_i}$ for all $i$, so indeed $\subrel{s,\delta,t}$.
%    % \item
%    %   If $s = \tuple{s_1}{s_n}$ then $t = \tuple{t_1}{t_n}$ with $\subrel{s_i,
%    %   \gamma,t_i}$ for all $i$; the induction hypothesis gives $\subrel{s_i,\delta,
%    %   t_i}$ as well, so indeed $\subrel{s,\delta,t}$.
%    % \item
%    %   If $s = \abs{\avar}{s'}$ then $t = \abs{\cvar}{t'}$ where
%    %   (a) $\cvar$ has the same type as $\avar$,
%    %   (b) $\cvar \notin \FV(\gamma(\bvar))$ for any $\bvar \in \FV(s)$, and
%    %   therefore also $\cvar \notin \FV(\delta(\bvar))$ for any $\bvar \in \FV(s)$,
%    %   and
%    %   (c) $\subrel{s',[\avar:=\cvar] \cup [\bvar:=\gamma(\bvar) \mid \bvar \in
%    %   \domain(\gamma) \setminus \{\avar\}],t'}$, which means that also $\subrel{s',
%    %   [\avar:=\cvar] \cup [\bvar:=\delta(\bvar) \mid \bvar \in \domain(\delta)
%    %   \setminus \{\avar\}],t'}$ by the induction hypothesis.
%    % \item
%    %   If $s = \meta{\Avar}{s_1,\dots,s_k}$ with $\Avar \notin \domain(\gamma)$, then
%    %   by definition $t = \meta{\Avar}{t_1,\dots,t_k}$ with $\subrel{s_i,\gamma,t_i}$
%    %   for all $i$, so by the induction hypothesis $\subrel{s_i,\delta,t_i}$ as well.
%    %   Since $\Avar \notin \domain(\gamma) \cap \FMV(s)$, by assumption also
%    %   $\Avar \notin \domain(\delta) \cap \FMV(s)$, so since $\Avar \in \FMV(s)$ we
%    %   have $\Avar \notin \domain(\delta)$; hence, we conclude
%    %   $\subrel{s,\delta,t}$ as well.
%    % \item
%    %   If $s = \meta{\Avar}{s_1,\dots,s_k}$ and $\Avar \in \domain(\gamma)$ then
%    %   there exist $t_1,\dots,t_k$ with $\subrel{s_i,\gamma,t_i}$ and
%    %   $\gamma(\Avar) = \abs{\avar_1 \dots \avar_k}{t} = \delta(\Avar)$.
%    %   By the induction hypothesis also $\subrel{s_i,\delta,t_i}$, and therefore
%    %   $\subrel{s,\delta,t}$ follows immediately.
%    %   \qedhere
%    % \end{itemize}
\end{proof}

Hence, the only relevant part of a substitution is its value on the
meta-variables and unbound variables of the term it is applied on.

Our following result is a helper to assess what substitution does to a longer
abstraction.

\begin{lemma}\label{lem:abssubst}
If $\subrel{\abs{\avar_1 \dots \avar_n}{s},\gamma,u}$ then $u = \abs{\bvar_1
\dots \bvar_n}{t}$ such that:
\begin{itemize}
\item each $\bvar_i \notin \FV(\gamma(\cvar))$ for any $\cvar \in \FV(s)
  \setminus \{\avar_1,\dots,\avar_n\}$
\item if $\bvar_i = \bvar_j$ for some $i < j$, then $\avar_i \notin \FV(s)$ or
  $\avar_i = \avar_k$ for some $k > i$ 
\item $\subrel{s,\gamma',t}$ where $\gamma'$ is any substitution with
  $\domain(\gamma') \cap \M = \domain(\gamma) \cap \M$ and
  $\gamma'(\Avar) = \gamma(\Avar)$ for $\Avar \in \FMV(s) \cap \domain(\gamma)$,
  and $\gamma'(\avar_i) = \bvar_i$ if there is no $j > i$ with $\avar_i =
  \avar_j$, and $\gamma'(\avar) = \gamma(\avar)$ for $\avar \in \V \setminus
  \{\avar_1, \dots,\avar_n\}$.
\end{itemize}
\end{lemma}

\begin{proof}
By induction on $n$.  For $n = 0$ the lemma clearly holds (using Lemma
\ref{lem:substextend}).
Now if $\subrel{\abs{\avar_1 \cdots \avar_n}{s},\gamma,u}$ and $n > 0$ then
by definition $u = \abs{\bvar_1}{u'}$ with $\bvar_1 \notin \FV(\gamma(\cvar))$
for any $\cvar \in \FV(\abs{\avar_1 \dots \avar_n}{s}) = \FV(s) \setminus
\{\avar_1,\dots,\avar_n\}$, and $\subrel{\abs{\avar_2 \dots \avar_n}{s},
[\avar_1:=\bvar_1] \cup [\cvar:=\gamma(\cvar) \mid \cvar \in \domain(\gamma)
\setminus \{\avar_1\}],u'}$.
Each of the three requirements now follows easily by the induction hypothesis and the properties above.
%    % 
%    % Write $\gamma^{\avar_1:=\bvar_1} = [\avar_1:=\bvar_1] \cup [\cvar:=\gamma(\cvar)
%    % \mid \cvar \in \domain(\gamma) \setminus \{\avar_1\}]$.  We have:
%    % \begin{itemize}
%    % \item By the first part of the induction hypothesis:
%    %   for $i > 1$ each $\bvar_i \notin \FV(\gamma^{\avar_1:=\bvar_1}(\cvar))$ for any
%    %   $\cvar \in \FV(s) \setminus \{\avar_2,\dots,\avar_n\}$.  Since
%    %   $\gamma^{\avar_1:=\bvar_1}(\cvar) = \gamma(\cvar)$ for any $\cvar \neq
%    %   \avar_1$, this implies $\bvar_i \notin \FV(\gamma(\cvar))$ for any $\cvar \in
%    %   \FV(s) \setminus \{\avar_1,\dots,\avar_n\}$.
%    % \item Using the case $\cvar = \avar_1$, this also implies that, if $\avar_1 \in
%    %   \FV(s) \setminus \{\avar_2,\dots,\avar_n\}$, then $y_i \neq y_1$.
%    %   Hence, if $y_1 = y_i$ for $1 < i$ then $x_1 \notin \FV(s)$ or $x_1 = x_k$ for
%    %   some $k > 1$.
%    % \item By the second part of the induction hypothesis: if $\bvar_i = \bvar_j$ for
%    %   $1 < i < j$ then $\avar_i \notin \FV(s)$ or $\avar_i = \avar_k$ for some $k >
%    %   i$.  Hence, combined with the previous point, this holds for \emph{all}
%    %   $i < j$; which gives the second requirement we have to prove.
%    % \item Let $\gamma'$ be any substitution with $\domain(\gamma') \cap \M =
%    %   \domain(\gamma) \cap \M$ and $\gamma'(\Avar) = \gamma(\Avar)$ for $\Avar \in
%    %   \FMV(s) \cap \domain(\gamma)$ and $\gamma'(\avar_i) = \bvar_i$ if there is no
%    %   $j > i$ with $\avar_i = \avar_j$, and $\gamma'(\cvar) = \gamma(\cvar)$ for
%    %   $\cvar \in \V \setminus \{\avar_1,\dots,\avar_n\}$.  Then:
%    %   \begin{itemize}
%    %   \item $\domain(\gamma') \cap \M = \domain(\gamma^{\avar_1:=\bvar_1}) \cap \M
%    %     = \domain(\gamma) \cap \M$
%    %   \item $\gamma'(\Avar) = \gamma^{\avar_1:=\bvar_1} = \gamma(\Avar)$ for all
%    %     $\Avar \in \FMV(s) \cap \domain(\gamma)$
%    %   \item clearly, for $1 < i \leq n$: $\gamma'(\avar_i) = \bvar_i$ if there is no
%    %     $j > i$ with $\avar_i = \avar_j$
%    %   \item for $\cvar \in \V \setminus \{\avar_2,\dots,\avar_n\}$ we have
%    %     $\gamma'(\cvar) = \gamma^{\avar_1:=\bvar_1}(\cvar)$:
%    %     \begin{itemize}
%    %     \item if $\cvar = \avar_1$ then since $\cvar \notin \{\avar_2,\dots,
%    %       \avar_n\}$ we have by assumption that $\gamma'(\avar_1) = \bvar_1 =
%    %       \gamma^{\avar_1:=\bvar_1}(\avar_1)$
%    %     \item if $\cvar \neq \avar_1$ then by assumption $\gamma'(\cvar) =
%    %       \gamma(\cvar) = \gamma^{\avar_1:=\bvar_1}(\cvar)$ by definition
%    %     \end{itemize}
%    %   \end{itemize}
%    %   \ \\
%    %   Hence, by the induction hypothesis we obtain $\subrel{s,\gamma',t}$.
%    %   \qedhere
%    % \end{itemize}
\end{proof}

\subsection{Combining substitutions}

The proof that $(s\gamma)\delta = s(\gamma\delta)$ is actually quite complex,
both due to abstraction and meta-application.  For this reason, we split it up
into multiple smaller steps.

To start, let us define the combination of two substitutions as a relation:

\begin{definition}
For substitutions $\gamma,\delta,\eta$ we say that $\subrel{\gamma,\delta,\eta}$
if:
\begin{itemize}
\item $\domain(\eta) = \domain(\gamma) \cup \domain(\delta)$, and
\item $\subrel{\gamma(\avar),\delta,\eta(\avar)}$ for all $\avar \in
  \domain(\gamma)$
\item $\delta(\avar) = \eta(\avar)$ for all $\avar \in \domain(\delta) \setminus
  \domain(\gamma)$
\end{itemize}
\end{definition}

We first prove a version of our desired result where $\gamma$ has a limited
domain.

\begin{lemma}\label{lem:combinesubst:allbound}
Let $\domain(\gamma) \subseteq \V$, let $s,t$ be terms and let
$\delta,\eta$ be substitutions, with:
\begin{itemize}
\item $\domain(\delta) \setminus \V = \domain(\eta) \setminus \V$
\item $\delta(\Avar) = \eta(\Avar)$ for all $\Avar \in \domain(\delta) \setminus
  \V$
\item $\subrel{\gamma(\avar),\delta,\eta(\avar)}$ for all $\avar \in \FV(s)$
\item $\subrel{s,\gamma,t}$
\end{itemize}
Then there exists $u$ such that both $\subrel{t,\delta,u}$ and
$\subrel{s,\eta,u}$.
\end{lemma}

Note that the requirements above all satisfied if $\domain(\gamma) \subseteq \V$
and $\subrel{\gamma,\delta,\eta}$ and $\subrel{s,\gamma,t}$.  We phrase it more
generally for the sake of an induction hypothesis.

\begin{proof}
By induction on the size of $s$.

In the case of an abstraction $s = \abs{\avar}{s'}$ and $t = \abs{\bvar}{t'}$,
we choose the fresh variable $\cvar$ outside both $\FV(\delta(w))$ for all $w
\in \FV(t) \cup \FMV(t)$, and $\FV(\eta(w))$ for all $w \in \FV(s) \cup
\FMV(s)$.  Since $\delta$ extended with $[\bvar:=\cvar]$ and $\eta$ extended
with $[\avar:=\cvar]$ are both substitutions with domain $\subseteq \Vbound$,
we can apply the induction hypothesis.  The case of a meta-application
$\meta{\Avar}{s_1,\dots,s_k}$ is trivial with the induction hypothesis if
$\Avar \notin \domain(\delta)$, and quite easy using
Lemma~\ref{lem:substdefined} if $\Avar \in \domain(\delta)$.
    % 
    % TODO: update the proof to have $\domain(\gamma) \subseteq \Vbound$ instead
    % of just containing variables
    % \begin{itemize}
    % \item If $s = \afun$ or $s = \avar \in \Vfree$ then $t = \afun$;
    %   we choose $u := \afun$ as well.
    % \item If $s = \avar \in \V$ then $t = \gamma(\avar)$, and we let
    %   $u := \eta(\avar)$.  Then by choice of $\eta$, we indeed have
    %   $\subrel{t,\delta,u}$ and clearly we also have $\subrel{s,\eta,u}$.
    % \item If $s = h(s_1,\dots,s_n)$ with $n > 0$, then by definition
    %   $t = q \cdot t_1 \cdots t_n$ where $\subrel{h,\gamma,q}$ and
    %   $\subrel{s_i,\gamma,t_i}$ for all $i$.  By the induction hypothesis, there
    %   exist $w$ such that both $\subrel{q,\delta,w}$ and $\subrel{h,\eta,w}$, and
    %   $u_1,\dots,u_n$ such that for all $i$ both $\subrel{t_i,\delta,u_i}$ and
    %   $\subrel{s_i,\eta,u_i}$.  Choose $u := w \cdot u_1 \cdots u_n$.
    %   Then $\subrel{t,\delta,u}$ by Lemma \ref{lem:appsubstitute}, and
    %   $\subrel{s,\eta,u}$ by definition.
    % \item If $s = \tuple{s_1}{s_n}$ then by definition $t = \tuple{t_1}{t_n}$ with
    %   $\subrel{s_i,\gamma,t_i}$ for all $i$, and by the induction hypothesis we
    %   find $u_1,\dots,u_n$ such that both $\subrel{t_i,\delta,u_i}$ and
    %   $\subrel{s_i,\eta,u_i}$ for all $i$.  Choose $u := \tuple{u_1}{u_n}$.
    % \item If $s = \abs{\avar}{s'}$ then $t = \abs{\bvar}{t'}$ with
    %   $\bvar \notin \FV(\gamma(w))$ for any $w \in \FV(s)$, and we have
    %   $\subrel{s',\gamma^{\avar:=\bvar},t'}$ where $\gamma^{\avar:=\bvar} =
    %   [\avar:=\bvar] \cup [w:=\gamma(w) \mid w \in \domain(\gamma) \setminus
    %   \{\avar\}]$.  Note that the domain of $\gamma^{\avar:=\bvar}$ is also
    %   limited to variables.
    %   Let $\cvar$ be a variable of the same type as $\avar$, such that:
    %   \begin{itemize}
    %   \item $\cvar \notin \FV(\delta(w))$ for any $w \in \FV(t) \cup
    %     (\FMV(t) \cap \domain(\delta))$;
    %   \item $\cvar \notin \FV(\eta(w))$ for any $w \in \FV(s) \cup
    %     (\FMV(s) \cap \domain(\eta))$.
    %   \end{itemize}
    %   \ \\
    %   Since $\cvar$ is only required to be unequal to a finite number of variables,
    %   and $\Vbound$ has infinitely many elements of each type, such a variable can
    %   always be found.
    % 
    %   Now let $\delta^{\bvar:=\cvar} = [\bvar:=\cvar] \cup [w:=\delta(w) \mid w \in
    %   \domain(\delta)\setminus \{\bvar\}]$ and let $\eta^{\avar:=\cvar} = [\avar:=
    %   \cvar] \cup [w:=\eta(w) \mid w \in \domain(\eta) \setminus \{\avar\}]$.  Then
    %   the domains and values for the meta-variables are the same as in $\delta$ and
    %   $\eta$ respectively, and we have $\subrel{\gamma^{\avar:=\bvar}(w),
    %   \delta^{\bvar:=\cvar},\eta^{\avar:=\cvar}(w)}$ for all $w \in \FV(s')$:
    %   \begin{itemize}
    %   \item for $w = \avar$ we have $\subrel{\bvar,\delta^{\bvar:=\cvar},\cvar}$;
    %   \item and for all other $w$ we have $\subrel{\gamma(w),\delta^{\bvar:=\cvar},
    %     \eta(w)}$ by Lemma \ref{lem:substextend}, because $\bvar \notin
    %     \FV(\gamma(w))$ when $w \in \FV(s) = \FV(s') \setminus \{\avar\}$ (by
    %     choice of $\bvar$).
    %   \end{itemize}
    %   \ \\
    %   Hence, we can apply the induction hypothesis on $\subrel{s',\gamma^{\avar:=
    %   \bvar},t'}$ to obtain $u'$ such that both $\subrel{t',\delta^{\bvar;=\cvar},
    %   u'}$ and $\subrel{s',\eta^{\avar:=\cvar}, u'}$.  We choose $u :=
    %   \abs{\cvar}{u'}$.  Due to the requirements on $\cvar'$ we have both
    %   $\subrel{t,\delta,u}$ and $\subrel{s,\eta,u}$.
    % \item If $s = \meta{\Avar}{s_1,\dots,s_k}$ with $\avar \in \M$ and $\Avar \notin
    %   \domain(\delta)$, note that $\Avar \notin \domain(\eta)$ as well, and
    %   $\Avar \notin \domain(\gamma)$ by assumption.  Hence,
    %   $t = \meta{\Avar}{t_1,\dots,t_k}$ for some $t_1,\dots,t_k$ such that
    %   $\subrel{s_i,\gamma,t_i}$ for all $i$.  By the induction hypothesis on each
    %   $s_i$, there exist $u_1,\dots,u_k$ such that $\subrel{t_i,\delta,u_i}$ and
    %   $\subrel{s_i,\eta,u_i}$ for all $i \in \{1,\dots,k\}$.  We are done choosing
    %   $u := \meta{\Avar}{u_1,\dots,u_k}$.
    % \item Finally, if $s = \meta{\Avar}{s_1,\dots,s_k}$ with $\Avar \in \M$ and
    %   $\Avar \in \domain(\delta)$, then note that $\Avar \notin \domain(\gamma)$ and
    %   $\eta(\Avar) = \gamma(\Avar)$.  But then $t = \meta{\Avar}{t_1,\dots,t_k}$
    %   with $\subrel{s_i,\gamma,t_i}$ for all $i$, and by the induction hypothesis we
    %   can find $u_1,\dots,u_k$ such that both $\subrel{t_i,\delta,u_i}$ and
    %   $\subrel{s_i,\eta,u_i}$ for all $i$.
    %   Write $\delta(\Avar) = \eta(\Avar) = \abs{z_1 \dots z_k}{q}$, and let
    %   $\chi := [z_1:=u_1,\dots,z_k:=u_k]$.  Then by
    %   Lemma~\ref{lem:substdefined} there exists $u$ such that
    %   $\subrel{q,\chi,u}$.  Then by definition both $\subrel{t,\delta,u}$ and
    %   $\subrel{s,\eta,u}$.
    %   \qedhere
    % \end{itemize}
\end{proof}

=====================

We use Lemma \ref{lem:combinesubst:allbound} to obtain the full result we want:

\begin{lemma}\label{lem:combinesubst}
Let $s,t$ be terms and let $\gamma,\delta,\eta$ be substitutions, such that:
\begin{itemize}
\item $\domain(\eta) \setminus \V = (\domain(\gamma) \cup \domain(\delta))
  \setminus \V$
\item $\subrel{\gamma(\avar),\delta,\eta(\avar)}$ for all $\avar \in \FV(s)
  \cup (\FMV(s) \cap \domain(\gamma))$
\item $\delta(\Avar) = \eta(\Avar)$ for all $\Avar \in (\FMV(s) \setminus
  (\V \cup \domain(\gamma))) \cap \domain(\delta)$
\item $\subrel{s,\gamma,t}$
\end{itemize}
Then there exists $u$ such that both $\subrel{t,\delta,u}$ and
$\subrel{s,\eta,u}$.
\end{lemma}

\begin{proof}
By induction on the size of $s$.

\begin{itemize}
\item If $s = \afun$ then $t = u = \afun$ and we are quickly done.
\item If $s = \avar$ then let $u = \eta(\avar)$.
\item If $s = h(s_1,\dots,s_n)$ or $\tuple{s_1}{s_n}$ then we complete by
  induction.
\item If $s = \meta{\Avar}{s_1,\dots,s_k}$ with $\Avar \notin \domain(\gamma)$
  and $\Avar \notin \domain(\delta)$, then note that $\Avar \notin
  \domain(\eta)$; by the induction hypothesis we find $u_i$ with $\subrel{s_i,
  \eta,u_i}$ and $\subrel{t_i,\delta,u_i}$ for all $i$ and we can choose
  $u := \meta{\Avar}{u_1,\dots,u_k}$.
\item If $s = \meta{\Avar}{s_1,\dots,s_k}$ with $\avar \notin \domain(\gamma)$
  but $\Avar \in \domain(\delta)$ then $t = \meta{\Avar}{t_1,\dots,t_k}$ and
  we can find $u_i$ with $\subrel{s_i,\eta,u_i}$ and $\subrel{t_i,\delta,u_i}$.
  Moreover, $\delta(\Avar) = \eta(\Avar)$.
  If $\delta(\Avar) = \abs{\avar_1 \cdots \avar_k}{q}$ then choose $u$ such
  that $\subrel{q,[\avar_1:=u_1,\dots,\avar_k:=u_k],u}$.  Then both
  $\subrel{s,\eta,u}$ and $\subrel{t,\delta,u}$.
\item If $s = \abs{\avar}{s'}$ then $t = \abs{\bvar}{t'}$ with $\bvar
  \notin \FV(\gamma(w))$ for any $w \in \FV(s) \cup \FMV(s)$, and
  $\subrel{s',\gamma^{\avar:=\bvar},t'}$.
  Let $\cvar$ be a variable of the same type as $\avar$, which does not occur
  in any $\delta(w)$ for variables or meta-variables $w$ in either $s$ or $t$.
  Then $\gamma^{\avar:=\bvar},\delta^{\bvar:=\cvar}$ and $\eta^{\avar:=\cvar}$
  satisfy the requirement to apply the induction hypothesis:
  \begin{itemize}
  \item $\domain(\eta^{\avar:=\cvar}) \setminus \V =
    (\domain(\eta) \cup \{\avar\}) \setminus \V = \domain(\eta) \setminus \V =
    (\domain(\gamma) \cup \domain(\delta)) \setminus \V =
    (\domain(\gamma) \cup \{\avar\} \cup \domain(\delta) \cup \{\bvar\})
    \setminus \V =
    (\domain(\gamma^{\avar:=\bvar}) \cup \domain(\delta^{\bvar:=\cvar}))
    \setminus \V$
  \item $\subrel{\gamma^{\avar:=\bvar}(w),\delta^{\bvar:=\cvar},\eta^{\avar:=
    \cvar}(w)}$ for all $w \in \FV(s') \cup (\FMV(s') \cap \domain(\gamma^{
    \avar:=\bvar}))$:
    \begin{itemize}
    \item if $w = \avar$, then indeed $\subrel{\bvar,\delta^{\bvar:=\cvar},
      \cvar}$
    \item if not, $\gamma^{\avar:=\bvar}(w) = \gamma(w)$ and $w \in \FV(s) \cup
      (\FMV(s) \cap \domain(\gamma))$. Since $w \in \FV(s') \setminus \{\avar\}$
      we have $w \in \FV(s)$, so $\bvar \notin \FV(\gamma(w))$.
      Hence, by Lemma~\ref{lem:substextend},
      $\subrel{\gamma(w),\delta,\eta(w)}$ implies $\subrel{\gamma(w),
      \delta^{\bvar:=\cvar},\eta(w)}$.  This suffices because
      $\eta(w) = \eta^{\avar:=\cvar}(w)$.
    \item for meta-variables $\Avar$, $\delta^{\bvar:=\cvar}(\Avar) =
      \delta(\Avar)$ and $\eta^{\avar:=\cvar}(\Avar) = \eta(\Avar)$, so the
      third requirement is still satisfied
    \item $\subrel{s',\gamma^{\avar:=\bvar},t'}$ is given. \\
    \end{itemize}
    Hence, we find $u'$ such that both $\subrel{t',\delta^{\bvar:=\cvar},u'}$
    and $\subrel{s',\eta^{\avar:=\cvar},u'}$.  We are done choosing
    $u = \abs{\cvar}{u'}$.
  \end{itemize}
\item This leaves only the case $s = \meta{\Avar}{s_1,\dots,s_k}$ with
  $\Avar \in \domain(\gamma)$; let $\gamma(\Avar) = \abs{\avar_1 \dots
  \avar_k}{a}$.  We make a number of observations.
  \begin{enumerate}
  \item\label{obs:stot}
    there exist $t_1,\dots,t_k$ such that
    $\subrel{s_i,\gamma,t_i}$ for all $i$, and
    $\subrel{a,[\avar_1:=t_1,\dots,\avar_k:=t_k],t}$
  \item\label{obs:stoui}
    there exist $u_1,\dots,u_k$ such that both
    $\subrel{t_i,\delta,u_i}$ and $\subrel{s_i,\eta,u_i}$ for all $i$
    \\ (by the induction hypothesis and \ref{obs:stot})
  \item\label{obs:etaX}
    we can write $\eta(\Avar) = \abs{\bvar_1 \dots \bvar_k}{b}$ with
    $\subrel{a,\delta^{\avar_1:=\bvar_1,\dots,\avar_k:=\bvar_k},b}$
    \\ (by Lemma~\ref{lem:abssubst} because $\subrel{\gamma(\Avar),\delta,
    \eta(\Avar)}$)
  \item\label{obs:bvari}
    each $\bvar_i \notin \FV(\delta(\cvar))$ for any $\cvar \in \FV(a)
    \setminus \{\avar_1,\dots,\avar_k\}$
    \\ (also by Lemma~\ref{lem:abssubst})
  \item if $\bvar_i = \bvar_j$ for some $i < j$ then $\avar_i \notin
    \FV(a)$ or $\avar_i = \avar_l$ for some $l > i$
    \\ (also by Lemma~\ref{lem:abssubst})
  \end{enumerate}
  By (\ref{obs:stot}) and Lemma~\ref{lem:combinesubst:allbound} we obtain that
  some $u$ exists such that both $\subrel{t,\delta,u}$, and
  $\subrel{a,\zeta,u}$ where $\zeta$ is any substitution with
  $\subrel{[\avar_1:=t_1,\dots,\avar_k:=t_k],\delta,\zeta}$.
\end{itemize}

====================

Needed: we need to define $\zeta$ in such a way that we can get from
$\subrel{a,\delta^{\avar_1:=\bvar_1,\dots,\avar_k:=\bvar_k},b}$ and
$\subrel{a,\zeta,u}$ that also $\subrel{b,[\vec{y}:=\vec{u}],u}$---which is a
result in the other direction.  I think this has potential because the second
substitution $[\vec{\bvar}:=\vec{u}]$ has as domain exactly the variables that
the $\avar_i$ are mapped to.
\end{proof}

\newcommand{\aprel}[1]{\mathsf{ap}(#1)}

From Lemma \ref{lem:combinesubst:allbound}, we obtain the following helper result regarding superapplication:

\begin{lemma}\label{lem:boundsubsthelper}
Suppose:
\begin{itemize}
\item $\domain(\gamma) \subseteq \Vbound$ and $\FV(a) \cap \domain(\gamma) = \emptyset$,
\item $\aprel{a, [s_1,\dots,s_n], s}$,
\item $\subrel{s_i,\gamma,t_i}$ for $1 \leq i \leq n$.
\end{itemize}
Then there exists $t$ such that $\aprel{a,[t_1,\dots,t_n],t}$ and $\subrel{s,\gamma,t}$.
\end{lemma}

That is, if $\gamma$ does not affect the free variables in $a$, then
$\superapply(a,[s_1,\dots,s_n])\gamma = \superapply(a,[s_1\gamma,\dots,s_n\gamma])$.
We limit interest to $\gamma$ with domain in $\Vbound$ since we only need it in this form,
and this avoids the meta-application case.

\begin{proof}
We can always write $a = \abs{\avar_1 \dots \avar_k}{b}$ with either $k = n$ or $k < n$ and
$b$ not an abstraction.
Then $\aprel{a,[s_1,\dots,s_n],s}$ implies $\subrel{b,[\avar_1:=s_1,\dots,\avar_k:=s_k],s'}$
for some $s'$ such that $s = s' \cdot s_{k+1} \cdots s_n$ (if $k = n$, then $s = s'$).
%
We apply Lemma \ref{lem:combinesubst:allbound} to obtain $t'$ such that both
$\subrel{s',\gamma,t'}$ and $\subrel{b,\eta,t'}$, and are done choosing $t := t \cdot t_{k+1}
\cdots t_n$.
    % Write $\chi := [\avar_k:=s_1,\dots,\avar_k:=s_n]$, and let $\eta := [\avar_k:=t_1,\dots,t_k]$.
    % Then:
    % \begin{itemize}
    % \item As $\{\avar_1,\dots,\avar_k\} \subseteq \Vbound$, each of $\chi$, $\gamma$ and $\eta$ has a
    %   domain $\subseteq \Vbound$.
    % \item We have $\subrel{\chi(\bvar),\gamma,\eta(\bvar)}$ for all $\bvar \in \FV(b)$:
    %   \begin{itemize}
    %   \item if $\bvar = \avar_i$ this holds because $\subrel{s_i,\gamma,t_i}$ by assumption;
    %   \item otherwise, $\chi(\bvar) = \eta(\bvar) = \bvar$, and it holds because $\gamma(\avar) =
    %     \avar$ too: $\FV(a) \cap \domain(\gamma) = \emptyset$, so since $\bvar \in \FV(a)$
    %     necessarily $\bvar \notin \domain(\gamma)$
    %   \end{itemize}
    % \item As noted above, we have $\subrel{b,\chi,s}$.
    % \end{itemize}
    % Hence we can apply Lemma \ref{lem:combinesubst:allbound} to obtain $t'$ such that both
    % $\subrel{s',\gamma,t'}$ and $\subrel{b,\eta,t'}$.
    % Let $t := t' \cdot t_{k+1} \cdots t_n$.  Then:
    % \begin{itemize}
    % \item $\subrel{s,\gamma,t}$ by Lemma \ref{lem:appsubstitute};
    % \item $\aprel{a,[t_1,\dots,t_n],t}$ by definition.
    %   \qedhere
    % \end{itemize}
\end{proof}

%The next helper result is phrased in a very technical way so as to be usable in other proofs, but
%it essentially expands on Lemma \ref{lem:combinesubst:allbound} by allowing the second substitution,
%but not the first, to have non-binder variables in its domain.  It is not quite the same, however;
%rather than showing that $(s\gamma)\delta = s(\gamma\delta)$ it states that, under certain
%restrictions, $(s\gamma)\delta = s\delta_1(\gamma\delta)_2$, where $\delta_1$ is the
%restriction of $\delta$ to variables which do not occur in $\domain(\gamma)$, and $(\gamma\delta)_2$
%is the restriction of $\gamma\delta$ to variables in the domain of $\gamma$.
%This for instance allows us to conclude that $a[\avar_1:=s_1,\dots,\avar_k:=s_k]\delta =
%a[\avar_1:=s_1\delta,\dots,\avar_k:=s_k\delta]$ if $\FV(s) \cap (\domain(\delta) \setminus
%\{\avar_1,\dots,\avar_k\}) = \emptyset$.
%
%\begin{lemma}\label{lem:combinesubst:gammabound}
%Let $\domain(\gamma),\domain(\eta) \subseteq \Vbound$ with $\eta$ having a finite domain,
%$s$ a term, and $\chi$ a mapping from $\Vbound$ to $\Vbound$, such that:
%\begin{itemize}
%\item $\subrel{s,\gamma,t}$
%\item $\subrel{\gamma(\avar),\delta,\eta(\chi(\avar))}$ for all $\avar \in \domain(\gamma) \cap
%  \FV(s)$
%\item $\domain(\eta) \cap \FV(\delta(\avar)) = \emptyset$ for all $\avar \in \FV(s) \setminus
%  \domain(\gamma)$
%\item $\subrel{s,\delta^\chi,q}$, where $\delta^\chi$ is the substitution with
%  $\delta^\chi(\avar) = \chi(\avar)$ if $\avar \in \domain(\gamma)$ and $\delta^\chi(\avar) =
%  \delta(\avar)$ otherwise.
%\end{itemize}
%Then there exists $u$ such that both $\subrel{t,\delta,u}$ and $\subrel{q,\eta,u}$.
%\end{lemma}
%
%\begin{proof}
%We prove the result by induction on the size of $s$.
%
%\begin{itemize}
%\item If $s = h(s_1,\dots,s_n)$ with $n > 0$ then:
%  \begin{itemize}
%  \item $t = v_0 \cdot v_1 \cdots v_n$ with $\subrel{h,\gamma,v_0}$ and $\subrel{s_i,\gamma,v_i}$
%    for $1 \leq i \leq n$, and
%  \item $q = w_0 \cdot w_1 \cdots w_n$ with $\subrel{h,\delta^\chi,w_0}$ and
%    $\subrel{s_i,\delta^\chi,w_i}$ for $1 \leq i \leq n$.
%  \end{itemize}
%  \ \\
%  Hence, by the induction hypothesis, there exist $u_0,\dots,u_n$ with
%  $\subrel{v_i,\delta,u_i}$ and $\subrel{w_i,\eta,u_i}$ for $0 \leq i \leq n$.
%  Choose $u := u_0 \cdot u_1 \cdots u_n$. Then both
%  $\subrel{t,\delta,u}$ and $\subrel{q,\eta,u}$ by Lemma \ref{lem:appsubstitute}.
%\item If $s = \afun$ then necessarily $t = q = \afun$ as well; we let $u := \afun$.
%\item If $s = \avar \in \domain(\gamma)$ then $t = \gamma(\avar)$ and $q = \delta^\chi(\avar) =
%  \chi(\avar)$.  We choose $u := \eta(\chi(\avar))$.  Then by assumption $\subrel{t,\delta,u}$
%  (since $\avar \in \FV(s) \cap \domain(\gamma)$) and by definition $\subrel{q,\eta,u}$.
%\item If $s = \avar \in \V \setminus \domain(\gamma)$, then $t = \avar$ and $q = \delta(\avar)$.
%  We choose $u := q$.  Then clearly $\subrel{t,\delta,u}$.  Since $\domain(\eta) \cap \FV(q) =
%  \emptyset$ we have $\subrel{q,\eta,u}$ by Lemma \ref{lem:substitutionrefl}.
%\item If $s = \abs{\avar}{s'}$ then $t = \abs{\bvar_1}{t'}$ and $q = \abs{\bvar_2}{q'}$ with:
%  \begin{itemize}
%  \item $\bvar_1 \notin \FV(\gamma(w))$ for any $w \in \FV(s) = \FV(s') \setminus \{\avar\}$
%  \item $\bvar_2 \notin \FV(\delta^\chi(w))$ for any $w \in \FV(s) = \FV(s') \setminus \{\avar\}$
%  \end{itemize}
%  \ \\
%  Let $\cvar$ be a variable of the same type as $\avar$, such that $\cvar \notin \FV(\delta(w))$
%  for any $w \in \FV(t)$, $\cvar \notin \FV(\eta(w))$ for any $w \in \FV(q)$, $\cvar \neq \bvar_2$
%  and $\cvar \notin \domain(\eta)$.
%  Since there are infinitely many binder variables of all types, we can always find such $\cvar$.
%  Now let:
%  \begin{itemize}
%  \item $\gamma^{\avar:=\bvar_1} := [\avar:=\bvar_1] \cup [w:=\gamma(w) \mid w \in \domain(\gamma)
%    \setminus \{\avar\}$
%  \item $\delta^{\bvar_1:=\cvar} := [\bvar_1:=\cvar] \cup [w:=\delta(w) \mid w \in \domain(\delta)
%    \setminus \{\bvar\}]$
%  \item $\eta^{\bvar_2:=\cvar} := [\bvar_2:=\cvar] \cup [w:=\eta(w) \mid w \in \domain(\eta)
%    \setminus \{\bvar\}]$
%  \item $\chi^{\avar:=\bvar_2} := [\avar:=\bvar_2] \cup [w := \chi(w) \mid w \in \domain(\chi)
%    \setminus \{\avar\}]$
%  \end{itemize}
%  \ \\
%  We can apply the induction hypothesis, because:
%  \begin{itemize}
%  \item $\domain(\gamma^{\avar:=\bvar_1}),\domain(\eta^{\bvar_2:=\cvar}) \subseteq \Vbound$;
%  \item $\subrel{s',\gamma^{\avar:=\bvar},t'}$ follows from $\subrel{s,\gamma,t}$
%  \item $\subrel{\gamma^{\avar:=\bvar_1}(w),\delta^{\bvar_1:=\cvar},\eta^{\bvar_2:=\cvar}(\chi^{
%    \avar:=\bvar_2}(w))}$ for all $w \in \FV(s') \cap \domain(\gamma^{\avar:=\bvar})$:
%    \begin{itemize}
%    \item if $w = \avar$ then this exactly says $\subrel{\bvar_1,\delta^{\bvar_1:=\cvar},\cvar}$,
%      which clearly holds;
%    \item otherwise, we must show $\subrel{\gamma(w),\delta^{\bvar_1:=\cvar},\eta^{\bvar_2:=\cvar}(
%      \chi(w))}$; since $\bvar_2 \notin \FV(\delta^\chi(w))$ because $w \in \FV(s') \setminus \{
%      \avar\}$, clearly $\chi(w) \neq \bvar_2$, so $\eta^{\bvar_2:=\cvar}(\chi(w)) =
%      \eta(\chi(w))$; and since $\bvar_1 \notin \FV(\gamma(w))$, Lemma \ref{lem:substextend} allows
%      us to derive $\subrel{\gamma(w),\delta^{\bvar_1:=\cvar},\eta(\chi(w))}$ from
%      $\subrel{\gamma(w),\delta,\eta(\chi(w))}$
%    \end{itemize}
%  \item $\domain(\eta^{\bvar_2:=\cvar}) \cap \FV(\delta^{\bvar_1:=\cvar}(w)) = \emptyset$ for all
%    $w \in \FV(s') \setminus \domain(\gamma^{\avar:=\bvar_1})$:
%    \begin{itemize}
%    \item if $w = \bvar_1$ then $\delta^{\bvar_1:=\cvar}(w) = \cvar$, and since we chose
%      $\cvar \neq \bvar_2$ and $\cvar \notin \domain(\eta)$, the overlap is empty
%    \item otherwise, if $w = \avar$ then $w \notin \FV(s') \setminus
%      \domain(\gamma^{\avar:=\bvar_1})$;
%    \item otherwise, note that note that since $w \notin \domain(\gamma^{\avar:=\bvar})$ and $w \neq
%      \avar$ implies $w \notin \domain(\gamma)$, so $\delta^{\bvar_1:=\cvar}(w) = \delta(w) =
%      \delta^\chi(w)$; we both have $\bvar_2 \notin \FV(\delta^\chi(w))$ and $\domain(\eta) \cap
%      \FV(\delta^\chi(w)) = \emptyset$, so $\eta^{\bvar_2:=\cvar} \cap \FV(\delta^{\bvar_1:=\cvar}(w))
%      = \emptyset$ as well
%    \end{itemize}
%  \item ====================
%  \item $\subrel{s',\delta',q'}$, where $\delta'$ is the substitution with $\delta'(w) =
%    \chi^{\avar:=\bvar_2}(w)$ if $w \in \domain(\gamma^{\avar:=\bvar_1})$ and $\delta'(w) =
%    \delta^{\bvar_1:=\cvar}(w)$ otherwise:
%    \begin{itemize}
%      \item from $\subrel{s,\delta^\chi,q}$ we obtain $\subrel{s',\delta_2,q'}$, where
%        $\delta_2(\avar) = \bvar_2$ and $\delta_2(w) = \delta^\chi(w)$ for all other $w$;
%      \item for all $w \in \FV(s')$: $\delta'(w) = \delta_2(w)$:
%        \begin{itemize}
%        \item if $w = \avar$ and $\avar = \bvar_1$, then PROBLEM
%        \end{itemize}
%      \item hence, by Lemma \ref{lem:substextend}, also $\subrel{s',\delta',q'}$
%    \end{itemize}
%  \end{itemize}
%  \ \\
%  NODIG: $\subrel{t', \delta^{\bvar_1:=\cvar}, u'}$ and
%  $\subrel{q',\eta^{\bvar_2:=\cvar}, u'}$
%
%
%\item ======================
%\item If $s = \meta{\avar}{s_1,\dots,s_k}$ then $\avar \in \Vbound$ so $\avar \notin
%  \domain(\gamma)$.  Hence, $t = \meta{\avar}{t_1,\dots,t_k}$ with $\subrel{s_i,\gamma,t_i}$ for
%  $1 \leq i \leq k$.  Moreover, $\delta^\chi(\avar) = \delta(\avar)$, so
%  $\aprel{\delta(\avar),[q_1,\dots,q_k],q}$, where $\subrel{s_i,\delta^\chi,q_i}$ for $1 \leq i
%  \leq k$.  We apply the induction hypothesis to find $u_1,\dots,u_k$ such that
%  $\subrel{t_i,\delta,u_i}$ and $\subrel{q_i,\eta,u_i}$ for $1 \leq i \leq k$.
%
%  Now, we have:
%  \begin{itemize}
%  \item $\domain(\eta) \subseteq \Vbound$ and $\FV(\delta(\avar)) \cap \domain(\eta) = \emptyset$
%    by assumption
%  \item $\aprel{\delta(\avar),[q_1,\dots,q_k],q}$
%  \item $\subrel{q_i,\eta,u_i}$ for $1 \leq i \leq k$
%  \end{itemize}
%  \ \\
%  Hence, by Lemma \ref{lem:boundsubsthelper}, there exists $u$ such that both
%  $\aprel{\delta(\avar),[u_1,\dots,u_k],u}$ and $\subrel{q,\eta,u}$.  We obtain
%  $\subrel{t,\delta,u}$ from $\aprel{\delta(\avar),[u_1,\dots,u_k],u}$ and
%  $\subrel{t_i,\delta,u_i}$ for all $i$.
%\end{itemize}
%\end{proof}

The next helper result roughly states that $a[\avar_1:=s_1,\dots,\avar_k:=s_k]\gamma =
(a\gamma')[\avar_1:=s_1\gamma,\dots,\avar_k:=s_k\gamma]$ if the $\avar_i$ do not occur in the range
of $\gamma$, where $\gamma'$ is the restriction of $\gamma$ to variables other than $\avar_1,\dots,
\avar_k$.  It is a bit more general, allowing also for a renaming of the variables $\avar_i$; we
will need it in this form for Lemma \ref{lem:combinesubsthelper}.

\begin{lemma}\label{lem:switchordercombihelper}
Assume given variables $\bvar_i,\cvar_i \in \Vbound$ for $1 \leq i \leq k$,
a substitution $\delta$,
and terms $a,b,t_1,\dots,t_k,v_1,\dots,v_k$ such that:
\begin{itemize}
\item $\subrel{a,[\bvar_1:=t_1,\dots,\bvar_k:=t_k],t}$
\item $\subrel{t_i,\delta,v_i}$ for $1 \leq i \leq k$ if $\bvar_i \neq \bvar_j$ for any $j > i$ and
  $\bvar_i \in \FV(a)$
\item $\subrel{a,\delta^{\vec{\bvar}:=\vec{\cvar}},b}$ where $\delta^{\vec{\bvar}:=\vec{\cvar}}$ is the
  substitution with $\delta^{\vec{\bvar}:=\vec{\cvar}}(\bvar_i) = \cvar_i$ if there is no $j > i$ with
  $\bvar_i = \bvar_j$, and $\delta^{\vec{\bvar}:=\vec{\cvar}}(\avar) = \delta(\avar)$ if $\avar \notin
  \{\bvar_1,\dots,\bvar_n\}$
\item each $\cvar_i \notin \FV(\delta(\avar))$ for any $\avar \in \FV(a) \setminus \{\bvar_1,\dots,\bvar_n\}$
\item if $\cvar_i = \cvar_j$ for some $j > i$, then $\bvar_i \notin \FV(a)$ or $\bvar_i = \bvar_l$ for some $l > i$ 
\end{itemize}
Then there exists $u$ such that $\subrel{t,\delta,u}$ and $\subrel{b,[\cvar_1:=v_1,\dots,\cvar_k:=
v_k],u}$.
\end{lemma}

\begin{proof}
We prove the result by induction on the size of $a$.  We use

We use Lemma \ref{lem:appsubstitute} for the case of $a$ being an application $h(s_1,\dots,s_n)$ with $n > 0$;
Lemma \ref{lem:substitutionrefl} for the case of a variable $\avar \notin \{\bvar_1,\dots,\avar_k\}$;
Lemma \ref{lem:substextend} and careful variable renaming for the case of an abstraction (we let
$a = \abs{\bvar_{k+1}}{a'}$ and $b = \abs{\cvar_{k+1}}{b'}$); and
Lemma \ref{lem:boundsubsthelper} for the case of a meta-application.
    % \begin{itemize}
    % \item If $a = h(s_1,\dots,s_n)$ with $n > 0$ then:
    %   \begin{itemize}
    %   \item $t = q_0 \cdot q_1 \cdots q_n$ with $\subrel{h,[\bvar_1:=t_1,\dots,\bvar_k:=t_k],q_0}$ and
    %     $\subrel{s_i,[\bvar_1:=t_1,\dots,\bvar_k:=t_k],q_i}$ for $1 \leq i \leq n$, and
    %   \item $b = w_0 \cdot w_1 \cdots w_n$ with $\subrel{h,\delta^{\vec{\bvar}:=\vec{\cvar}},w_0}$ and
    %     $\subrel{s_i,\delta^{\vec{\bvar}:=\vec{\cvar}},w_i}$ for $1 \leq i \leq n$.
    %   \end{itemize}
    %   \ \\
    %   Hence, by the induction hypothesis, there exist $u_0,\dots,u_n$ with
    %   $\subrel{q_i,\delta,u_i}$ and $\subrel{w_i,[\cvar_1:=v_1,\dots,\cvar_k:=v_k],u_i}$ for $0 \leq i
    %   \leq n$.  Choose $u := u_0 \cdot u_1 \cdots u_n$. Then:
    %   \begin{itemize}
    %   \item $\subrel{t,\delta,u}$ by Lemma \ref{lem:appsubstitute}, because $t = q_0 \cdots q_n$,
    %     $u = u_0 \cdots u_n$, and each $\subrel{q_i,\delta,u_i}$
    %   \item $\subrel{b,[\cvar_1:=v_1,\dots,\cvar_k:=v_k],u}$ also by Lemma \ref{lem:appsubstitute}.
    %   \end{itemize}
    % \item If $a = \afun$ then necessarily $t = b = \afun$ as well; we let $u := \afun$.
    % \item If $a = \bvar_i$ for some $i$, we can safely assume that there is no $j > i$ with $a =
    %   \bvar_j$ (otherwise we just consider $\bvar_j$).  Then $t = t_i$, and $b = \cvar_i$.
    %   By assumption, $\subrel{t_i,\delta,v_i}$ and since clearly $\bvar_i \in \FV(a)$, there is no
    %   $j > i$ with $\cvar_i = \cvar_j$.  Hence, we have $\subrel{b,[\cvar_1:=v_1,\dots,\cvar_k:=v_k],
    %   b_i}$ as well.  We are done choosing $u := v_i$.
    % \item If $a = \avar \in \V \setminus \{\bvar_1,\dots,\bvar_k\}$ then $t = \avar$ and $b =
    %   \delta(\avar)$.  Choose $u := \delta(\avar)$.  Then obviously $\subrel{t,\delta,u}$.  Also
    %   $\subrel{b,[\cvar_1:=u_1,\dots,\cvar_k:=u_k],u}$ by Lemma \ref{lem:substitutionrefl}, since
    %   $\cvar_1,\dots,\cvar_i \notin \FV(\delta(\avar))$ by assumption.
    % \item If $a$ is an abstraction, then we can write:
    %   \begin{itemize}
    %   \item $a = \abs{\bvar_{k+1}}{a'}$
    %   \item $t = \abs{\avar}{t'}$ with $\avar \notin \FV(a) \setminus \{\bvar_1,\dots,\bvar_k\}$ (X0);
    %     $\avar \notin \FV(t_i)$ for any $i$ with $\bvar_i \in \FV(a)$
    %     if there is no $j > i$ with $\bvar_j = \bvar_i$ (X1);
    %     moreover, $\subrel{a',[\bvar_1:=t_1,\dots,\bvar_k:=t_k,\bvar_{k+1}:=\avar],t'}$ (X2);
    % %  \item $b = \abs{\cvar_{k+1}}(b')$ with $\cvar_{k+1} \notin \FV(\delta^{\vec{\bvar}:=\vec{\cvar}}(
    % %    w))$ for any $w \in \FV(a)$
    % %    \begin{itemize}
    % %    \item[] (that is, for $w \in \FV(a') \setminus \{\bvar_{k+1}\}$ we have:
    % %      \begin{itemize}
    % %      \item if $w = \bvar_i$ such that no $j > i$ exists with $w = \bvar_j$: $\cvar_{k+1} \neq \cvar_i$
    % %      \item otherwise, $\bvar_{k+1} \notin \FV(\delta(w))$
    % %      \end{itemize}
    % %      \ \\)
    % %    \end{itemize}
    % %    \ \\and $\subrel{a',\delta^{\bvar_1:=\cvar_1,\dots,\bvar_{k+1}:=
    % %    \cvar_{k+1}},b'}$
    %   \item $b = \abs{\cvar_{k+1}}{b'}$ with $\cvar_{k+1} \notin \FV(\delta^{\bvar_1:=\cvar_1,\dots,
    %     \bvar_k:=\cvar_k}(w))$ for any $w \in \FV(a') \setminus \{\bvar_{k+1}\}$ (X3)
    %     and we have $\subrel{a',\delta^{\bvar_1:=\cvar_1,\dots,
    %     \bvar_{k+1}:=\cvar_{k+1}},b'}$ (X4)
    %   \end{itemize}
    %   \ \\
    %   Now let $\avar'$ be a variable in $\Vbound$ such that $\avar' \notin \{\bvar_1,\dots,\bvar_{k+1},
    %   \cvar_1,\dots,\cvar_{k+1}\}$
    %   and let $\delta' := [\avar:=\avar'] \cup [w := \delta(w) \mid w \in \domain(\delta) \setminus
    %   \{\avar\}]$.
    %   Then we can apply the induction hypothesis on $(\bvar_1,\dots,\bvar_{k+1}),(\cvar_1,\dots,
    %   \cvar_{k+1}),\delta',a',b',(t_1,\dots,t_k,\avar),(v_1,\dots,v_k,\avar')$:
    %   \begin{itemize}
    %   \item $\subrel{a',[\bvar_1:=t_1,\dots,\bvar_k:=t_k,\bvar_{k+1}:=\avar],t'}$ by (X2);
    %   \item for $1 \leq i \leq k$, if $\bvar_i \in \FV(a')$ and there is no $j \in \{i+1,\dots,k+1\}$
    %     with $\bvar_i = \bvar_j$, then both $\bvar_i \in \FV(a)$ (since $\bvar_i \in \FV(a') \setminus
    %     \{\bvar_{k+1}\}$) and $\subrel{t_i,\delta,v_i}$ is assumed, so since $\avar \notin \FV(t_i)$ by
    %     (X1) also $\subrel{t_i,\delta',v_i}$ by Lemma \ref{lem:substextend}; \\
    %     clearly also $\subrel{\avar,\delta',\avar'}$
    %   \item $\subrel{a',(\delta')^{\bvar_1:=\cvar_1,\dots,\bvar_{k+1}:=\cvar_{k+1}},b'}$ by (X3) and
    %     Lemma \ref{lem:substextend}: if $\avar \in \FV(a')$ then by (X0) $\avar \in \{\bvar_1,\dots,
    %     \bvar_{k+1}\}$, so the assignment is removed by the $[\bvar_1:=\cvar_1,\dots,\bvar_{k+1}:=
    %     \cvar_{k+1}]$ replacement;
    %   \item for $1 \leq i \leq k$ we have $\cvar_i \notin \FV(\delta(w))$ for any $w \in \FV(a)
    %     \setminus \{\bvar_1,\dots,\bvar_k\} = \FV(a') \setminus \{\bvar_1,\dots,\bvar_{k+1}\}$ and
    %     therefore also $\cvar_i \notin \FV(\delta'(w))$ if $w \neq \avar$; and $\cvar_i \notin
    %     \FV(\delta'(\avar))$ because $\delta(\avar) = \avar'$ which we chose unequal to all
    %     $\cvar_i$; \\
    %     we also have $\cvar_{k+1} \neq \avar' = \delta'(\avar)$, and $\cvar_i \notin \FV(\delta(w))$
    %     for any other $w \in \FV(a') \setminus \{\bvar_1,\dots,\bvar_{k+1}\}$ by (X3): on this
    %     domain, $\delta'(w) = \delta(w)$
    %   \item if $\cvar_i = \cvar_j$ for some $i < j \leq k$ then by assumption either $\bvar_i =
    %     \bvar_l$ for some $l > j$, or $\bvar_i \notin \FV(a) = \FV(a') \setminus \{\bvar_{k+1}\}$;
    %     so in the latter case, $\bvar_i \notin \FV(a')$ or $\bvar_i = \bvar_{k+1}$ \\
    %     otherwise, if $\cvar_i = \cvar_{k+1}$ for some $i < k+1$ then by (X3), $\bvar_i \notin
    %     \FV(a') \setminus \{\bvar_{k+1}\}$
    %   \end{itemize}
    %   \ \\
    %   Hence, by the induction hypothesis, there exists $u'$ such that $\subrel{t',\delta',u'}$ and
    %   $\subrel{b',[\cvar_1:=v_1,\dots,\cvar_k:=v_k,\cvar_{k+1}:=\avar'],u'}$.  We are done choosing
    %   $u := \abs{\avar'}{u'}$.
    % \item ========================
    % \item If $a = \meta{\avar}{s_1,\dots,s_n}$ then $\avar \in \Vfree$ so $\avar \notin \{\bvar_1,
    %   \dots,\bvar_k\}$.  Hence, $t = \meta{\avar}{q_1,\dots,q_n}$ with $\subrel{s_i,[\bvar_1:=t_1,
    %   \dots,\bvar_k:=t_k],q_i}$ for $1 \leq i \leq n$.  Moreover, $\delta^\chi(\avar) = \delta(\avar)$
    %   since $\avar \notin \{\cvar_1,\dots,\cvar_n\}$.
    %   
    %   First suppose $\avar \notin \domain(\delta)$.  Then $b = \meta{\avar}{w_1,\dots,w_n}$ with
    %   $\subrel{s_i,\delta^\chi,w_i}$ for $1 \leq i \leq n$.  By the induction hypothesis, there exist
    %   $u_1,\dots,u_n$ such that $\subrel{q_i,\delta,u_i}$ and
    %   $\subrel{w_i,[\bvar_1:=v_1,\dots,\bvar_n:=v_n],u_i}$ for $1 \leq i \leq n$; we are done
    %   choosing $u = \meta{\avar}{u_1,\dots,u_n}$.
    % 
    %   If $\avar \in \domain(\delta)$ then $\aprel{\delta(\avar),[w_1,\dots,w_n],b}$ for some
    %   $\vec{w}$ with $\subrel{s_i,\delta^\chi,w_i}$ for $1 \leq i \leq n$.
    %   Here, too, we apply the induction hypothesis to find $u_1,\dots,u_k$ such that
    %   $\subrel{q_i,\delta,u_i}$ and $\subrel{w_i,[\cvar_1:=v_1,\dots,\cvar_k:=v_k],u_i}$ for
    %   $1 \leq i \leq k$.
    %   Now, we have:
    %   \begin{itemize}
    %   \item $\domain([\cvar_1:=v_1,\dots,\cvar_k=v_k]) \subseteq \Vbound$ and
    %     $\FV(\delta(\avar)) \cap \{\cvar_1,\dots,\cvar_k\} = \emptyset$
    %     by assumption
    %   \item $\aprel{\delta(\avar),[w_1,\dots,w_n],b}$
    %   \item $\subrel{w_i,[\cvar_1:=v_1,\dots,\cvar_k=v_k],u_i}$ for $1 \leq i \leq n$
    %   \end{itemize}
    %   \ \\
    %   Hence, by Lemma \ref{lem:boundsubsthelper}, there exists $u$ such that both
    %   $\aprel{\delta(\avar),[u_1,\dots,u_k],u}$ and $\subrel{b,[\cvar_1:=v_1,\dots,\cvar_k=v_k],u}$.
    %   We obtain $\subrel{t,\delta,u}$ from $\aprel{\delta(\avar),[u_1,\dots,u_k],u}$ and
    %   $\subrel{q_i,\delta,u_i}$ for all $i$.
    %   \qedhere
    % \end{itemize}
\end{proof}

With these preparations we are finally ready to show that $s(\gamma\delta) = (s\gamma)\delta$.
Using the $\subrel{}$ relation, this is formulated as follows:

\begin{lemma}\label{lem:combinesubsthelper}
Let $s,t$ be terms, and $\gamma,\delta,\eta$ be substitutions, such that for all variables $x$ in
$\FV(s)$: $\subrel{\gamma(x),\delta,\eta(x)}$.  Then,
  if $\subrel{s,\gamma,t}$ then there exists $u$ such that both $\subrel{t,\delta,u}$ and
  $\subrel{s,\eta,u}$.
\end{lemma}

\begin{proof}
By induction on the definition of $\subrel{}$ (or, put differently, induction first on the presence
of $\Vfree$ variables in $\domain(\gamma)$, second on the size of $s$).

\begin{itemize}
\item If $s = h(s_1,\dots,s_n)$ with $n > 0$, then there exist $q, t_1,\dots,t_n$ such that
  $\subrel{h,\gamma,q}$ and $\subrel{s_i,\gamma,t_i}$ for all $i$, and $t = q \cdot t_1 \cdots
  t_n$.  By the induction hypothesis, there exist $w$ such that both
  $\subrel{q,\delta,w}$ and $\subrel{h,\eta,w}$, and $u_1,\dots,u_n$ such that for all $i$
  both $\subrel{t_i,\delta,u_i}$ and $\subrel{s_i,\eta,u_i}$.
  Choose $u := w \cdot u_1 \cdots u_n$.
  Then $\subrel{t,\delta,u}$ by Lemma \ref{lem:appsubstitute}, and $\subrel{s,\eta,u}$ by
  definition.
\item If $s = \afun$ then $t = \afun$; we choose $u := \afun$ as well.
\item If $s = \avar$ then $t = \gamma(\avar)$ and $\subrel{t,\delta,\eta(\avar)}$ by assumption.
  We let $u := \eta(\avar)$.
\item If $s = \abs{\avar}{s'}$ then $t = \abs{\bvar}{t'}$, with:
  \begin{itemize}
  \item $\bvar \notin \FV(\gamma(w))$ for any $w \in \FV(s)$;
  \item $\subrel{s',\gamma^{\avar:=\bvar},t'}$
    where $\gamma^{\avar:=\bvar} = [\avar:=\bvar] \cup [w:=\gamma(w) \mid w \in \domain(\gamma)
    \setminus \{\avar\}]$;
  \end{itemize}
  \ \\
  Let $\cvar$ be a variable of the same type as $\avar$, such that:
  \begin{itemize}
  \item $\cvar \notin \FV(\delta(w))$ for any $w \in \FV(t)$, and
  \item $\cvar \notin \FV(\eta(w))$ for any $w \in \FV(s)$.
  \end{itemize}
  \ \\
  Since $\cvar$ is only required to be unequal to a finite number of variables, and $\Vbound$ has
  infinitely many elements of each type, such a variable can always be found.

  Now let $\delta^{\bvar:=\cvar} = [\bvar:=\cvar] \cup [w:=\delta(w) \mid w \in \domain(\delta)
  \setminus \{\bvar\}]$ and let $\eta^{\avar:=\cvar} = [\avar:=\cvar] \cup [w:=\eta(w) \mid w \in
  \domain(\eta) \setminus \{\avar\}]$.  Then clearly $\subrel{\gamma^{\avar:=\bvar}(w),\delta^{
  \bvar:=\cvar},\eta^{\avar:=\cvar}(w)}$ for all $w \in \FV(s')$:
  \begin{itemize}
  \item for $w = \avar$ we have $\subrel{\bvar,\delta^{\bvar:=\cvar},\cvar}$;
  \item and for all other $w$ we have $\subrel{\gamma(w),\delta^{\bvar:=\cvar},\eta(w)}$ by Lemma
    \ref{lem:substextend},because $\bvar \notin \FV(\gamma(w))$ when $w \in \FV(s) = \FV(s')
    \setminus \{\avar\}$ (by choice of $\bvar$).
  \end{itemize}
  \ \\
  Hence, we can apply the induction hypothesis on $\subrel{s',\gamma^{\avar:=\bvar},t'}$ to obtain
  $u'$ such that both $\subrel{t',\delta^{\bvar;=\cvar},u'}$ and $\subrel{s',\eta^{\avar:=\cvar},
  u'}$.  We choose $u := \abs{\cvar}{u'}$.  Due to the requirements on $\cvar'$ we have both
  $\subrel{t,\delta,u}$ and $\subrel{s,\eta,u}$.
\item If $s = \meta{\avar}{s_1,\dots,s_k}$ and $\avar \notin \domain(\gamma)$, then
  $t = \meta{\avar}{t_1,\dots,t_k}$ with $\subrel{s_i,\gamma,t_i}$ for all $i$.  By the induction
  hypothesis we find $u_1,\dots,u_k$ such that both $\subrel{t_i,\delta,u_i}$ and $\subrel{s_i,
  \eta,u_i}$ for all $i$.

  Let $u$ be such that $\aprel{\delta(\avar),[u_1,\dots,u_k],u}$ (this exists by Lemma
  \ref{lem:substdefined}).  This property both gives $\subrel{t,\delta,u}$ and $\subrel{s,\eta,u}$.
\item =================
\item If $s = \meta{\avar}{s_1,\dots,s_k}$ and $\avar \in \domain(\gamma)$, we can write
  $\gamma(\avar) = \abs{\bvar_1 \dots \bvar_n}{a}$ with either $n = k$ or $n < k$ and $a$ is not an
  abstraction.
  Then there exist $t_1,\dots,t_k,t'$ such that $\subrel{s_i,\gamma,t_i}$ for all $i$,
  $t = t' \cdot t_{n+1} \cdots t_k$ and $\subrel{a,[\bvar_1:=t_1,\dots,\bvar_n:=t_n],t'}$.
  By the induction hypothesis on each $s_i$, there exist $u_1,\dots,u_k$ such that both
  $\subrel{t_i,\delta,u_i}$ and $\subrel{s_i,\eta,u_i}$ for all $i \in \{1,\dots,k\}$.

  By Lemma \ref{lem:abssubst} and $\subrel{\gamma(\avar),\delta,\eta(\avar)}$, we can write
  $\eta(\avar) = \abs{\cvar_1 \cdots \cvar_n}{b}$ such that:

  PROBLEEM: wat als $\eta(\avar)$ een langere abstractie is?  Dit is absoluut mogelijk\dots
\end{itemize}

TODO
\end{proof}


\begin{lemma}\label{lem:combinesubst}
Always $s(\gamma\delta) =_\alpha (s\gamma)\delta$.
\end{lemma}

\begin{proof}
Let $\eta = \gamma\delta$; that is, $\domain(\eta) = \domain(\gamma) \cup \domain(\delta)$ and
$\eta(x) = \gamma(x)\delta$ for all $x \in \domain(\eta)$ and $\eta(x) = \delta(x)$ for aother $x$
in its domain.  We will prove that there exists $t$ such that $s\gamma = t$ and $t\delta = s\eta$.
The desired result then follows by Lemma \ref{lem:substitutionalpha}.
\end{proof}


\end{document}

%\begin{lemma}\label{lem:combinesubst:innocent}
%Let $s,t$ be terms, and $\gamma,\delta,\eta$ substitutions such that:
%\begin{itemize}
%\item $\domain(\gamma) \subseteq \Vbound$ and $\gamma(x) \in \Vbound$ for all
%  $x \in \domain(\gamma)$
%\item $\domain(\delta) = \{ \gamma(x) \mid x \in \domain(\gamma) \}$
%\item $\domain(\delta) \cap \domain(\eta) = \emptyset$
%\item $\domain(\zeta) = \domain(\gamma) \cup \domain(\eta)$
%\item $\zeta(\avar) = \delta(\gamma(\avar))$ for $\avar \in \domain(\gamma)$ and
%  $\zeta(\avar) = \eta(\avar)$ for $\avar \in \domain(\zeta) \setminus
%    \domain(\gamma)$
%\item $\FV(s) \cap \domain(\delta) \subseteq \domain(\gamma)$
%\end{itemize}
%If $\subrel{s,\gamma,t}$ and $\subrel{t,\delta \cup \eta,u}$ then
%$\subrel{s,\zeta,u}$.
%\end{lemma}
%
%\begin{proof}
%By induction on the size of $s$.
%\begin{itemize}
%\item If $s = \afun$ then $t = \afun$ and $u = \afun$, so indeed
%  $\subrel{s,\zeta,u}$.
%\item If $s = \avar \in \domain(\gamma)$ then $t = \gamma(\avar)$, a variable,
%  so $u = \delta(\gamma(\avar)) = \zeta(\avar)$; indeed $\subrel{\avar,\zeta,
%  u}$.
%\item If $s = \avar \in \V \setminus \domain(\gamma)$ then $t = \avar$.
%  If also $\avar \notin \domain(\delta) \cup \domain(\eta)$ then $\avar \notin
%  \domain(\zeta)$ as well, and we have $u = \avar$; then indeed
%  $\subrel{\avar,\zeta,u}$ as well.
%  Otherwise, we must have $\avar \in \domain(\eta)$ by the last requirement,
%  so $u = \eta(\avar) = \zeta(\avar)$, and indeed $\subrel{\avar,\zeta,u}$.
%\item If $s = h(s_1,\dots,s_n)$ with $n > 0$ then $t = w \cdot t_1 \cdots t_n$
%  for some $w,\vec{t}$ such that $\subrel{h,\gamma,w}$ and $\subrel{s_i,\gamma,
%  t_i}$ for all $i$.  By Lemma~\ref{lem:appsubstitute} we can write
%  $u = q \cdot u_1 \cdots u_n$ such that $\subrel{w,\delta \cup \eta,q}$ and
%  $\subrel{t_i,\delta \cup \eta,u_i}$ for all $i$, and by the induction
%  hypothesis $\subrel{h,\zeta,q}$ and $\subrel{s_i,\zeta,u_i}$ for all $i$
%  follows.  By definition then $\subrel{s,\zeta,u}$.
%\item If $s = \tuple{s_1}{s_n}$ then $t = \tuple{t_1}{t_n}$ and $u = \tuple{
%  u_1}{u_n}$, where $\subrel{s_i,\gamma,t_i}$ and $\subrel{t_i,\delta \cup
%  \eta,u_i}$ for all $i$.  By the induction hypothesis $\subrel{s_i,\zeta,u_i}$
%  follows.
%\item If $s = \abs{\avar}{s'}$ then $t = \abs{\bvar}{t'}$
%  and $u =\abs{\cvar}{u'}$ with:
%  \begin{itemize}
%  \item $\bvar \notin \FV(\gamma(a))$ for any $a \in \FV(s)$
%  \item $\subrel{s',\gamma^{\avar:=\bvar},t'}$ where $\gamma^{\avar:=\bvar} =
%  [\avar:=\bvar] \cup [a:=\gamma(a) \mid a \in \domain(\gamma) \setminus
%  \{\avar\}]$
%  \item $\cvar \notin \FV(\delta \cup \eta)(a)$ for any $a \in \FV(t) \cup
%    (\FMV(t) \cap \domain(\delta \cup \eta))$
%  \item $\subrel{t',\delta\eta^{\bvar:=\cvar},u'}$, where $\delta\eta^{\bvar:=
%    \cvar} = [\bvar:=\cvar] \cup [a:=(\delta \cup \eta)(a) \mid a \in
%    \domain(\delta \cup \eta) \setminus \{ \bvar \}]$ \\
%  \end{itemize}
%  Let us define:
%  \begin{itemize}
%  \item $\delta^{\neg\gamma(\avar),\bvar:=\cvar} = [\bvar:=\cvar] \cup
%    [a:=\delta(a) \mid a \in \domain(\delta) \setminus \{ \gamma(\avar),\bvar\}]$;
%  \item $\eta^{\neg\bvar} := [a:=\eta(a) \mid a \in \domain(\eta) \setminus
%    \{\bvar\}$.
%  \item $\zeta' = [a:=\delta(\gamma(a)) \mid a \in \domain(\gamma) \setminus
%    \{\avar\}] \cup \\{}
%    [\avar:=\cvar] \cup [a:=\eta(a) \mid a \in \domain(\eta) \setminus
%    (\domain(\gamma) \cup \{\avar,\bvar\})]$. \\
%  \end{itemize}
%  Then observe that:
%  \begin{itemize}
%  \item $\domain(\gamma^{\avar:=\bvar}) \subseteq \Vbound$ and
%    $\gamma^{\avar:=\bvar}(a) \in \Vbound$ for all $a \in \domain(
%    \gamma^{\avar:=\bvar})$
%  \item $\domain(\delta^{\neg\gamma(\avar),\bvar:=\cvar}) = (\domain(\delta)
%    \setminus \{\gamma(\avar)\}) \cup \{\bvar\} = \{ \gamma^{\avar:=\bvar}(a)
%    \mid a \in \domain(\gamma) \}$
%  \item $\domain(\delta^{\neg\gamma(\avar),\bvar:=\cvar}) \cap
%    \domain(\eta^{\neg\bvar}) = ((\domain(\delta) \setminus \{\gamma(\avar)\})
%    \cup \{\bvar\}) \cap (\domain(\eta) \setminus \{\bvar\}) =
%    (\domain(\delta) \setminus \{\gamma(\avar)\}) \cap (\domain(\eta) \setminus
%    \{\bvar\}) \subseteq \domain(\delta) \cap \domain(\eta) = \emptyset$
%  \item $\domain(\zeta') = (\domain(\gamma) \setminus \{\avar\}) \cup
%    \{\avar\} \cup (\domain(\eta) \setminus (\domain(\gamma) \cup \{\avar,
%    \bvar\})) = \domain(\gamma) \cup \{\avar\} \cup (\domain(\eta) \setminus
%    \{\bvar\}) = \domain(\gamma^{\avar:=\bvar}) \cup \domain(\eta^{\neg\bvar})$
%  \item =================
%  \item $\zeta'(a) = \delta^{\neg\gamma(\avar),\bvar:=\cvar}(\gamma^{\avar:=
%    \bvar}(a))$ for $a \in \domain(\gamma^{\avar:=\bvar})$:
%    \begin{itemize}
%    \item if $a = \avar$ then $\zeta'(a) = \cvar$ and
%      $\delta^{\neg\gamma(\avar),\bvar:=\cvar}(\gamma^{\avar:=\bvar}(a)) =
%      \delta^{\neg\gamma(\avar),\bvar:=\cvar}(\bvar) = \cvar$
%    \item if $a \neq \avar$ then $a \in \domain(\gamma)$, so $\zeta'(a) =
%      \delta(\gamma(a))$ and $\delta^{\neg\gamma(\avar),\bvar:=
%      \cvar}(\gamma^{\avar:=\bvar}(a)) = \delta^{\neg\gamma(\avar),\bvar:=
%      \cvar}(\gamma(a))$ TODO: $\gamma(a) \neq \gamma(\avar)$ and
%      $\gamma(a) \neq \bvar$\dots
%    \end{itemize}
%  \end{itemize}
%\end{itemize}
%\end{proof}




%\section{Unconstrained first-order term rewriting}
%
%Although first-order (many-sorted) term rewriting systems can be seen as a kind of higher-order
%term rewriting system, we will present their definition separately first, and later explain how
%they can be viewed as part of the larger framework.
%
%\subsection{Terms} When considering \emph{first-order} term rewriting, we limit interest to $\F$
%with the following property: for every $(\afun : \atype) \in \F$ we have $\order(\atype) \leq 1$.
%First-order terms are those expressions $s$ such that $s : \asort$ can be derived for some
%\emph{base type} $\asort$ using the following clauses:
%\begin{itemize}
%\item if $(\afun : \atype_1 \arrtype \dots \arrtype \atype_n \arrtype \asort) \in \F$ and
%  $s_1 : \atype_1,\dots,s_n : \atype_n$ then $\afun(s_1,\dots,s_n) : \asort$;
%\item if $(\avar : \asort) \in \V$ then $\avar : \asort$.
%\end{itemize}
%We denote $\FOTerms(\F,\V)$ for the set of all first-order terms $s$.
%A first-order term of the form $\afun(s_1,\dots,s_n)$ is called a \emph{functional term} and
%$\afun$ is its root; a term $\avar$ is simply called a variable.
%If $s : \asort$ then we say that $\asort$ is the type of $s$; it is clear from the definitions
%above that each term has a unique type (which is a base type).
%
%The set $\FV(s)$ of \emph{variables} of a term $s$ is inductively defined as follows:
%\begin{itemize}
%\item $\FV(\afun(s_1,\dots,s_n)) = \FV(s_1) \cup \dots \cup \FV(s_n)$;
%\item $\FV(\avar) = \{ \avar \}$.
%\end{itemize}
%%That is, $\FV(s)$ contains all variables in $s$.
%
%\bigskip
%The \emph{subterm} relation $\subtermeq$ is defined as follows:
%\begin{itemize}
%\item $s \subtermeq s$ for all $s$;
%\item $s \subtermeq \afun(s_1,\dots,s_n)$ if $s \subtermeq s_i$ for some $i$.
%\end{itemize}
%If $s \subtermeq t$ we say that $s$ \emph{is a subterm of} $t$.
%
%\subsection{Substitution}
%
%A substitution is a function $\gamma$ that maps each variable $\avar \in \V$ to a term
%$\gamma(\avar)$ of the same type.  A substitution is applied to an arbitrary first-order term as
%follows:
%\begin{itemize}
%\item $\afun(s_1,\dots,s_n)\gamma = \afun(s_1\gamma,\dots,s_n\gamma)$;
%\item $\avar\gamma = \gamma(\avar)$.
%\end{itemize}
%
%The \emph{domain} $\domain(\gamma)$ of a substitution $\gamma$ is the set of all variables $x$
%such that $\gamma(x) \neq x$.
%We denote $[x_1:=s_1,\dots,x_n:=s_n]$ for the substitution $\gamma$ with $\gamma(x_i) = s_i$ for
%$1 \leq i \leq n$ and $\gamma(y) = y$ for $y \notin \{x_1,\dots,x_n\}$.
%For two substitutions $\gamma$ and $\delta$, we let $\gamma\delta$ denote the substitution that
%maps each variable $x$ to $\gamma(x)\delta$.
%
%\subsection{Positions}
%
%The \emph{positions} of a given first-order term are the paths to specific subterms, defined as
%follows:
%
%\begin{itemize}
%\item $\Positions(\afun(s_1,\dots,s_n)) = \{ \epsilon \} \cup \{ i \cdot p \mid 1 \leq i \leq n
%  \wedge p \in \Positions(s_i) \}$;
%\item $\Positions(\avar) = \{ \epsilon \}$.
%\end{itemize}
%Note that positions are associated to a term; thus, not every integer sequence is a position.
%
%For a term $s$ and a position $p \in \Positions(s)$, the \emph{subterm of $s$ at position $p$},
%denoted $s|_p$, is defined as follows:
%\begin{itemize}
%\item $s|_\epsilon = s$;
%\item $\afun(s_1,\dots,s_n)|_{i \cdot p} = s_i|_p$;
%\end{itemize}
%
%Note that $t \subtermeq s$ if and only if there is some position $p \in \Positions(s)$ with
%$t = s|_p$.
%If $s|_p$ has the same type as some term $t$, then $s[t]_p$ denotes $s$ with the subterm at position
%$p$ replaced by $t$.  Formally, $s[t]_p$ is obtained as follows:
%\begin{itemize}
%\item $s[t]_p = t$;
%\item $\afun(s_1,\dots,s_n)[t]_{i \cdot p} = \afun(s_1,\dots,s_{i-1},s_i[t]_p,s_{i+1},\dots,s_n)$.
%\end{itemize}
%
%\subsection{Rules and reduction}
%
%A rule is a pair $\ell \arrz r$ of two terms with the same type.
%For a given set of rules $\Rules$, the reduction relation $\arr{\Rules}$ is given by:
%\begin{itemize}
%\item if there exist $\ell \arrz r \in \Rules$ and $p \in \Positions(s)$ and substitution $\gamma$
%  such that $s|_p = \ell\gamma$, then $s \arr{\Rules} s[r\gamma]_p$.
%\end{itemize}
%
%\bigskip
%A \emph{first-order term rewriting system (TRS)} is an abstract rewriting system of the form
%$(\FOTerms(\F,\V),\arr{\Rules})$.
%
%In principle, we have defined a \emph{many-sorted} TRS here; a traditional unsorted TRS is obtained
%by limiting interest to the case $\Sorts = \{ \unitsort \}$.


