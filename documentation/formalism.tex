\documentclass{lmcs}
\pdfoutput=1

\usepackage{enumerate}
\usepackage[colorlinks=true]{hyperref}
\usepackage{tikz}
\usepackage{amssymb}
\usepackage{xcolor,latexsym,amsmath,extarrows,alltt}
\usepackage{xspace}
\usepackage{booktabs}
\usepackage{mathtools}
\usepackage{enumitem}
\usepackage{stmaryrd}
\usepackage{microtype}

\theoremstyle{theorem}\newtheorem{theorem}{Theorem}
\theoremstyle{theorem}\newtheorem{lemma}[theorem]{Lemma}
\theoremstyle{theorem}\newtheorem{corollary}[theorem]{Corollary}
\theoremstyle{definition}\newtheorem{definition}[theorem]{Definition}
\theoremstyle{definition}\newtheorem{example}[theorem]{Example}

\newcommand{\N}{\mathbb{N}}
\newcommand{\F}{\mathcal{F}}
\newcommand{\V}{\mathcal{V}}
\newcommand{\Sorts}{\mathcal{S}}
\newcommand{\Types}{\mathcal{Y}}
\newcommand{\Terms}{\mathcal{T}}
\newcommand{\ATerms}{\mathcal{T}_{\mathcal{A}}}
\newcommand{\FOTerms}{\mathcal{T}_{\mathcal{FO}}}
\newcommand{\Rules}{\mathcal{R}}
\newcommand{\FV}{\mathit{Var}}

\newcommand{\domain}{\mathtt{dom}}
\newcommand{\order}{\mathit{order}}

\newcommand{\asort}{\iota}
\newcommand{\bsort}{\kappa}
\newcommand{\atype}{\sigma}
\newcommand{\btype}{\tau}
\newcommand{\ctype}{\pi}
\newcommand{\dtype}{\alpha}
\newcommand{\identifier}[1]{\mathtt{#1}}
\newcommand{\afun}{\identifier{f}}
\newcommand{\bfun}{\identifier{g}}
\newcommand{\cfun}{\identifier{h}}
\newcommand{\avar}{x}
\newcommand{\bvar}{y}
\newcommand{\cvar}{z}

\newcommand{\abs}[2]{\lambda #1.#2}

\newcommand{\arity}{\mathit{arity}}
\newcommand{\head}{\mathsf{head}}
\newcommand{\arrtype}{\rightarrow}
\newcommand{\arrz}{\Rightarrow}
\newcommand{\arr}[1]{\arrz_{#1}}
\newcommand{\arrr}[1]{\arr{#1}^*}
\newcommand{\subtermeq}{\unlhd}
\newcommand{\headsubtermeq}{\unlhd_{\bullet}}
\newcommand{\supterm}{\rhd}
\newcommand{\suptermeq}{\unrhd}

\newcommand{\symb}[1]{\mathtt{#1}}

\newcommand{\nul}{\symb{0}}
\newcommand{\one}{\symb{1}}
\newcommand{\nil}{\symb{nil}}
\newcommand{\cons}{\symb{cons}}
\newcommand{\strue}{\symb{true}}
\newcommand{\sfalse}{\symb{false}}
\newcommand{\suc}{\symb{s}}
\newcommand{\map}{\symb{map}}
\newcommand{\bool}{\symb{bool}}
\newcommand{\nat}{\symb{nat}}
\newcommand{\lijst}{\symb{list}}

\newcommand{\cora}{\textsf{CORA}\xspace}

\newcommand{\secshort}{\S}
\newcommand{\myparagraph}[1]{\paragraph{\textbf{#1}}}

\setlength{\parindent}{0pt}
\setlength{\parskip}{\bigskipamount}
\setlist[itemize]{topsep=-\bigskipamount}

\begin{document}

\title{COnstrained Rewriting Analyser: formalism}
\author{Cynthia Kop}
\address{Department of Software Science, Radboud University Nijmegen}
\email{C.Kop@cs.ru.nl}

\maketitle

\begin{abstract}
\cora\ is a tool meant to analyse constrained term rewriting systems, both
first-order and higher-order.  This document explains the underlying formalism.
\end{abstract}

\section{Types}

We fix a set $\Sorts$ of \emph{sorts} and define the set $\Types$ of \emph{types} inductively:
\begin{itemize}
\item all elements of $\Sorts$ are types (also called \emph{base types});
\item if $\atype,\btype \in \Types$ then $\atype \arrtype \btype$ is also a type (called an arrow
  type).
\end{itemize}
The arrow operator $\arrtype$ is right-associative, so all types can be denoted in a form
$\atype_1 \arrtype \dots \arrtype \atype_m \arrtype \asort$ with $\asort \in \Sorts$; we say the
\emph{arity} of this type is $m$, and the \emph{output sort} is $\asort$.

The \emph{order} of a type is recursively defined as follows:
\begin{itemize}
\item for $\asort \in \Sorts$: $\order(\asort) = 0$;
\item for arrow types: $\order(\atype \arrtype \btype) = \max(\order(\atype) + 1,\order(\btype))$.
\end{itemize}

\section{Terms}

We fix a set $\F$ of \emph{function symbols} and a set $\V$ of \emph{variables}.  Each function
symbol and variable is a typed constant.  Notation: $\afun \in \F$ or $(\afun : \atype) \in \F$
if we wish to explicitly refer to the type (but the type should be considered implicit in the
function symbol and variable).

Terms are those expressions $s$ such that $s : \atype$ can be derived for some $\atype \in \Types$
using the following clauses:
\begin{itemize}
\item if $(\afun : \atype_1 \arrtype \dots \arrtype \atype_n \arrtype \btype) \in \F$ and
  $s_1 : \atype_1,\dots,s_n : \atype_n$ then $\afun(s_1,\dots,s_n) : \btype$;
\item if $(\avar : \atype_1 \arrtype \dots \arrtype \atype_n \arrtype \btype) \in \V$ and
  $s_1 : \atype_1,\dots,s_n : \atype_n$ then $\avar(s_1,\dots,s_n) : \btype$.
%\item if $(\avar : \atype) \in \V$ and $s : \btype$ then $\abs{\avar}{s} : \atype \arrtype \btype$.
\end{itemize}
A term of the form $\afun(s_1,\dots,s_n)$ is called a \emph{functional term} and $\afun$ is its
root.
A term of the form $\avar(s_1,\dots,s_n)$ is called a \emph{var term}, and $\avar$ is its variable.
If $s : \atype$ then we say that $\atype$ is the type of $s$; it is clear from the definitions
above that each term has a unique type.
We denote $\ATerms(\F,\V)$ for the set of all terms $s$.

Note that $n$ above is not required to be maximal; for example, if $\symb{greater} : \mathtt{int}
\arrtype \mathtt{int} \arrtype \mathtt{bool}$, then each of $\symb{greater}(),
\symb{greater}(\avar)$ and $\symb{greater}(\avar,\bvar)$ are terms (with distinct types).
When no arguments are given (i.e., a term $\afun()$ or $\avar()$), we may omit the brackets and
just denote $\afun$ or $\avar$ for the term.  In this sense, $\F,\V \subseteq \Terms(\F,\V)$.  A
term $\avar$ can simply be called a variable (but is also still a var term); a term $\afun$ may be
called a constant.

The \emph{subterm} relation $\subtermeq$ is defined as follows:
\begin{itemize}
\item $s \subtermeq s$ for all $s$;
\item $s \subtermeq \afun(s_1,\dots,s_n)$ if $s \subtermeq s_i$ for some $i$;
\item $s \subtermeq \avar(s_1,\dots,s_n)$ if $s \subtermeq s_i$ for some $i$.
\end{itemize}
If $s \subtermeq t$ we say that $s$ \emph{is a subterm of} $t$.  If $\afun(s_1,\dots,s_n) \subtermeq
t$ then we say that for $0 \leq i \leq n$: $\afun(s_1,\dots,s_i)$ \emph{is a head-subterm of} $t$,
notation $\afun(s_1,\dots,s_i) \headsubtermeq t$.

A term $s$ is a \emph{pattern} if for every subterm $t \subtermeq s$ we have: if $t$ is a var term,
then $t$ is a variable.  That is: for every occurrence of $\avar(s_1,\dots,s_n)$ in $s$ we have $n
= 0$.
A term $s$ is \emph{first-order} if (1) it is a pattern, and (2) for every subterm $t \subtermeq s$:
the type of $t$ is a base type.
The set of all first-order terms over $\F$ and $\V$ is denoted $\FOTerms(\F,\V)$.

\section{Application}

A term $s : \atype_1 \arrtype \dots \arrtype \atype_m \arrtype \btype$ can be applied to a sequence
$[t_1,\dots,t_m]$ of terms, provided $t_1 : \atype_1,\dots,t_m : \atype_m$, as follows:
\begin{itemize}
\item if $s = \afun(s_1,\dots,s_n)$ then $s \cdot [t_1,\dots,t_m] = \afun(s_1,\dots,s_n,t_1,\dots,
  t_m)$;
\item if $s = \avar(s_1,\dots,s_n)$ then $s \cdot [t_1,\dots,t_m] = \avar(s_1,\dots,s_n,t_1,\dots,
  t_m)$.
\end{itemize}
When $m = 1$, we can also write $s \cdot t$ instead of $s \cdot [t]$.

\begin{lemma}\label{lem:applicative_notation}
The set $\ATerms(\F,\V)$ is the smallest set such that:
\begin{itemize}
\item $\F \cup \V \subseteq \ATerms(\F,\V)$;
\item if $s,t \in \ATerms(\F,\V)$ and $s : \atype \arrtype \btype$ and $t : \atype$ then
  $s \cdot t \in \btype$.
\end{itemize}
\end{lemma}

\begin{proof}
Trivial.
\end{proof}

Lemma~\ref{lem:applicative_notation} shows that are terms are simply applicative terms
constructed in the traditional way; however, for convenience we denote them in a functional
notation.

\section{Substitution}

A substitution is a function $\gamma$ that maps each variable $\avar \in \V$ to a term
$\gamma(\avar)$ of the same type.  A substitution is applied to an arbitrary term as follows:
\begin{itemize}
\item $\afun(s_1,\dots,s_n)\gamma = \afun(s_1\gamma,\dots,s_n\gamma)$;
\item $\avar(s_1,\dots,s_n)\gamma = \gamma(\avar) \cdot [s_1\gamma,\dots,s_n\gamma]$.
\end{itemize}

The \emph{domain} $\domain(\gamma)$ of a substitution $\gamma$ is the set of all variables $x$
such that $\gamma(x) \neq x$.
We denote $[x_1:=s_1,\dots,x_n:=s_n]$ for the substitution $\gamma$ with $\gamma(x_i) = s_i$ for
$1 \leq i \leq n$ and $\gamma(y) = y$ for $y \notin \{x_1,\dots,x_n\}$.
For two substitutions $\gamma$ and $\delta$, we let $\gamma\delta$ denote the substitution that
maps each variable $x$ to $\gamma(x)\delta$.

\section{Positions}

The \emph{positions} of a given term are the paths to specific subterms, defined as follows:

\begin{itemize}
\item $\mathit{Positions}(\afun(s_1,\dots,s_n)) = \{ \epsilon \} \cup \{ i \cdot p \mid 1 \leq i
  \leq n \wedge p \in \mathit{Positions}(s_i) \}$;
\item $\mathit{Positions}(\avar(s_1,\dots,s_n)) = \{ \epsilon \} \cup \{ i \cdot p \mid 1 \leq i
  \leq n \wedge p \in \mathit{Positions}(s_i) \}$.
\end{itemize}
Note that positions are associated to a term; thus, not every integer sequence is a position.

For a term $s$ and a position $p \in \mathit{Positions}(s)$, the \emph{subterm of $s$ at position
$p$}, denoted $s|_p$, is defined as follows:
\begin{itemize}
\item $s|_\epsilon = s$;
\item $\afun(s_1,\dots,s_n)|_{i \cdot p} = s_i|_p$;
\item $\avar(s_1,\dots,s_n)|_{i \cdot p} = s_i|_p$.
\end{itemize}

Note that $t \subtermeq s$ if and only if there is some position $p \in \mathit{Positions}(s)$ with
$t = s|_p$.

If $s|_p$ has the same type as some term $t$, then $s[t]_p$ denotes $s$ with the subterm at position
$p$ replaced by $t$.  Formally, $s[t]_p$ is obtained as follows:
\begin{itemize}
\item $s[t]_p = t$;
\item $\afun(s_1,\dots,s_n)[t]_{i \cdot p} = \afun(s_1,\dots,s_{i-1},s_i[t]_p,s_{i+1},\dots,s_n)$;
\item $\avar(s_1,\dots,s_n)[t]_{i \cdot p} = \avar(s_1,\dots,s_{i-1},s_i[t]_p,s_{i+1},\dots,s_n)$.
\end{itemize}

\section{Rules and reduction}

A rule $\rho$ is a pair $\ell \arrz r$ of two terms with the same type.
A rule generates the set $\mathit{Pairs}_\rho := \{ (u,v) \mid$ there exist a substitution $\gamma$
and terms $w_1,\dots,w_n$ ($n \geq 0$) such that $u = (\ell\gamma) \cdot [w_1,\dots,w_n]$ and $v =
(r\gamma) \cdot [w_1,\dots,w_n] \}$.

For a given set of rules $\Rules$, the reduction relation $\arr{\Rules}$ is given by:
\begin{itemize}
\item if there is some $\rho \in \Rules$ and a pair $(u,v) \in \mathit{Pairs}_\rho$ such that
  $s|_p = u$, then $s \arr{\Rules} s[v]_p$.
\end{itemize}

\medskip
Note that this explicitly includes applications of rules at the head of a subterm.

\section{ATRSs and TRSs}

An \emph{applicative term rewriting system} is a pair $(\ATerms(\F,\V),\arr{\Rules})$.

A \emph{first-order term rewriting system} is a pair $(\FOTerms(\F,\V),\arr{\Rules})$ where:
\begin{itemize}
\item for all $(\afun : \atype) \in \F$: $\order(\atype) \leq 1$;
\item for all $\ell \arrz r \in \Rules$: $\ell,r \in \FOTerms(\F,\V)$;
\item for all $\ell \arrz r \in \Rules$: $\ell$ is not a variable, and $\FV(r) \subseteq \FV(\ell)$.
\end{itemize}

\end{document}

