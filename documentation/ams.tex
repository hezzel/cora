\documentclass{lmcs}
\pdfoutput=1

\usepackage{enumerate}
\usepackage[colorlinks=true]{hyperref}
\usepackage{amssymb}
\usepackage{xcolor,latexsym,amsmath,extarrows,alltt}
\usepackage{xspace}
\usepackage{booktabs}
\usepackage{mathtools}
\usepackage{enumitem}
\usepackage{stmaryrd}
\usepackage{microtype}

\theoremstyle{theorem}\newtheorem{theorem}{Theorem}
\theoremstyle{theorem}\newtheorem{lemma}[theorem]{Lemma}
\theoremstyle{theorem}\newtheorem{corollary}[theorem]{Corollary}
\theoremstyle{definition}\newtheorem{definition}[theorem]{Definition}
\theoremstyle{definition}\newtheorem{example}[theorem]{Example}

\newcommand{\N}{\mathbb{N}}
\newcommand{\F}{\mathcal{F}}
\newcommand{\V}{\mathcal{V}}
\newcommand{\Vmeta}{\mathcal{V}_{\mathit{meta}}}
\newcommand{\Vfree}{\mathcal{V}_{\mathit{nonb}}}
\newcommand{\Vbound}{\mathcal{V}_{\mathit{binder}}}
\newcommand{\Sorts}{\mathcal{S}}
\newcommand{\Types}{\mathcal{Y}}
\newcommand{\Terms}{\mathcal{T}}
\newcommand{\ATerms}{\mathcal{T}_{\mathcal{A}}}
\newcommand{\FOTerms}{\mathcal{T}_{\mathcal{FO}}}
\newcommand{\Rules}{\mathcal{R}}
\newcommand{\FV}{\mathit{FV}}
\newcommand{\Positions}{\mathit{Positions}}
\newcommand{\SubPositions}{\mathit{SubPositions}}
\newcommand{\HeadPositions}{\mathit{HeadPositions}}
\newcommand{\Pairs}{\mathit{Pairs}}

\newcommand{\domain}{\mathtt{dom}}
\newcommand{\order}{\mathit{order}}

\newcommand{\asort}{\iota}
\newcommand{\bsort}{\kappa}
\newcommand{\atype}{\sigma}
\newcommand{\btype}{\tau}
\newcommand{\ctype}{\pi}
\newcommand{\dtype}{\alpha}
\newcommand{\identifier}[1]{\mathtt{#1}}
\newcommand{\afun}{\identifier{f}}
\newcommand{\bfun}{\identifier{g}}
\newcommand{\cfun}{\identifier{h}}
\newcommand{\avar}{x}
\newcommand{\bvar}{y}
\newcommand{\cvar}{z}
\newcommand{\Avar}{X}
\newcommand{\Bvar}{Y}
\newcommand{\Cvar}{Z}
\newcommand{\AFvar}{F}
\newcommand{\BFvar}{G}
\newcommand{\CFvar}{H}

\newcommand{\clause}[1]{\textbf{#1}}

\newcommand{\abs}[2]{\lambda #1.#2}
\newcommand{\meta}[2]{#1\langle#2\rangle}

\newcommand{\arity}{\mathit{ar}}
\newcommand{\head}{\mathsf{head}}
\newcommand{\arrtype}{\Rightarrow}
\newcommand{\arrz}{\Rightarrow}
\newcommand{\arr}[1]{\arrz_{#1}}
\newcommand{\arrr}[1]{\arr{#1}^*}
\newcommand{\subtermeq}{\unlhd}
\newcommand{\headsubtermeq}{\unlhd_{\bullet}}
\newcommand{\supterm}{\rhd}
\newcommand{\suptermeq}{\unrhd}

\newcommand{\symb}[1]{\mathtt{#1}}

\newcommand{\nul}{\symb{0}}
\newcommand{\one}{\symb{1}}
\newcommand{\nil}{\symb{nil}}
\newcommand{\cons}{\symb{cons}}
\newcommand{\strue}{\symb{true}}
\newcommand{\sfalse}{\symb{false}}
\newcommand{\suc}{\symb{s}}
\newcommand{\map}{\symb{map}}
\newcommand{\bool}{\symb{bool}}
\newcommand{\nat}{\symb{nat}}
\newcommand{\lijst}{\symb{list}}
\newcommand{\unitsort}{\mathtt{o}}

\newcommand{\cora}{\textsf{CORA}\xspace}

\newcommand{\secshort}{\S}
\newcommand{\myparagraph}[1]{\paragraph{\textbf{#1}}}

\setlength{\parindent}{0pt}
\setlength{\parskip}{\bigskipamount}
\setlist[itemize]{topsep=-\bigskipamount}

\newcommand{\mysubsection}[1]{\vspace{-12pt}\subsubsection{#1}}

\begin{document}

\title{COnstrained Rewriting Analyser: formalism}
\author{Cynthia Kop}
\address{Department of Software Science, Radboud University Nijmegen}
\email{C.Kop@cs.ru.nl}

\maketitle

\begin{abstract}
\cora\ is a tool meant to analyse constrained term rewriting systems, both
first-order and higher-order.  This document explains the underlying formalism.
\end{abstract}

\section{Types}

We fix a set $\Sorts$ of \emph{sorts} and define the set $\Types$ of \emph{types} inductively:
\begin{itemize}
\item all elements of $\Sorts$ are types (also called \emph{base types});
\item if $\atype,\btype \in \Types$ then $\atype \arrtype \btype$ is also a type (called an arrow
  type).
\end{itemize}
The arrow operator $\arrtype$ is right-associative, so all types can be denoted in a form
$\atype_1 \arrtype \dots \arrtype \atype_m \arrtype \asort$ with $\asort \in \Sorts$; we say the
\emph{arity} of this type is $m$, and the \emph{output sort} is $\asort$.

The \emph{order} of a type is recursively defined as follows:
\begin{itemize}
\item for $\asort \in \Sorts$: $\order(\asort) = 0$;
\item for arrow types: $\order(\atype \arrtype \btype) = \max(\order(\atype) + 1,\order(\btype))$.
\end{itemize}

\bigskip
Type equality is literal equality (i.e., $\atype_1 \arrtype \btype_1$ is equal to $\atype_2 \arrtype
\btype_2$ iff $\atype_1 = \atype_2$ and $\btype_1 = \btype_2$).

\subsection*{Remarks}

We do not impose limitations on the set of sorts.  In traditional, unsorted term rewriting, there
is only one sort (e.g., $\Sorts = \{ \unitsort \}$). However, we may also have a larger finite or
even infinite sort set.
In the future, we may consider a shallow form of polymorphic types, but for the moment we will limit
interest to these simple types.

\section{Unconstrained Applicative Meta-variable Systems (AMSs)}

Let us start by explaining systems without constraints. Most of the notions will be directly
relevant to constrained systems as well.

\subsection{Terms}
Terms are \emph{well-typed} expressions built over given sets of \emph{function symbols} and
\emph{variables}. The full definition is presented below.

\mysubsection{Symbols and variables}

We fix a set $\F$ of \emph{function symbols}, also called the \emph{alphabet}; each function symbol
is a \emph{typed constant}. Notation: $\afun \in \F$ or $(\afun :: \atype) \in \F$ if we wish to
explicitly refer to the type (but the type should be considered implicit in the symbol).
Function symbols will generally be referred to as $\afun,\bfun,\cfun$ or using more suggestive
notation.

In addition, we fix a set $\Vbound$ of \emph{binder variables}, which are typed constants in the
same way.  $\Vbound$ should be disjoint from $\F$ and contain infinitely many binder variables of
each type.  We also assume given a set $\Vmeta$ of \emph{meta-variables}, which are constants with
a type and arity, i.e., $(\avar :: \atype,k) \in \Vmeta$; we require that $k$ is at most the arity
of $\atype$, and may also denote this as $(\avar :: [\atype_1 \times \dots \times \atype_k] \arrtype
\btype)$, which means a symbol named $\avar$ with type $\atype_1 \arrtype \dots \arrtype \atype_k
\arrtype \btype$ and arity $k$.
We define the subset $\Vfree := \{ \avar :: \atype \mid (\avar :: \atype,0) \in \Vmeta \}$ of
\emph{non-binder variables} to be the set of those meta-variables which have arity $0$, and
collectively refer to \emph{variables} for the set $\V := \Vbound \cup \Vfree$.
Binder variables and meta-variables will generally be referred to as $\avar,\bvar,\cvar,\Avar,\Bvar,
\Cvar,\AFvar,\BFvar,\CFvar$ or using more suggestive notation.

\mysubsection{Term formation}\label{subsec:form}

Terms are those expressions $s$ such that $s :: \atype$ can be derived for some $\atype \in \Types$
using the following clauses:

\begin{description}
\item[constant] if $(\afun :: \atype) \in \F$ then $\afun :: \atype$
\item[variable] if $(\avar :: \atype) \in \V$ then $\avar :: \atype$
\item[abstraction] if $(\avar :: \atype) \in \Vbound$ and $t :: \btype$ then $\abs{\avar}{t} ::
  \atype \arrtype \btype$
\item[meta-application] if $(\avar :: [\atype_1 \times \dots \times \atype_k] \arrtype \btype) \in
  \Vmeta$ with $k > 0$ and $t_1 :: \atype_1,\dots,t_k :: \atype_n$ then $\meta{\avar}{t_1,\dots,t_k}
  :: \btype$
\item[application] if $h :: \atype_1 \arrtype \dots \arrtype \atype_n \arrtype \btype$ with $n > 0$
  and $h$ is a constant, variable, abstraction or meta-application,
  and if $s_1 :: \atype_1,\dots,s_n :: \atype_n$,
  then $h(s_1,\dots,s_n) :: \btype$
\end{description}

If we identify $h() = h$, then any term can be written in a form $h(s_1,\dots,s_n)$ with the
\emph{head} $h$ either a constant, variable, abstraction or meta-application.  We will often use
this notation.
Similarly, we will sometimes refer to $\meta{\avar}{t_1,\dots,t_k}$ with $k \geq 0$, where
$\meta{\avar}{}$ is just $\avar \in \Vfree$.

A term of the form $\afun(s_1,\dots,s_n)$ is called a \emph{functional term} and $\afun$ is its
root. \\
A term of the form $\avar(s_1,\dots,s_n)$ is called a \emph{var term}, and $\avar$ is its
variable. \\
The variable of an abstraction $\abs{\avar}{t}$ or a meta-application $\meta{x}{t_1,\dots,t_k}$ is
$\avar$. \\
An application of the form $(\abs{\avar}{t})(s_0,\dots,s_n)$ is called a \emph{$\beta$-redex} and
$\avar$ is its variable. \\
A term of the form $\meta{x}{t_1,\dots,t_k}(s_{k+1},\dots,s_n)$ with $n > k > 0$ is a
meta-application-application, and $\avar$ is its variable.  However, while we permit the formation
of meta-application-applications to avoid requiring many exceptions in for instance termination
techniques, in practice we will not consider terms of this form.

If $s :: \atype$ then we say that $\atype$ is the type of $s$; it is clear from the definitions
above that each term has a unique type.

Note that in the \clause{application} clause, $n$ is not required to be maximal; for example, if
$\symb{greater} :: \mathtt{int} \arrtype \mathtt{int} \arrtype \mathtt{bool}$, then each of
$\symb{greater}(),\symb{greater}(\avar)$ and $\symb{greater}(\avar,\bvar)$ are terms (with distinct
types); each having $\symb{greater}$ as its head.
Note also that a variable $\avar$ is also considered a var term, and a constant $\afun$ is a
functional term, but a plain abstraction is \emph{not} a $\beta$-redex.
A term $\avar(s)$ is a varterm, not a meta-application, even if $\avar \in \Vmeta$.

\mysubsection{$\alpha$-equality}
We let $=_\alpha$ be the usual $\alpha$-renaming equivalence relation as used in the
$\lambda$-calculus. This relation can be formally defined as follows:
\begin{itemize}
\item Let $\mu_0,\nu_0 : \Vbound \rightarrow \N$ be defined as follows:
  $\mu_0(\avar) = \nu_0(\avar) = 0$ for all $\avar \in \Vbound$.
\item Let $s =_\alpha t$ iff $s =_\alpha^{\mu_0,\nu_0,1} t$.
\item For $\mu,\nu : \Vbound \rightarrow \N$ and $k \in \N$, let
  $s =_\alpha^{\mu,\nu,k} t$ if and only if this can be defined by the following clauses:
  \begin{itemize}
  \item $a(s_1,\dots,s_n) =_\alpha^{\mu,\nu,k} b(t_1,\dots,t_n)$ with $n > 0$ if
    $a =_\alpha^{\mu,\nu,k} b$ and $s_i =_\alpha^{\mu,\nu,k} t_i$ for all $i \in \{1,\dots,n\}$;
  \item $\afun =_\alpha^{\mu,\nu,k} \afun$ for all $\afun \in \F$ and
        $\avar =_\alpha^{\mu,\nu,k} \avar$ for all $\avar \in \Vfree$;
  \item $\avar =_\alpha^{\mu,\nu,k} \bvar$ for $\avar,\bvar \in \Vbound$ if either
    $\avar = \bvar$ and $\mu(\avar) = \nu(\bvar) = 0$, or $\mu(\avar) = \mu(\bvar) > 0$;
  \item $\abs{\avar}{s} =_\alpha^{\mu,\nu,k} \abs{\bvar}{t}$ iff $s =_\alpha^{\mu[\avar:=k],
    \mu[\bvar:=k],k+1} t$; \\
    (Here, $\mu[\avar:=k]$ is the function that maps $\avar$ to $k$ and all other $\cvar$ to
    $\mu(\cvar)$; similar for $\nu[\bvar:=k]$.)
  \item $\meta{\avar}{s_1,\dots,s_k} =_\alpha^{\mu,\nu,k} \meta{\avar}{t_1,\dots,t_k}$ with $k
    > 0$ if $s_i =_\alpha^{\mu,\nu,k+1} t_i$ for all $i \in \{1,\dots,k\}$.
  \end{itemize}
\end{itemize}
That is, we progressively descend into the term and keep track of where variables are bound; the
structure of the two terms has to be exactly the same, and function symbols and unbound variable
should occur at the same positions in both terms. However, when encountering a bound variable, we
only require that this variable was bound by the same $\lambda$ in both terms.
We can straightforwardly prove that $=_\alpha$ is an equivalence relation (Corollary
\ref{corr:alphaequiv}).

\mysubsection{Restricted terms}\label{subsec:termsets}

The definition of terms is deliberately broad, to support a rather liberal kind of higher-order
rewriting; in some other works, what we call ``terms'' would be referred to as \emph{metaterms},
with the word \emph{terms} referring to what we will call ``true terms''.
We use this terminology simply because in an analysis tool, the vast majority of reasoning
happens on metaterms, so these are the primary objects we wish to consider.

However, in practice we do often consider limitations: specific kinds of terms which are built
using subsets and restrictions of the clauses in Section \ref{subsec:form}.  We consider the most
important ones:

\begin{itemize}
\item A \emph{true term} is a term without meta-applications; that is, a term whose type can be
  derived using only the clauses \clause{constant}, \clause{variable}, \clause{abstraction} and
  \clause{application}.
  (Note that variables in $\Vfree$ are still allowed!)
\item A term $s$ is a \emph{pattern} if abstractions, meta-applications and single meta-variables
  may not occur at the head of an application, and the arguments to a meta-application must all be
  distinct bound variables.  That is, a pattern is a term whose type can be derived using the
  clauses:
  \begin{description}
  \item[func] if $(\afun :: \atype_1\!\arrtype\!\dots\!\arrtype\!\atype_n \arrtype \btype) \in \F$
    and $s_1 :: \atype_1,\dots,s_n :: \atype_n$ then $\afun(s_1,\dots,s_n) :: \btype$
  \item[bvarterm] if $(\avar :: \atype_1 \arrtype \dots \arrtype \atype_n \arrtype \btype) \in
    \Vbound$ and $s_1 :: \atype_1,\dots,s_n :: \atype_n$ then $\avar(s_1,\dots,s_n) :: \btype$
  \item[abstraction] if $(\avar :: \atype) \in \Vbound$ and $t :: \btype$ then $\abs{\avar}{t} ::
    \atype \arrtype \btype$
  \item[pmeta] if $(\avar :: [\atype_1 \times \dots \times \atype_k] \arrtype \btype) \in \Vmeta$
    for $k \geq 0$ and $(\bvar_1 :: \atype_1),\dots,(\bvar_k :: \atype_k) \in \Vbound$ are all
    distinct, then $\meta{\avar}{\bvar_1,\dots,\bvar_k} :: \btype$
  \end{description}
  (Note that \clause{func} is a combination of \clause{constant} and \clause{application},
  \clause{bvarterm} combines a restriction of \clause{variable} and \clause{application},
  \clause{abstraction} is the same as before, and \clause{pmeta} combines a restriction of
  \clause{variable} (if we identify $\meta{x}{}$ with $x$) with a restriction of \clause{meta}.
  So, any expression typed using these expressions is indeed a term.)

\item Hence, a true term is a pattern if it can be typed using only \clause{func},
  \clause{bvarterm}, \clause{abstraction} and:
  \begin{description}
  \item[fvar] if $(\avar :: \atype) \in \Vfree$ then $\avar :: \atype$
  \end{description}

\item A term $s$ is \emph{applicative} if it does not use meta-applications or variables in
  $\Vbound$.  Note that this limitation excludes abstractions.  Hence, a term is applicative if it
  can be typed using just the clauses \clause{constant}, \clause{fvar} and \clause{application}.
  Equivalently, an applicative term can be typed using just \textbf{func} and:
  \begin{description}
  \item[fvarterm] if $(\avar :: \atype_1 \arrtype \dots \arrtype \atype_n \arrtype \btype) \in
    \Vfree$ and $s_1 :: \atype_1,\dots,s_n :: \atype_n$ then $\avar(s_1,\dots,s_n) :: \btype$
  \end{description}
\item Hence, an \emph{applicative pattern} can be typed using just the clauses \clause{func} and
  \clause{fvar}.
\item A term is \emph{first-order} if it is an applicative pattern with only base-type subterms;
  that is, a term whose type can be derived using only:
  \begin{description}
  \item[(fofunc)] if $(\afun :: \atype_1 \arrtype \dots \arrtype \atype_n \arrtype \asort) \in \F$
    with $\asort \in \Sorts$ and $s_1 :: \atype_1,\dots,s_n :: \atype_n$ then
    $\afun(s_1,\dots,s_n) :: \asort$
  \item[(fovar)] if $(\avar :: \asort) \in \Vfree$ with $\asort \in \Sorts$ then $\avar :: \asort$.
  \end{description}
\end{itemize}

We will consider term equality modulo $=_\alpha$.  Although in practice we will typically reason
with terms directly rather than with equivalence classes, definitions should take $\alpha$-equality
into account.

We denote $\Terms(\F,\V)$ for the set of all true terms $s$, modulo $=_\alpha$.

The set of applicative terms is denoted $\ATerms(\F,\V)$.  Note that $=_\alpha$ is the
identity on applicative terms, and that every applicative term is a true term, so
$\ATerms(\F,\V) \subseteq \Terms(\F,\V)$.

The set of first-order terms is denoted $\FOTerms(\F,\V)$.  Since every first-order term is
also an applicative pattern, $\FOTerms(\F,\V) \subseteq \ATerms(\F,\V) \subseteq \Terms(\F,\V)$.

\subsection{Free variables}
The set of \emph{free variables} of a term is inductively defined as follows:
\begin{itemize}
\item $\FV(h(s_1,\dots,s_n)) = \FV(h) \cup \FV(s_1) \cup \dots \cup \FV(s_n)$ if $n > 0$;
\item $\FV(\afun) = \emptyset$;
\item $\FV(\avar) = \{ \avar \}$;
\item $\FV(\abs{\avar}{t}) = \FV(t) \setminus \{ \avar \}$;
\item $\FV(\meta{\avar}{t_1,\dots,t_k}) = \{ \avar \} \cup \FV(t_1) \cup \dots \cup \FV(t_k)$.
\end{itemize}
That is, $\FV(s)$ contains all binder variables and meta-variables in $s$ except for those bound
by a $\lambda$.
For applicative and first-order terms $s$, this is the set of \emph{all} variables occurring in
$s$.

A term $s$ is \emph{closed} if $\FV(s) \subseteq \Vmeta$.
A term $s$ is \emph{ground} if $\FV(s) = \emptyset$.

Since $\FV(s) = \FV(t)$ whenever $s =_\alpha t$ (Lemma \ref{corr:alphafreevar}),
$\FV$ also defines a function on equivalence classes of terms.

\subsection{Application}\label{subsec:application}
We have defined application as a clause, but it will sometimes be convenient to have it as an
operation.  This is defined as follows:

\begin{definition}
For any term $h(s_1,\dots,s_n) :: \atype \arrtype \btype$ with $n \geq 0$, and $t :: \atype$, we
let $h(s_1,\dots,s_n) \cdot t$ be the term $h(s_1,\dots,s_n,t) :: \btype$.
The application operator $\cdot$ is left-associative, so $h(s_1,\dots,s_n) \cdot t_1 \cdots t_m$
denotes $((h(s_1,\dots,s_n) \cdot t_1) \cdot t_2) \cdots t_m = h(s_1,\dots,s_n,t_1,\dots,t_m)$.
\end{definition}

We easily obtain the following result;

\begin{lemma}\label{lem:applicative_notation}
The set $\ATerms(\F,\V)$ is the smallest set such that:
\begin{itemize}
\item $\F \cup \V \subseteq \ATerms(\F,\V)$;
\item if $s,t \in \ATerms(\F,\V)$ and $s : \atype \arrtype \btype$ and $t : \atype$ then
  $s \cdot t \in \ATerms(\F,\V)$.
\end{itemize}
\end{lemma}

\begin{proof}
Trivial.
\end{proof}

Lemma~\ref{lem:applicative_notation} shows that our applicative terms are the same as applicative
terms constructed in the traditional way; however, for convenience we denote them in a functional
notation.
We can similarly see easily that the set of all terms is exactly the smallest set which includes
$\F,\V$ and is closed under abstraction, meta-application and this application operator.

Application interacts with $\alpha$-equality as you would expect: $u \cdot s_1 \cdots s_n
=_\alpha v \cdot t_1 \cdots t_n$ if and only if $u =_\alpha v$ and each $s_i =_\alpha t_i$ (Lemma
\ref{lem:alphaappl}).

\subsection{Substitution}

A \emph{substitution} is a partial function $\gamma$ from $\Vbound \cup \Vmeta$ to the set of all
terms, with the following properties:
\begin{itemize}
\item if $(\avar :: \atype) \in \domain(\gamma) \cap \Vbound$ then $\gamma(\avar) :: \atype$
\item if $(\avar :: [\atype_1 \times \dots \times \atype_n] \arrtype \btype \in \domain(\gamma)
  \cap \Vmeta$ then we can write $\gamma(\avar) = \abs{\bvar_1 \dots \bvar_n}{t}$ with
  each $\bvar_i :: \atype_i$ and $t :: \btype$
\end{itemize}
Here, $\domain(\gamma) \subseteq \Vbound \cup \Vmeta$, is the domain of the partial function
$\gamma$.  For $\avar \in (\Vbound \cup \Vmeta) \setminus \domain(\gamma)$ we will abuse notation
and also write $\gamma(\avar) = \avar$.
We denote $[\avar_1:=s_1,\dots,\avar_n:=s_n]$ for the substitution $\gamma$ on domain $\{\avar_1,
\dots,\avar_n\}$ with $\gamma(x_i) = s_i$ for $1 \leq i \leq n$.
Applying a substitution $\gamma$ to a term $s$, notation $s\gamma$, yields a new term of the same
type, by the clauses:

\begin{itemize}
\item $h(s_1,\dots,s_n)\gamma = (h\gamma) \cdot (s_1\gamma) \cdots (s_n\gamma)$ if $n > 0$;
\item $\afun\gamma = \afun$;
\item $\avar\gamma = \gamma(\avar)$;
\item $(\abs{\avar}{t})\gamma = \abs{\cvar}{(t ([\avar:=\cvar] \cup [\bvar:=\gamma(\bvar) \mid
  \bvar \in \domain(\gamma) \setminus \{\avar\}]))}$ \\
  for $\cvar$ a \emph{fresh}** variable in $\Vbound$ with the same type as $\avar$;
\item $\meta{\avar}{s_1,\dots,s_k}\gamma = \meta{\avar}{s_1\gamma,\dots,s_k\gamma}$ if
  $\avar \notin \domain(\gamma)$
\item $\meta{\avar}{s_1,\dots,s_k}\gamma = t[\bvar_1:=s_1\gamma,\dots,\bvar_k:=s_k\gamma]$
  if $\avar \in \domain(\gamma)$ and $\gamma(\avar) = \abs{\bvar_1 \dots \bvar_k}{t}$ \\
  (Here, if some $x_i = x_j$ for $i < j$, the substitution $[x_1:=u_1,\dots,x_n:=u_n]$ maps
  $x_i$ to $u_j$.)
\end{itemize}
** A \emph{fresh} variable $\cvar$ is one that does not occur in $\FV(\gamma(\bvar))$ for any
$\bvar \in \FV(s)$.

It is not immediately obvious that this definitions is well-founded: it is not the case that each
step is defined in terms of an obviously smaller substitution / application.  However, we can see
this in two steps: substitutions are well-defined when $\domain(\gamma) \subseteq \Vbound$; and
because of this, we can handle the meta-application case of the proof that all substitutions are
well-defined (Lemma \ref{lem:substdefined}).

More critically, this definition technically does not define a function on terms: the substitution
of an abstraction may lead to any fresh variable being chosen.  Hence, we should perhaps think of
it as defining a \emph{relation} between terms.  However, substitution does define a function on
\emph{equivalence classes} (by Corollary \ref{cor:substitutionalpha}):
  if $s_1 =_\alpha s_2$ and $s_1\gamma = t_1$ and $s_2\gamma' = t_2$ and $\gamma(x) =_\alpha
  \gamma'(x)$ for all $x \in \FV(s_1)$, then $t_1 =_\alpha t_2$.
As a special case we see that, if $s_1 =_\alpha s_2$ and $s_1\gamma = t_1$ and $s_2\gamma = t_2$,
  then $t_1 =_\alpha t_2$.
Hence, the difference is not significant, and we can safely think of substitution as defining a
function.

For two substitutions $\gamma$ and $\delta$, we let $\gamma\delta$ denote the substitution
$[\avar := \gamma(\avar)\delta \mid \avar \in \domain(\gamma)] \cup
[\avar := \delta(\avar) \mid \avar \in \domain(\delta) \setminus \domain(\gamma)]$.
Essentially, applying a substitution $\gamma$ to a term corresponds with replacing each variable
$\avar$ by $\avar\gamma$ (and evaluating the applications of abstractions that are created as a
result), and applying $\gamma\delta$ corresponds to replacing $\avar$ by $(\avar\gamma)\delta$.
We can prove (Lemma \ref{lem:combinesubst}) that $s(\gamma\delta)$ is exactly $(s\gamma)\delta$.

A \emph{renaming} is a substitution $[x_1:=y_1,\dots,x_n:=y_n]$ with $x_1,\dots,x_n$ pairwise
distinct, and $y_1,\dots,y_n$ pairwise distinct.

We have the following results, which will often be used quietly in proofs:
\begin{itemize}
\item $(s \cdot t_1 \cdots t_n)\gamma = (s\gamma) \cdot (t_1\gamma) \cdots (t_n\gamma)$
  (Lemma \ref{lem:appsubstitute})
\item For all terms $s$, $s[] = s$, where $[]$ is the empty substitution
  (Lemma \ref{lem:substitutionrefl})
\item For all terms $s$ and substitutions $\gamma,\delta$ such that $\gamma(\avar) = \delta(\avar)$
  for all $\avar \in \FV(s)$: $s\gamma = s\delta$ (Lemma \ref{lem:substextend})
\item For all abstractions $\abs{\avar}{s}$ and $\abs{\bvar}{t}$:
  $\abs{\avar}{s} =_\alpha \abs{\bvar}{t}$ if and only if $\bvar \notin \FV(\abs{\avar}{s})$ and
  $s[\avar:=\bvar] = t$ (Lemma \ref{lem:alternativealpha})
\end{itemize}

\subsection{Positions}

The \emph{head positions} of a given term are the paths to specific subterms, defined as follows:

\begin{itemize}
\item $\HeadPositions(h(s_1,\dots,s_n)) = \{ \epsilon, \star 0, \dots, \star (n-1) \} \cup
  \{ i \cdot p \mid \exists i \in \{1,\dots,n\}. p \in \HeadPositions(s_i) \} \cup
  \SubPositions(h)$
\item $\SubPositions(\afun) = \SubPositions(\avar) = \emptyset$
\item $\SubPositions(\abs{\avar}{t}) = \{ 0 \cdot p \mid p \in \HeadPositions(t) \}$
\item $\SubPositions(\meta{\avar}{t_1,\dots,t_k}) = \{ !i \cdot p \mid p \in
  \HeadPositions(t) \}$
\end{itemize}

The \emph{positions} of a given term are all the head positions of a form $p_1 \cdots p_n \epsilon$,
so those that do not contain $\star$.
Note that (head) positions are associated to a term; thus, not every sequence of natural numbers
and exclamation marks is a position.

For a term $s$ and a position $p \in \HeadPositions(s)$, the \emph{subterm of $s$ at position $p$},
denoted $s|_p$, is defined as follows:
\begin{itemize}
\item $s|_\epsilon = s$;
\item $h(s_1,\dots,s_n)|_{\star i} = h(s_1,\dots,s_i)$ (so just $h$ if $i = 0$);
\item $h(s_1,\dots,s_n)|_{i \cdot p} = s_i|_p$ if $1 \leq i \leq n$;
\item $(\abs{\avar}{t})(s_1,\dots,s_n)|_{0 \cdot p} = t|_p$;
\item $\meta{\avar}{t_1,\dots,t_k}(s_1,\dots,s_n)|_{!i \cdot p} = t_i|_p$.
\end{itemize}

If $s|_p$ has the same type as some term $u$, then $s[u]_p$ denotes $s$ with the subterm at position
$p$ replaced by $u$.  Formally, $s[u]_p$ is obtained as follows:
\begin{itemize}
\item $s[u]_\epsilon = u$;
\item $h(s_1,\dots,s_n)[u]_{\star i} = u \cdot s_{i+1} \cdots s_n$;
\item $h(s_1,\dots,s_n)[u]_{i \cdot p} = h(s_1,\dots,s_{i-1},s_i[u]_p,s_{i+1},\dots,s_n)$;
\item $(\abs{\avar}{t})(s_1,\dots,s_n)[u]_{0 \cdot p} = (\abs{\avar}{t[u]_p})(s_1,\dots,s_n)$;
\item $\meta{x}{t_1,\dots,t_k}(s_1,\dots,s_n)[u]_{!i \cdot p} = \meta{x}{t_1,\dots,t_{i-1},t_i[u]_p,t_{i+1},\dots,t_k}(s_1,\dots,s_n)$;
\end{itemize}
Thus, we can find and replace the subterm at a given position.
Note that this is \emph{not} an operation on equivalence classes of terms; for example,
$\abs{x}{x} =_\alpha \abs{y}{y}$, but $(\abs{x}{x})[x]_0 = \abs{x}{x} \not =_\alpha
\abs{y}{x} = (\abs{y}{y})[y]_0$.  Hence, when subterms are considered this should usually be
accompanied by some result that guarantees the operation handles bound variables properly.

Note that we do \emph{not} have the property that if $p$ is a position of $s$, then $p$ is a
position of $s\gamma$ for any substitution $\gamma$; for example, $x(\afun(y))$ has a position
$1\ 1\ \epsilon$, with $x(\afun(y))|_{1\ 1\ \epsilon} = y$.  However,
$x(\afun(y))[x:=\bfun(\identifier{a})] = \bfun(\identifier{a},\afun(y))$, which does not have this
position.  If we ever need it, we could define alternative positions with the property that
if $s|_p = t$ then $(s\gamma)|_p = t\gamma$ for $s$ a proper term.  However, we cannot have both
this, and that positions correspond to the usual definition of positions in the first-order setting.

\subsection{Subterms}

We say that \emph{$t$ is a subterm of $s$}, notation $t \subtermeq s$, if there is some position
$p \in \Positions(s)$ with $t = s|_p$.  This could equivalently be formulated as follows:

\begin{lemma}
$u \subtermeq s$ if and only if one of the following holds:
\begin{itemize}
\item $s = u$;
\item $s = h(s_1,\dots,s_n)$ and $u \subtermeq s_i$ for some $i$;
\item $s = (\abs{x}{t})(s_1,\dots,s_n)$ and $u \subtermeq t$;
\item $s = \meta{x}{t_1,\dots,t_k}(s_1,\dots,s_n)$ and $u \subtermeq t_i$ for some $i$.
\end{itemize}
\end{lemma}

We also observe that $\subtermeq$ is transitive:

\begin{lemma}
If $s \subtermeq t$ and $t \subtermeq q$ then $s \subtermeq q$.
\end{lemma}

This is obvious because if $t = q|_p$ and $s = t|_{p'}$ then $s = q|_{p \cdot p'}$.

It should be noted that in contrast to most definitions of higher-order rewriting, we do \emph{not}
consider, for example, $\afun(x)$ to be a subterm of $\afun(x,y)$.  Instead, we define the
following: \emph{$t$ is a head-subterm of $s$}, notation $t \headsubtermeq s$ if $s|_p = t$ for
$p \in \HeadPositions(s)$.

It should also be noted that if $s =_\alpha t$, it does not follow that $s$ and $t$ have the same
subterms: $\abs{x}{x}$ has a subterm $x$, while $\abs{y}{y}$ does not.  For applicative and
first-order terms, this is not an issue.
In addition, subterms are not preserved under substitution: $\symb{a} \subtermeq \meta{\avar}{\symb{a}}$
but not $\symb{a}[\avar:=\abs{\bvar}{\symb{b}}] \subtermeq \meta{\avar}{\symb{a}}[\avar:=\abs{\bvar}{\symb{b}}]$.
Subterms of true terms \emph{are} preserved under substitution; i.e., if $s \subtermeq t
\in \Terms(\F,\V)$ then $s\gamma \subtermeq t\gamma$.

Regarding different kinds of terms: the subterms and positions of a first-order term by these
definitions are exactly the subterms and positions as they are usually considered in first-order
term rewriting; however, head-subterms are generally not considered.  For applicative terms,
both subterms and head-subterms are usually referred to as just ``subterms''; we distinguish them
here because doing so is practical for analysis.

\subsection{Rules and rewriting}

A rule $\rho$ is a pair $\ell \arrz r$ of two closed terms with the same type, such that
$\FV(r) \subseteq \FV(\ell)$.

A rule is a \emph{pattern rule} if the left-hand side $\ell$ is a pattern.

For terms $s,t$ and a set of rules $\Rules$, we say $s \arr{\Rules} t$ if there exist a head
position $p$ of $s$, rule $\ell \arrz r \in \Rules$ and substitution $\gamma$ such that $s|_p =
\ell\gamma$ and $t = s[r\gamma]_p$.

Although we use positions -- which is dangerous in the presence of binders -- we can see without
much effort that $\arr{\Rules}$ really does define a reduction on the set of terms: if
$s =_\alpha s'$ and $s \arr{\Rules} t$, then $s' \arr{\Rules} t'$ with $t =_\alpha t'$ (Lemma
\ref{lem:reductionwelldefined}).  This holds because $\FV(r) \subseteq \FV(\ell)$ for all rules.

\subsection{Different kinds of term rewriting systems.}

Using these ingredients, we can define a term rewriting system in a variety of ways.  Formally,
an abstract rewriting system is a pair $(\mathcal{A},\arrz)$ where $\mathcal{A}$ is a set and
$\arrz$ a binary relation on that set.  Properties such as termination and confluence can be
expressed in terms of abstract rewriting systems.  We consider a number of such kinds of ARSs.
In all cases, we often omit the set of terms (assuming it understood from context) and only
refer to it by the ARS kind (e.g., AMS or MTRS) and the set of $\Rules$ that generates the
rewrite relation.

An \emph{applicative meta-variable system} (AMS) is an ARS of the form
$(\Terms(\F,\V),\arr{\Rules})$.  That is, we rewrite true terms, and impose no restrictions on
reduction.

A \emph{well-behaved meta-variable system} (WAMS) is an AMS with rules $\Rules$ where all
$\ell \arrz r \in \Rules$ are pattern rules of the form $\afun\ \ell_1 \cdots \ell_k \arrz r$.

A \emph{curried functional system} (CFS) is an AMS with rules $\Rules$ where both sides of all
$\ell \arrz r \in \Rules$ are true terms.

An \emph{algebraic functional system} (AFS) is an ARS of the form $(T^\arity,\arr{\Rules})$ where
$T^\arity \subseteq \Terms(\F,\V)$ and:
\begin{itemize}
\item $\arity$ is a function from $\F$ to $\N$ with $\arity(\afun) \leq \mathit{arity}(\atype)$
  for every $(\afun :: \atype) \in \F$
\item $\F$ contains, for all types $\atype,\btype$, a symbol $@_{\atype,\btype} :: (\atype
  \arrtype \btype) \arrtype \atype \arrtype \btype$ with $\arity(@_{\atype,\btype}) = 2$
\item $\Rules \supseteq B$ where $B = \{ @_{\atype,\btype}(\abs{\avar}{\meta{F}{\avar}},Y) \to
  \meta{F}{Y} \mid \atype,\btype \in \Types \}$
\item all terms in $T^\arity$ are \emph{algebraic}: they are constructed using only the clauses
  \clause{variable}, \clause{abstraction} and:
  \begin{description}
  \item[ar-func] if $(\afun :: \atype_1 \arrtype \dots \arrtype \atype_k \arrtype \btype) \in \F$
    with $k = \arity(\afun)$ and $s_1 :: \atype_1,\dots,s_k :: \atype_k$ then $\afun(s_1,\dots,
    s_n) :: \btype$
  \end{description}
\item both sides of all $\ell \arrz r \in \Rules \setminus B$ are algebraic terms as well.
\end{itemize}
Hence, an AFS can be viewed as a system with mostly functional notation, but which does admit an
explicit application implemented as a function symbol in our system.

An \emph{applicative term rewriting system} (ATRS) is an ARS of the form $(\ATerms(\F,\V),
\arr{\Rules})$ where both sides of all rules are applicative terms as well.  If the left-hand sides
of all rules are applicative patterns, we condier it a \emph{pattern ATRS}.

A \emph{many-sorted term rewriting system} (MTRS) is an ARS of the form $(\FOTerms(\F,\V),
\arr{\Rules})$ with the following properties:
\begin{itemize}
\item for all $(\afun : \atype) \in \F$: $\order(\atype) \leq 1$;
\item for all $\ell \arrz r \in \Rules$: $\ell$ is not a variable and both $\ell$ and $r$ are in
  $\FOTerms(\F,\V)$.
\end{itemize}
Note that in a many-sorted term rewriting system, rules have base type (since $\ell$ and $r$ in
$\FOTerms(\F,\V)$), so they cannot be applied at the head of a term.

An \emph{unsorted first-order term rewriting system} (TRS) is a many-sorted term rewriting system
with $\Sorts = \{ \unitsort \}$.

\newpage\appendix

\newcommand{\hideproof}[1]{[Details hidden]}
\newcommand{\showproof}[1]{\ \\\textbf{(details:)}\\#1}

\section{Correctness of the unconstrained formalism}

\subsection{$\alpha$-equality}

We first see that $=_\alpha$ is an equivalence relation (on terms, so also on true terms), so that
we can reason modulo it!

\begin{lemma}\label{lem:alphaequiv}
For all terms $s,t,q$, functions $\mu,\xi,\chi : \V \to \N$ and $k \in \N$:
\begin{enumerate}
\item\label{lem:alphaequiv:reflexive}
  $s =_\alpha^{\mu,\mu,k} s$;
\item\label{lem:alphaequiv:symmetric}
  if $s =_\alpha^{\mu,\xi,k} t$ then $t =_\alpha^{\xi,\mu,k} s$;
\item\label{lem:alphaequiv:transitive}
  if $s =_\alpha^{\mu,\xi,k} t$ and $t =_\alpha^{\xi,\chi,k} q$ then $s =_\alpha^{\mu,\chi,k} q$.
\end{enumerate}
\end{lemma}

\begin{proof}
All follow by a straightforward induction on the size of $s$.
\hideproof{
    (\ref{lem:alphaequiv:reflexive})
    If $s = h(s_1,\dots,s_n)$ with $n > 0$ then by IH both $h =_\alpha^{\mu,\mu,k} h$ and $s_i
    =_\alpha^{\mu,\mu,k} s_i$ for all $i$, which immediately gives $s =_\alpha^{\mu,\mu,k} s$.
    We similarly complete by the IH if $s = \meta{\avar}{t_1,\dots,t_k}$.
    If $s = \afun \in \F$ or $\avar \in \Vfree$ the result is immediate.
    If $s = \avar \in \Vbound$, then whether $\mu(\avar) = 0$ or $\mu(\avar) = \mu(\avar) > 0$,
    $\avar =_\alpha^{\mu,\nu,k} \avar$ follows.
    If $s = \abs{x}{t}$ then by IH $t =_\alpha^{\mu[x:=k],\mu[x:=k],k+1} t$; here also
    $s =_\alpha^{\mu,\mu,k} s$ follows directly.
    
    (\ref{lem:alphaequiv:symmetric})
    If $s = a(s_1,\dots,s_n)$ with $n > 0$ then $t = b(t_1,\dots,t_n)$ with $a =_\alpha^{\mu,\xi,k} b$
    and $s_i =_\alpha^{\mu,\xi,k} t_i$ for all $i$; by IH then $b =_\alpha^{\xi,\mu,k} a$ and
    $t_i =_\alpha^{\xi,\mu,k} s_i$ for all $i$, so $t =_\alpha^{\xi,\mu,k} s$.
    We similarly complete by IH if $s = \meta{\avar}{t_1,\dots,t_k}$ and
    $t = \meta{\avar}{t_1,\dots,t_k}$.
    If $s \in \F \cup \Vfree$ then $t = s$ and clearly also $t =_\alpha^{\xi,\mu,k} s$.
    If $s = \avar \in \Vbound$ then $t = \bvar \in \Vbound$ and $\mu(\avar) = \xi(\bvar)$; either
    $\mu(\avar) = 0$ and $\bvar = \avar$ or $\mu(\avar) > 0$, but in both cases
    $t =_\alpha^{\xi,\mu,k}$ holds.
    If $s = \abs{x}{s'}$ then $t = \abs{y}{t'}$ and $s' =_\alpha^{\mu[x:=k],\xi[y:=k],k+1} t'$. By IH
    also $t' =_\alpha^{\xi[y:=k],[x:=k],k+1} s'$.
    
    (\ref{lem:alphaequiv:transitive})
    If $s = a(s_1,\dots,s_n)$ with $n > 0$ then $t = b(t_1,\dots,t_n)$ and $q = c(q_1,\dots,q_n)$ with
    $a =_\alpha^{\mu,\xi,k} b =_\alpha^{\xi,\chi,k} c$ and $s_i =_\alpha^{\mu,\xi,k} t_i =_\alpha^{\xi,
    \chi,k} q_i$ for all $i$.  By IH therefore $a =_\alpha^{\mu,\chi,k} c$ and $s_i =_\alpha^{\mu,\chi,
    k} t_i$ for all $i$, allowing for the conclusion $s =_\alpha^{\mu,\chi,k} q$.
    The case $s = \meta{\avar}{t_1,\dots,t_k}$ follows from IH in the same way.
    If $s \in \F \cup \Vfree$ then $t = s$ and $q = s$, and $s =_\alpha^{\mu,\chi,k} q$ follows
    directly.
    If $s = \avar \in \Vbound$ and $\mu(\avar) = 0$ then $t = \avar$ and $\xi(\avar) = 0$, and therefore
    $q = \avar$ and $\chi(\avar) = 0$; hence also $s =_\alpha^{\mu,\chi,k} q$.
    If $s = \avar \in \Vbound$ and $\mu(\avar) > 0$ then $t = \bvar \in \Vbound$ and $\xi(\bvar) =
    \mu(\avar) > 0$, and therefore $q = \cvar \in \Vbound$ and $\chi(\cvar) = \xi(\bvar) = \mu(\avar) >
    0$.
    Finally, if $s = \abs{\avar}{s'}$ then $t = \abs{\bvar}{t'}$ and $q = \abs{\cvar}{q'}$, with
    $s' =_\alpha{\mu[x:=k],\xi[x:=k],k+1} t' =_\alpha{\xi[x:=k],\chi[x:=k],k+1} q'$.
    By IH then $s' =_\alpha^{\mu[x:=k],\chi[x:=k],k+1} q'$.
}
\end{proof}

Choosing $k = 1$ and $\mu,\xi,\chi$ the function mapping everything to $0$, we obtain:

\begin{corollary}\label{corr:alphaequiv}
$=_\alpha$ is an equivalence relation.
\end{corollary}

As noted in the text, $\FV$ defines a function on equivalence classes:

\begin{lemma}\label{lem:alphafreevar}
If $s =_\alpha^{\mu,\xi,k} t$ then $\FV(s) \setminus \{ x \mid \mu(x) \neq 0 \} = \FV(t) \setminus \{ x \mid \xi(x) \neq 0 \}$.
\end{lemma}

\begin{proof}
By induction on the size of $s$.
All cases are straightforward.
\hideproof{
    For brevity, we denote $M := \{ x \mid \mu(x) \neq 0 \}$ and $X :=  \{ x \mid \xi(x) \neq 0 \}$.
    \begin{itemize}
    \item If $s = a(s_1,\dots,s_n)$ then $t = b(t_1,\dots,t_n)$ and both $a =_\alpha^{\mu,\xi,k} b$ and
      each $s_i =_\alpha^{\mu,\xi,k} t_i$.
      By the induction hypothesis, $\FV(a) \setminus M = \FV(b) \setminus X$ and
      $\FV(s_i) \setminus M = \FV(t_i) \setminus X$.
      Since $\FV(s) \setminus M = (\FV(a) \setminus M) \cup (\FV(s_1) \setminus M) \cup \dots \cup (\FV(s_n) \setminus M)$ for any set $M$,
      and similar for $\FV(t) \setminus X$, we are done.
    \item If $s = \meta{\avar}{s_1,\dots,s_k}$ and $t = \meta{\avar}{t_1,\dots,t_k}$ we similarly complete
      with the induction hypothesis.
    \item If $s = \afun$ then $t = \afun$ and both sets are empty.
    \item If $s = \avar \in \Vfree$ then $t = \avar$ and both sets are $\{\avar\}$.
    \item If $s = \avar \in \Vbound$ and $\mu(\avar) = 0$ then $t = \avar$ and $\xi(\avar) = 0$ as well.
      Therefore $\avar \notin M$ and $\avar \notin X$.
      Hence $\FV(s) \setminus M = \{ x \} = \FV(t) \setminus X$.
    \item If $s = \avar \in \Vbound$ and $\mu(\avar) > 0$ then $t = \bvar \in \Vbound$ and
      $\xi(\bvar) = mu(\avar) > 0$ as well.  Therefore $\avar \in M$ and $\bvar \in X$.
      Hence $\FV(s) \setminus M = \emptyset = \FV(t) \setminus X$.
    \item If $s = \abs{x}{s'}$ then $t = \abs{y}{t'}$ and $s' =_\alpha^{\mu[x:=k],\xi[y:=k],k+1} t'$.
      We have $\FV(s) \setminus M = (\FV(s') \setminus \{ x \}) \setminus M = \FV(s') \setminus \{ z \mid \mu[x:=k](z) \neq 0 \}$.
      Moreover, $\FV(t) \setminus X = (\FV(t') \setminus \{ z \mid \xi[y:=k](z) \neq 0 \}$.
      We complete again by the induction hypothesis.
      \qedhere
    \end{itemize}
}
\end{proof}

Choosing $k = 1$ and $\mu,\xi$ the empty functions again, we obtain:

\begin{corollary}\label{corr:alphafreevar}
If $s =_\alpha t$ then $\FV(s) = \FV(t)$.
\end{corollary}

Next, we define some helper results that have little meaning on their own, but will prove useful
when reasoning about $\alpha$-equivalence (especially in combination with substitution).

\begin{lemma}\label{lem:alphaincrease}
If $s =_\alpha^{\mu,\xi,k} t$ then $s =_\alpha^{\mu,\xi,k+1} t$.
\end{lemma}

\begin{proof}
For an integer $0 < n \leq k$, let $\mathit{up}_n(i) = i$ for $i \leq n$ and $\mathit{up}_n(i) = i+1$ for $i > n$.
Let $\mu_n(x) = \mathit{up}(\mu(x))$ and $\xi_n(x) = \mathit{up}_n(x)$.
We will prove by induction on $s$ that for fixed $n > 0$, all $k \geq n$: if $s =_\alpha^{\mu,\xi,k} t$ then $s =_\alpha^{\mu_n,\xi_n,k+1} t$.
The proof is straightforward.
\hideproof{
    \begin{itemize}
    \item If $s = a(s_1,\dots,s_m)$ then $t = b(t_1,\dots,b_m)$ with $a =_\alpha^{\mu,\xi,k} b$ and
      each $s_i =_\alpha^{\mu,\xi,k} t_i$.
      By the IH, $a =_\alpha^{\mu_n,\xi_n,k+1} b$ and each $s_i =_\alpha^{\mu_n,\xi_n,k+1} t_i$;
      hence, the result follows.
    \item If $s = \afun$ or $s = \avar \in \Vfree$, then $t = s$ and the result immediately follows.
    \item If $s = \avar \in \Vbound$ and $\mu(\avar) = 0$ then $t = \avar$ and $\xi(\avar) = 0$;
      then also $\mu_n(\avar) = \xi_n(\avar) = 0$, and we are done.
    \item If $s = \avar \in \Vbound$ and $\mu(\avar) > 0$ then $t = \bvar \in \Vbound$ and
      $\xi(\avar) = \mu(\avar) > 0$.  But then, whether $\mu(\avar) > n$ or not, $\mu_n(\avar) =
      \mathit{up}(\mu(\avar)) = \mathit{up}(\xi(\bvar)) = \xi_n(\avar)$ as well, giving the
      required result.
    \item If $s = \abs{x}{s'}$ then $t = \abs{y}{t'}$ and $s' =_\alpha^{\mu[x:=k],\xi[y:=k],k+1} t'$.
      Since $k \geq n$, certainly $(\mu[x:=k])_n = \mu_n[x:=k+1]$ and $(\xi[y:=k])_n = \xi_n[y:=k+1]$.
      Hence, by the induction hypothesis, $s' =_\alpha^{\mu_n[x:=k+1],\xi_n[x:=k+1],k+2} t'$ and
      therefore $s =_\alpha^{\mu_n,\xi_n,k+1} t$.
      \qedhere
    \end{itemize}
}
\end{proof}

\begin{lemma}\label{lem:alphaunusedvar}
Suppose $s =_\alpha^{\mu,\xi,k} t$ and $\mu'(x) = \mu(x)$ for all $x \in \FV(s) \cap \Vbound$, and $\xi'(y) = \xi(y)$ for all $y \in \FV(t)$.
Then $s =_\alpha^{\mu',\xi',k} t$.
\end{lemma}

\begin{proof}
By induction on the size of $s$.
All cases are straightforward.
\hideproof{
    \begin{itemize}
    \item If $s = a(s_1,\dots,s_n)$ with $n > 0$ then $t = b(t_1,\dots,t_n)$ and $a =_\alpha^{\mu,\xi,k} b$
      and each $s_i =_\alpha^{\mu,\xi,k} t_i$.
      By the induction hypothesis, $a =_\alpha^{\mu',\xi',k} b$ and each $s_i =_\alpha^{\mu',\xi',k} t_i$.
      This immediately gives $s =_\alpha^{\mu',\xi',k} t$ as required.
    \item If $s = \afun$ or $s = x \in \Vfree$ then $t = s$ and the result immediately follows.
    \item If $s = \avar \in \Vbound$ and $\mu(\avar) = 0$ then $t = \avar$ and $\xi(\avar) = 0$.
      Since clearly $\avar \in \FV(s)$ and $\avar \in \FV(t)$ we have $\mu'(x) = \xi'(y) = 0$ as well.
      Hence indeed $s =_\alpha^{\mu',\xi',k} t$.
    \item If $s = \avar \in \Vbound$ and $\mu(\avar) > 0$ then $t = \bvar \in \Vbound$ and $\xi(\bvar)
      > 0$.  Since clearly $\avar \in \FV(s)$ and $\bvar \in \FV(t)$ we have $\mu'(\avar) = \mu(\avar)
      = \xi(\bvar) = \xi'(\bvar) > 0$ as well.
    \item If $s = \abs{x}{s'}$ then $t = \abs{y}{t'}$ and $s' =_\alpha^{\mu[x:=k],\xi[y:=k],k+1} t'$.
      Now,for all $z \in \FV(s')$: either $z = x$ and $\mu'[x:=k](z) = k = \mu[x:=k](z)$, or $z \in
      \FV(s)$ and $\mu[x:=k](z) = \mu(z) = \mu'(z) = \mu'[x:=k](z)$ by assumption.
      Similarly, for all $z \in \FV(t')$ we have $\xi'[y:=k](z) = \xi[y:=k](z)$.
      Hence we can apply the induction hypothesis on $s'$ to obtain $s' =_\alpha^{\mu'[x:=k],\xi'[y:=k],k+1} t'$.
      This immediately implies $s =_\alpha^{\mu',\xi',k} t$.
      \qedhere
    \end{itemize}
}
\end{proof}

We also observe that $\alpha$-equality interacts well with application:

\begin{lemma}\label{lem:alphaappl}
Let $s = u \cdot s_1 \cdots s_n$ and $t = v \cdot t_1 \cdots t_n$.
Then $s =_\alpha^{\mu,\xi,k} t$ if and only if $u =_\alpha^{\mu,\xi,k} v$ and
$s_i =_\alpha^{\mu,\xi,k} t_i$ for $1 \leq i \leq n$.
\end{lemma}

\begin{proof}
This is easily seen by inspecting the definitions.
\hideproof{
    First suppose $u =_\alpha^{\mu,\xi,k} v$ and each $s_i =_\alpha^{\mu,\xi,k} t_i$.
    Consider the form of $u$; we can always write $u = a(u_1,\dots,u_m)$ with $m \geq 0$, and
    since $u =_\alpha^{\mu,\xi,k}$, necessarily $v = b(v_1,\dots,v_m)$ with $a =_\alpha^{\mu,
    \xi,k} b$ and each $u_i =_\alpha^{\mu,\xi,k} v_i$.  But then $s = a(u_1,\dots,u_m,s_1,
    \dots,s_n)$ and $t = b(v_1,\dots,v_m,t_1,\dots,t_n)$; it immediately follows that
    $s =_\alpha^{\mu,\xi,k} t$.
    
    Alternatively, suppose that $s =_\alpha^{\mu,\xi,k} t$.  Again denote $u = a(u_1,\dots,u_m)$.
    Then necessarily $s = a(u_1,\dots,u_m,s_1,\dots,s_n)$.  Also write $v = b(v_1,\dots,v_k)$;
    then $t = b(v_1,\dots,v_k,t_1,\dots,t_n)$.  But by definition of $s =_\alpha^{\mu,\xi,k} t$
    necessarily $m + n = k + n$; that is, $m = k$.  Moreover, necessarily $a =_\alpha^{\mu,\xi,k}
    b$, each $u_i =_\alpha^{\mu,\xi,k} b$, and each $s_i =_\alpha^{\mu,\xi,k} t_i$.  But then
    also $u = a(u_1,\dots,u_m) =_\alpha^{\mu,\xi,k} b(v_1,\dots,v_m) = v$.
}
\end{proof}

\subsection{Well-definedness of substitution}

\newcommand{\aprel}[1]{\mathsf{ap}(#1)}
\newcommand{\subrel}[1]{\mathsf{subst}(#1)}

Avoiding the suggestive equality notation, let us reformulate the notion of substitution as a \emph{relation}, as follows:

\begin{enumerate}
\item\label{subst:appl} $\subrel{h(s_1,\dots,s_n), \gamma, u \cdot t_1 \cdots t_n)}$ if $\subrel{h,\gamma,u}$ and $\subrel{s_i,\gamma,t_i}$ for $1 \leq i \leq n$;
\item\label{subst:func} $\subrel{\afun,\gamma,\afun}$
\item\label{subst:var} $\subrel{\avar,\gamma,\gamma(\avar)}$
\item\label{subst:abs} $\subrel{\abs{\avar}{s},\gamma,\abs{\cvar}{q}}$ if $\cvar$ has the same type as $\avar$ and:
  \begin{itemize}
  \item $\cvar \notin \FV(\gamma(\bvar))$ for any $\bvar \in \FV(\abs{\avar}{s})$;
  \item $\subrel{s,[\avar:=\cvar] \cup [\bvar := \gamma(\bvar) \mid \bvar \in \domain(\gamma) \setminus \{\avar\}],q}$.
  \end{itemize}
\item\label{subst:boringmeta} $\subrel{\meta{\avar}{s_1,\dots,s_n},\gamma,\meta{\avar}{t_1,\dots,t_n}}$ if
  $x \notin \domain(\gamma)$ and $\subrel{s_1,\gamma,t_i}$ for $1 \leq i \leq n$
\item\label{subst:meta} $\subrel{\meta{\avar}{s_1,\dots,s_k},\gamma,q}$ if $\avar \in \domain(\gamma)$ and
  $\gamma(\avar) = \abs{\bvar_1 \dots \bvar_k}{t}$ and there exist $t_1,\dots,t_k$ such that
  $\subrel{s_i,\gamma,t_i}$ for $1 \leq i \leq n$ and
  $\subrel{t,[\bvar_1:=t_1,\dots,\bvar_k:=t_k],q}$.
\end{enumerate}
It is easy to see that this defines the same relation as substitution in the main text.

We can now specify and prove the well-definedness result:

\begin{lemma}\label{lem:substdefined}
For every term $s$ and substitution $\gamma$ there exists $q$ with $\subrel{s,\gamma,q}$. %$s\gamma = q$.
\end{lemma}

\begin{proof}
We prove the statement by induction on (1) the boolean $\domain(\gamma) \subseteq \Vbound$ (with
$\bot > \top$), (2) the size of $s$.  All cases are easy by the second induction hypothesis, except
for the meta-application case $\meta{\avar}{s_1,\dots,s_k}$ with $\avar \in \domain(\gamma)$; but
here we quickly complete with the first induction hypothesis.
\hideproof{
    \begin{itemize}
    \item If $s = h(s_1,\dots,s_n)$ with $n > 0$ then by the second induction hypothesis there exist
      $u$ such that $\subrel{h,\gamma,u}$ and $t_1,\dots,t_n$ such that $\subrel{s_i,\gamma,t_i}$ for
      all $i$.  As normal application (the $\cdot$ operator) is unambiguously defined, we can choose
      $q := u \cdot t_1 \cdots t_n$.
    \item If $s = \afun$ we can choose $q := \afun$.
    \item If $s = \avar$ we can choose $q := \gamma(\avar)$ (note that $\gamma(\avar) = \avar$ when
      $\avar \notin \domain(\gamma)$).
    \item If $s = \abs{\avar}{s'}$ then first note that a suitable $\cvar$ can always be found, as $\Vbound$
      has infinitely many variables of all types and $\bigcup_{\bvar \in \FV(s)}
      \FV(\gamma(\bvar))$ is finite.  Then, writing $\delta := [\avar:=\cvar] \cup [\bvar:=\gamma(\bvar) \mid
      \bvar \in \domain(\gamma) \setminus \{\avar\}]$, there exists $q'$ such that
      $\subrel{s',\delta,q'}$ by the second induction hypothesis on $s'$ (adding a variable in
      $\Vbound$ to the domain does not affect the value for the first IH).
      We complete with $q := \abs{\cvar}{q}$.
    \item If $s = \meta{x}{s_1,\dots,s_k}$ with $\avar \notin \domain(\gamma)$, then by the second induction
      hypothesis we find $t_1,\dots,t_k$ with $\subrel{s_i,\gamma,t_i}$ for all $i$.  We can choose
      $q := \meta{x}{t_1,\dots,t_k}$.
    \item If $s = \meta{\avar}{s_1,\dots,s_k}$ with $\avar \in \domain(\gamma)$, then note that
      $\avar \in \Vmeta$ so the first value for the induction hypothesis is $\bot$. By the second
      induction hypothesis we find $t_1,\dots,t_k$ with $\subrel{s_i,\gamma,t_i}$ for all $i$.
      Now, we can write $\gamma(\avar) = \abs{\bvar_1,\dots,\bvar_k}{t}$.
      Let $\delta := [\bvar_1:=t_1,\dots,\bvar_k:=t_k]$.
      Observing that $\domain(\delta) = \{\bvar_1,\dots,\bvar_k\} \subseteq \Vbound$, we can apply
      the first induction hypothesis to find $q$ such that $\subrel{t,\delta,q}$ and therefore
      $\subrel{s,\gamma,q}$.
      \qedhere
    \end{itemize}
}
\end{proof}

Towards the second result for well-definedness, that substitution and application define functions
on equivalence classes, we specify the following lemma.  This is formulated more generally than we
need to obtain an easier induction.

\begin{lemma}\label{lem:substitutionalpha}
Let $k$ be an integer, and $\nu,\chi : \Vbound \mapsto \{0,\dots,k\}$.
Assume given terms $s,s'$, an integer $p$, mappings $\mu,\xi : \Vbound \mapsto \{0,\dots,p\}$ and substitutions $\gamma,\gamma'$ such that:
\begin{itemize}
\item $\domain(\gamma) \cap \Vmeta = \domain(\gamma') \cap \Vmeta$
\item for all $x \in \FV(s)$ with $\mu(x) = 0$ we have:
  $\gamma(x) =_\alpha^{\nu,\chi,k+1} \gamma'(x)$ \\
  (for simplicity, we let $\mu(\avar) = \chi(\avar) = 0$ when $\avar \in \Vmeta$)
\item for all $x \in \FV(s)$, $y \in \FV(s')$ such that $\mu(x) = \xi(y) > 0$ we have:
  $\gamma(x) =_\alpha^{\nu,\chi,k+1} \gamma'(y)$
\item $s =_\alpha^{\mu,\xi,p+1} s'$
\end{itemize}
Assume given $t,t'$ such that $\subrel{s,\gamma,t}$ and $\subrel{s',\gamma',t'}$.  Then $t =_\alpha^{\nu,\chi,k+1} t'$.
\end{lemma}

\begin{proof}
We prove this by induction on the definition of $\subrel$ (so, essentially on the pair
($\domain(\gamma)$ contains non-binder variables, size of $s$) as in the proof of Lemma \ref{lem:substdefined}).
Although the proof is not overly complex, it is quite long, and requires multiple applications of
both Lemmas \ref{lem:alphafreevar} and  \ref{lem:alphaappl}, as well as an application of both
Lemmas \ref{lem:alphaincrease} and \ref{lem:alphaunusedvar}.
\hideproof{
    Consider the shape of $s$.
    \begin{itemize}
    \item Suppose $s = h(s_1,\dots,s_n)$ with $n > 0$.
      Then $s' = h'(s_1',\dots,s_n')$ with $h =_\alpha^{\mu,\xi,p+1} h'$ and $s_i =_\alpha^{\mu,\xi,p+1} s_i'$ for all $i$. \\
      From $\subrel{s,\gamma,t}$ we obtain: $t = u \cdot t_1 \cdots t_n$ with $\subrel{h,\gamma,u}$ and $\subrel{s_i,\gamma,t_i}$ for all $i$.
      From $\subrel{s',\gamma',t'}$ we obtain: $t' = u' \cdot t_1' \cdots t_n'$ with $\subrel{h',\gamma',u'}$ and $\subrel{s_i',\gamma',t_i'}$ for all $i$. \\
      Hence, by the induction hypothesis on $\subrel{h,\gamma,u}$ and $\subrel{h',\gamma',u'}$ we have: $u =_\alpha^{\nu,\chi,k+1} u'$. \\
      Similarly, by the IH on each $\subrel{s_i,\gamma,t_i}$ and $\subrel{s_i',\gamma',t_i'}$ we have: $t_i =_\alpha^{\nu,\chi,k+1} t_i'$. \\
      By Lemma \ref{lem:alphaappl} we obtain the required conclusion $t =_\alpha^{\nu,\chi,k+1} t'$.
    \item Suppose $s = \afun$.  Then $s' = t = t' = \afun$ and indeed $t =_\alpha^{\nu,\chi,k+1} t'$.
    \item Suppose $s = \avar$ with $\avar \in \Vfree$, or $\avar \in \Vbound$ and $\mu(\avar) = 0$.
      Then $s' = \avar$ and $t = \gamma(x) =_\alpha^{\nu,\chi,k+1} \gamma'(x) = t'$ by assumption.
    \item Suppose $s = \avar$ with $\mu(\avar) > 0$.  Then $s' = \bvar$ with $\xi(\bvar) = \mu(\avar)$, so also by assumption
      $t = \gamma(x) =_\alpha^{\nu,\chi,k+1} \gamma'(y) = t'$.
    \item Suppose $s = \abs{x}{u}$. Then:
      \begin{itemize}
      \item $s' = \abs{x'}{u'}$ for some $u'$ with $u =_\alpha^{\mu[x:=p+1],\xi[x':=p+1],p+2} u'$;
      \item $t = \abs{z}{q}$ and $t' = \abs{z'}{q'}$ for some $z,z'$ such that:
        \begin{itemize}
        \item $z \notin \FV(\gamma(y))$ for any $y \in \FV(s)$ and \\
          $z' \notin \FV(\gamma'(y))$ for any $y \in \FV(s')$;
        \item $\subrel{u,\delta,q}$ for $\delta := [x:=z] \cup [y:=\gamma(y) \mid y \in \domain(\gamma) \setminus \{x\}]$ and \\
          $\subrel{u',\delta',q'}$ for $\delta' := [x':=z'] \cup [y:=\gamma'(y) \mid y \in \domain(\gamma') \setminus \{x'\}]$
        \end{itemize}
      \end{itemize}
      \ \\
      Now, let $\mu_b := \mu[x:=p+1]$ and $\xi_b := \xi[x':=p+1]$ and $\nu_b := \nu[z:=k+1]$ and $\chi_b := \chi[z':=k+1]$.
      We apply the induction hypothesis on $\subrel{u,\delta,q}$ and $\subrel{u',\delta',q'}$. (**)
      This gives $q =_\alpha^{\nu_b,\chi_b,k+2} q'$, and therefore $t = \abs{z}{q} =_\alpha^{\nu,\chi,k+1} \abs{z'}{q'} = t'$ as required.
    
      (**) To see that we may apply the induction hypothesis to obtain this conclusion, we observe that
      $\nu_b,\chi_b$ are functions in $\Vbound \to \{0,\dots,k+1\}$, that $\mu_b,\xi_b$ are functions in $\Vbound \to \{0,\dots,p+1\}$,
      and that $u =_\alpha^{\mu_b,\xi_b,p+2} u'$ as observed above.
      Clearly $\domain(\delta) \cap \Vmeta = \domain(\gamma) \cap \Vmeta = \domain(\gamma') \cap \Vmeta = \domain(\delta') \cap \Vmeta$.
      For the substitutions, we must show that for all $y \in \FV(u)$:
      \begin{itemize}
      \item if $\mu_b(y) = 0$ then $\delta(y) =_\alpha^{\nu_b,\chi_b,k+2} \delta'(y)$;
      \item if $\mu_b(y) > 0$ and $y' \in \FV(u')$ is such that $\mu_b(y) = \xi_b(y')$ then $\delta(y) =_\alpha^{\nu_b,\chi_b,k+2} \delta'(y')$.
      \end{itemize}
      \ \\
      So let $y \in \FV(u)$; if $y \in \Vfree$ or $\mu_b(y) = 0$ then let $y' := y$, otherwise let $y'$ be such that $\mu_b(y) = \xi_b(y')$.
    
      If $y = x$, then $\mu_b(y) = p + 1$, so we should show the second case.
      Since $\xi$ maps to $\{0,\dots,p\}$ and $\xi_b(x') = p + 1$, necessarily $y' = x'$.
      Hence we must show: $z = \delta(x) =_\alpha^{\nu_b,\chi_b,k+2} \delta'(x') = z'$.
      Since $\nu_b(z) = k + 1 = \chi_b(z')$ this clearly holds.
    
      Alternatively, if $y \neq x$, then $\mu_b(y) = \mu(y) \leq p$.
      \begin{itemize}
      \item If $\mu_b(y) > 0$ and $y' \in \FV(u')$ has $\mu_b(y) = \xi_b(y')$, then note that $y' \neq x'$, as $\xi_b(x') = p+1 > \mu_b(y)$.
        Hence, $\xi_b(y') = \xi(y')$.
      \item Otherwise, $y' = y$ by definition and since $y \in \FV(u) \setminus \{ x \} = \FV(s)$ we can apply Lemma \ref{lem:alphafreevar}
        to obtain $y \in \FV(s')$ and $\xi(y) = 0$; so $y \in \FV(u')$ and $y \neq x'$, hence $\xi_b(y) = \xi(y) = 0$.\\
      \end{itemize}
      Hence, either way, $y' \neq x'$ and $\xi_b(y') = \xi(y')$.
      Then also $\delta(y) = \gamma(y)$ and $\delta'(y') = \gamma'(y')$, and by the assumptions on $\gamma,\gamma'$ we have:
      $\delta(y) =_\alpha^{\nu,\chi,k+1} \delta'(y)$.
      Hence by Lemma \ref{lem:alphaincrease}, $\delta(y) =_\alpha^{\nu,\chi,k+2} \delta'(y)$. (a) \\
      Now, since $y \in \FV(u)$ and $y \neq x$, we have $y \in \FV(s)$.  Similarly, $y' \in \FV(s')$.
      By the freshness condition on $z,z'$ we have: $z \notin \FV(\gamma(y)) = \FV(\delta(y))$, and $z' \notin \FV(\delta'(y'))$.
      But then by (a) and Lemma \ref{lem:alphaunusedvar}, $\delta(w) =_\alpha^{\nu_b,\chi_b,k+2} \delta'(w')$ as required.
    \item Suppose $s = \meta{\avar}{s_1,\dots,s_n}$ and $\avar \notin \domain(\gamma)$, so by assumption (since $\avar \in \Vmeta$)
      also $\avar \notin \domain(\gamma)$.  There exist $t_1,\dots,t_n$ such that $\subrel{s_i,\gamma,t_i}$ and $t =
      \meta{\avar}{t_1,\dots,t_n}$, and $t_1',\dots,t_n'$ such that $\subrel{s_i,\gamma,t_i}$ and $t' = \meta{\avar}{t_1',\dots,t_n'}$.
      Since, by the induction hypothesis, each $t_i =_\alpha^{\nu,\chi,k+1} t_i'$, we directly obtain $t =_\alpha^{\nu,\chi,k+1} t'$.
    \item Finally, suppose $s = \meta{\avar}{s_1,\dots,s_n}$ and $\gamma(\avar) = \abs{\bvar_1 \dots \bvar_n}{q}$.  Then:
      \begin{itemize}
      \item $s' = \meta{\avar}{s_1',\dots,s_n'}$ with $s_i =_\alpha^{\mu,\xi,p+1} s_i'$ for all $i$;
      \item $\gamma'(\avar) = \abs{\cvar_1 \dots \cvar_n}{q'}$ with $q =_\alpha^{\nu',\chi',k+n+1} q'$,
        where $\nu' := \nu[\bvar_1:=k+1]\dots[\bvar_n:=k+n]$ and $\chi' := \chi[\cvar_1:=k+1]\dots[\cvar_n:=k+n]$
      \item there exist $t_1,\dots,t_n$ such that $\subrel{s_i,\gamma,t_i}$ for all $i$ and
        $\subrel{q,\delta[\bvar_1:=t_1,\dots,\bvar_n:=t_n],t}$ where $\delta := [\bvar_1:=t_1,\dots,\bvar_n:=t_n]$
      \item there exist $t_1',\dots,t_n'$ such that $\subrel{s_i',\gamma,t_i'}$ for all $i$ and
        $\subrel{q',\delta',t'}$ where $\delta' := [\cvar_1:=t_1',\dots,\cvar_n:=t_n']$
      \end{itemize}
      \ \\
      By the induction hypothesis on each $s_i$, we obtain from $\subrel{s_i,\gamma,t_i}$ and $\subrel{s_i',\gamma,t_i'}$ that
      $t_i =_\alpha^{\nu,\chi,k+1} t_i'$.
      By the induction hypothesis on the presence of meta-variables in the domain of the $\gamma$, we obtain from:
      \begin{itemize}
      \item $\domain(\delta) \cap \Vmeta = \domain(\delta') \cap \Vmeta = \emptyset$
      \item for all $\avar \in \FV(q)$ with $\nu'(\avar) = 0$ we have:
        $\delta(\avar) =_\alpha^{\nu,\chi,k+1} \delta'(\avar)$
        \begin{itemize}
        \item[]
          (if $\nu'(\avar) = 0$ then $\avar \notin \domain(\delta)$, so (a) $\delta(\avar) = \avar$;
          and since $\avar \in \FV(q) \setminus \{ \cvar \mid \nu'(\cvar) \neq 0 \}$ we have
          $\avar \in \FV(q') \setminus \{ \cvar \mid \chi'(\cvar) \neq 0 \}$ by Lemma
          \ref{lem:alphafreevar}, so (b) $\delta'(\avar) = \avar$;
          and since $\nu'(\avar) = \chi'(\avar) = 0$ also (c) $\nu(\avar) = \chi(\avar) = 0$;
          from (a)--(c) we obtain $\delta(\avar) = \avar =_\alpha^{\nu,\chi,k+1} \avar = \delta'(\avar)$)
        \end{itemize}
      \item for all $\avar \in \FV(q)$, $\avar' \in \FV(q')$ such that $\nu'(\avar) = \chi'(\avar') > 0$
        also $\delta(\avar) =_\alpha^{\nu,\chi,k+1} \delta'(\avar)$
        \begin{itemize}
        \item[]
          (if $k < \nu'(\avar) \leq k+n$ then $\avar = \bvar_i$ and $\avar' = \cvar_i$ for some $i$;
          and $\delta(\avar) = t_i =_\alpha^{\nu,\chi,k+1} t_i' = \delta'(\avar')$ by assumption;
          otherwise, $0 < \nu'(\avar) \leq k$ so $\nu'(\avar) = \nu(\avar)$ and $\chi'(\avar') =
          \chi(\avar')$, so $\delta(\avar) = \avar =_{\nu,\chi,k+1} \avar' = \delta'(\avar')$)
        \end{itemize}
      \item $q =_\alpha^{\nu',\chi',k+n+1} q'$
      \end{itemize}
      \ \\
      that $t =_\alpha^{\nu,\chi,k+1} t'$ as required.
      \qedhere
    \end{itemize}
}
\end{proof}

\begin{corollary}\label{cor:substitutionalpha}
If $s =_\alpha s'$ and $\subrel{s,\gamma,t}$ and $\subrel{s',\gamma',t'}$ and
  $\gamma(x) =_\alpha \gamma'(x)$ for all $x \in \FV(s)$, then $t =_\alpha t'$.
\end{corollary}

\subsection{Some results on $\alpha$-equivalence and substitution}

We will now prove several useful results regarding substitution that will often be silently used in
proofs.

\begin{lemma}\label{lem:appsubstitute}
We have:
\begin{enumerate}
\item
  If $\subrel{s_i,\gamma,t_i}$ for $0 \leq i \leq n$, then $\subrel{s_0 \cdot s_1 \cdots s_n,\gamma,
  t_0 \cdot t_1 \cdots t_n}$.
\item
  If $\subrel{s_0 \cdot s_1 \cdots s_n,\gamma,t}$, then we can write $t = t_0 \cdot t_1 \cdots t_n$
  with $\subrel{s_i,\gamma,t_i}$ for $0 \leq i \leq n$.
\end{enumerate}
\end{lemma}

\begin{proof}
Both sides are immediate by definition.
\hideproof{
    In the following, let $s = s_0 \cdot s_1 \cdots s_n$, and write $s_0 = h(u_1,\dots,u_k)$ with $k
    \geq 0$ (this is always possible), so $s = h(u_1,\dots,u_k,s_1,\dots,s_n)$.
    
    First suppose $\subrel{s_i,\gamma,t_i}$ for $0 \leq i \leq n$.
    Then (by definition of $\subrel{}$) necessarily $t_0 = w \cdot v_1 \cdots v_k$, where
    $\subrel{h,\gamma,w}$ and $\subrel{u_i,\gamma,v_i}$ for $1 \leq i \leq k$.
    Hence, writing $t = t_0 \cdot t_1 \cdots t_n$ we have $t = (w \cdot v_1 \cdots v_k) \cdot t_1
    \cdots t_n$.  Since $\cdot$ is left-associative, this is exactly $w \cdot v_1 \cdots v_k \cdot
    t_1 \cdots t_n$, and $\subrel{h(u_1,\dots,u_k,s_1,\dots,s_n),\gamma,t}$ follows immediately.
    
    Next suppose $\subrel{s,\gamma,t}$.  Then $t = w \cdot v_1 \cdots v_k \cdot t_1 \cdots t_n$ with
    $\subrel{h,\gamma,w}$ and $\subrel{u_i,\gamma,v_i}$ for $1 \leq i \leq k$ and $\subrel{s_i,\gamma,
    t_i}$ for $1 \leq i \leq n$.  Write $t_0 := w \cdot v_1 \cdots v_k$; then also $t = t_0 \cdot t_1
    \cdots t_n$, and $\subrel{s_0,\gamma,t_0}$ follows immediately.
}
\end{proof}

Hence, viewing substitution as a function rather than a relation, we have shown that $(s_0 \cdot
s_1 \cdots s_n)\gamma = (s_0\gamma) \cdot (s_1\gamma) \cdots (s_n\gamma)$ as claimed in the text.

\begin{lemma}\label{lem:substitutionrefl}
For all terms $s$ and substitutions $\gamma$ with $\gamma(\avar) = \avar$ for all $\avar \in \FV(s)$:
$\subrel{s,\gamma,s}$.
\end{lemma}

\begin{proof}
By a straightforward induction on the size of $s$.
\hideproof{
    \begin{itemize}
    \item If $s = h(s_1,\dots,s_n)$ with $n > 0$ then by the induction hypothesis,
      $\subrel{h,\gamma,h}$ and $\subrel{s_i,\gamma,s_i}$ for all $i$; hence
      $\subrel{s,\gamma,h \cdot s_1 \cdots s_n}$, and $h \cdot s_1 \cdots s_n$ is exactly $s$.
    \item If $s = \afun$ then clearly $\subrel{s,\gamma,s}$
    \item If $s = \avar$ then by assumption $\gamma(\avar) = \avar$, so $\subrel{s,\gamma,s}$.
    \item If $s = \abs{\avar}{s'}$ then note that $\gamma(\bvar) = \bvar$ for all $\bvar \in \FV(s)$,
      and that $\avar \notin \FV(s)$, so clearly $\avar \notin \FV(\gamma(\bvar))$ holds for all $\bvar
      \in \FV(s)$
      Also note that $\gamma' := [\avar:=\avar] \cup [\bvar:=\gamma(\bvar) \mid \bvar \in \domain(\gamma)
      \setminus \{\avar\}]$ satisfies the property that $\gamma'(\bvar) = \bvar$ for all $\bvar \in
      \FV(s')$ (since it holds both for $s'$ and all variables in $\FV(s)$).
      Hence, we can apply the induction hypothesis to obtain $\subrel{s',\gamma',s'}$, and therefore
      $\subrel{s,\gamma,\abs{\avar}{s'}}$.
    \item If $s = \meta{\avar}{s_1,\dots,s_k}$ with $k > 0$, then note that $\avar \notin
      \domain(\gamma)$, as otherwise $\gamma(\avar)$ would have to be an abstraction.
      By the induction hypothesis, $\subrel{s_i,\gamma,s_i}$ for $1 \leq i \leq k$.
      Since $\avar \notin \domain(\gamma)$, we indeed have
      $\subrel{s,\gamma,\meta{\avar}{s_1,\dots,s_k}}$.
    \qedhere
    \end{itemize}
}
\end{proof}

Hence, applying an irrelevant substitution can be done without effect; even without any variables
being renamed.

\begin{lemma}\label{lem:substextend}
If $\gamma(x) = \delta(x)$ for all $x \in \FV(s)$, then $\subrel{s,\gamma,t}$ implies $\subrel{s,\delta,t}$.
\end{lemma}

\begin{proof}
We prove the lemma by a straightforward induction on the size of $s$.
For the abstraction case, we observe that the substitutions
$[\avar:=\cvar] \cup [\bvar:=\gamma(\bvar) \mid \bvar \in \domain(\gamma) \setminus \{\avar\}]$ and
$[\avar:=\cvar] \cup [\bvar:=\delta(\bvar) \mid \bvar \in \domain(\delta) \setminus \{\avar\}]$
agree on all variables in $\FV(s) \cup \{\avar\}$ because $\gamma$ and $\delta$ do.
In the case $\meta{x}{s_1,\dots,s_k}$, we use that $\gamma(x) = \delta(x)$, and consequently do not
require an additional induction hypothesis on the presence of meta-variables in $\domain(\gamma)$.
\hideproof{
    Cases:
    \begin{itemize}
    \item 
      If $s = h(s_1,\dots,s_n)$ with $n > 0$ then $t = w \cdot t_1 \cdots t_n$ with $\subrel{h,\gamma,w}$ and
      $\subrel{s_i,\gamma,t_i}$ for all $i$.  Note that $\FV(w) \subseteq \FV(s)$, and therefore also
      $\gamma(x) = \delta(x)$ for all $x \in \FV(w)$; the same holds for all $s_i$.  Hence, by the induction
      hypothesis, also $\subrel{h,\delta,w}$ and $\subrel{s_i,\delta,t_i}$ for all $i$, so cleary also
      $\subrel{s,\delta,t}$.
    \item
      If $s = \afun$ then $t = \afun$ and indeed $\subrel{s,\delta,t}$.
    \item
      If $s = \avar$ then $t = \gamma(\avar) = \delta(\avar)$ by assumption (as $\avar \in \FV(s)$).
    \item
      If $s = \abs{\avar}{s'}$ then $t = \abs{\cvar}{t'}$ where
      (a) $\cvar$ has the same type as $\avar$,
      (b) $\cvar \notin \FV(\gamma(\bvar))$ for any $\bvar \in \FV(s)$, and therefore also $\cvar \notin
      \FV(\delta(\bvar))$ for any $\bvar \in \FV(s)$, and
      (c) $\subrel{s',[\avar:=\cvar] \cup [\bvar:=\gamma(\bvar) \mid \bvar \in \domain(\gamma) \setminus
      \{\avar\}],t'}$, which by the induction hypothesis means that also
      $\subrel{s',[\avar:=\cvar] \cup [\bvar:=\delta(\bvar) \mid \bvar \in \domain(\delta) \setminus
      \{\avar\}],t'}$.
    \item If $s = \meta{x}{s_1,\dots,s_k}$ and $x \notin \domain(\gamma)$, then $\delta(x) = \gamma(x) = x$
      is not an abstraction, so also $x \notin \domain(\delta)$; and there exist $t_1,\dots,t_k$ with
      $\subrel{s_i,\gamma,t_i}$ and $t = \meta{x}{t_1,\dots,t_k}$.  By the induction hypothesis also
      $\subrel{s_i,\delta,t_i}$, so $\subrel{s,\delta,t}$ follows immediately.
    \item
      If $s = \meta{\avar}{s_1,\dots,s_k}$ and $\avar \in \domain(\gamma)$, then we can write
      $\gamma(\avar) = \delta(\avar) = \abs{\bvar_1 \dots \bvar_k}{q}$;
      and there exist $t_1,\dots,t_k$ with $\subrel{s_i,\gamma,t_i}$ and
      $\subrel{q,[\bvar_1:=t_1,\dots,\bvar_k:=t_k],t}$.  By the induction hypothesis also
      $\subrel{s_i,\delta,t_i}$, so we directly obtain $\subrel{s,\delta,t}$.
      \qedhere
    \end{itemize}
}
\end{proof}

Hence, the only relevant part of a substitution is its value on the free variables of the term it
is applied on.

Our following result is a helper to assess what substitution does to a longer abstraction.

\begin{lemma}\label{lem:abssubst}
If $\subrel{\abs{\avar_1 \dots \avar_n}{s},\gamma,u}$ then $u = \abs{\bvar_1 \dots \bvar_n}{t}$ such
that:
\begin{itemize}
\item each $\bvar_i \notin \FV(\gamma(\cvar))$ for any $\cvar \in \FV(s) \setminus \{\avar_1,\dots,\avar_n\}$
\item if $\bvar_i = \bvar_j$ for some $i < j$, then $\avar_i \notin \FV(s)$ or $\avar_i = \avar_k$ for some $k > i$ 
\item $\subrel{s,\gamma',t}$ where $\gamma' = [\avar_1:=\bvar_1,\dots,\avar_n:=\bvar_n] \cup [\cvar :=
  \gamma(\cvar) \mid \cvar \in \domain(\gamma) \setminus \{\vec{\avar}\}]$ \\
  (here, as before, if $\avar_i = \avar_j$ for some $i < j$ and not $\avar_j = \avar_k$ for any $k > j$,
  then $\gamma'(\avar_i) = \bvar_j$ -- the later assignments in the mapping may overwrite the earlier ones)
\end{itemize}
\end{lemma}

\begin{proof}
By induction on $n$.  For $n = 0$ the lemma clearly holds.
Now if $\subrel{\abs{\avar_1 \cdots \avar_n}{s},\gamma,u}$ and $n > 0$ then clearly $u = \abs{\bvar_1}{u'}$
with $\bvar_1 \notin \FV(\gamma(\cvar))$ for any $\cvar \in \FV(\abs{\avar_1 \dots \avar_n}{s}) =
\FV(s) \setminus \{\avar_1,\dots,\avar_n\}$, and $\subrel{\abs{\avar_2 \dots \avar_n}{s},[\avar_1:=\bvar_1]
\cup [\cvar:=\gamma(\cvar) \mid \cvar \in \domain(\gamma) \setminus \{\avar_1\}],u'}$.
Each of the three requirements now follows easily by the induction hypothesis.
\hideproof{
    Write $\gamma^{\avar_1:=\bvar_1} = [\avar_1:=\bvar_1] \cup [\cvar:=\gamma(\cvar) \mid \cvar \in \domain(\gamma)
    \setminus \{\avar_1\}]$.  We have:
    \begin{itemize}
    \item By definition of substitution, $\bvar_1 \notin \FV(\abs{\vec{\avar}}{s}) = \FV(s) \setminus
      \{\avar_1,\dots,\avar_n\}$.
    \item By the first part of the induction hypothesis:
      for $i > 1$ each $\bvar_i \notin \FV(\gamma^{\avar_1:=\bvar_1}(\cvar))$ for any
      $\cvar \in \FV(s) \setminus \{\avar_2,\dots,\avar_n\}$.  Since $\gamma^{\avar_1:=\bvar_1}(\cvar) =
      \gamma(\cvar)$ for any $\cvar \neq \avar_1$, this implies $\bvar_i \notin \FV(\gamma(\cvar))$ for any
      $\cvar \in \FV(s) \setminus \{\avar_1,\dots,\avar_n\}$. \\
      Combined with the previous point, we have proved the first requirement for all $i$.
    \item From $\bvar_i \notin \FV(\gamma^{\avar_1:=\bvar_1}(\cvar))$ for $i > 1$ and $\cvar \in \FV(s)
      \setminus \{\avar_2,\dots,\avar_n\}$ we also obtain (choosing $\cvar := \avar_1$) that
      if $\bvar_i = \bvar_1$ then $\avar_1 \notin \FV(s)$ or $\avar_1 = \avar_k$ for some $k > 1$.
    \item By the second part of the induction hypothesis: if $\bvar_i = \bvar_j$ for $1 < i < j$ then
      $\avar_i \notin \FV(s)$ or $\avar_i = \avar_k$ for some $k > i$.  Hence, combined with the previous
      point, this holds for \emph{all} $i < j$; which gives the second requirement we have to prove.
    \item By the last part of the induction hypothesis, $\subrel{s,\gamma',t}$, where $\gamma'(\avar_i) =
      [\avar_2:=\bvar_2,\dots,\avar_n:=\bvar_n] \cup [\cvar := \gamma^{\avar_1:=\bvar_1}(\cvar) \mid
      \cvar \in \domain(\gamma^{\avar_1:=\bvar_1}) \setminus \{\avar_2,\dots,\avar_n\}]$.
      This substitution has domain $\{\avar_2,\dots,\avar_n\} \cup \domain^{\avar_1:=\bvar_2} =
      \{\avar_1,\dots,\avar_n\} \cup \domain(\gamma)$; it maps:
      \begin{itemize}
      \item $\avar_1$ to $\bvar_1$ unless there exists $j > 1$ with $\avar_1 = \avar_k$, in which
        case it maps $\avar_1$ to $\bvar_k$ for $k$ the largest number with $\avar_1 = \avar_k$
      \item $\avar_i$ to $\bvar_i$ if there is no $j > i$ with $\avar_i = \avar_j$ (for $i > 1$)
      \item all other variables $\cvar$ to $\gamma(\cvar)$
      \end{itemize}
      \ \\
      That is, this substitution is exactly $[\avar_1:=\bvar_1,\dots,\avar_n:=\bvar_n] \cup [\cvar:=
      \gamma(\cvar) \mid \cvar \in \domain(\gamma) \setminus \{\avar_1,\dots,\avar_n\}]$.
      \qedhere
    \end{itemize}
}
\end{proof}

\begin{lemma}\label{lem:alternativealpha}
Let $s,t$ be terms, and $\avar,\bvar \in \Vbound$ variables of the same type.
Then $\abs{\avar}{s} =_\alpha \abs{\bvar}{t}$ if and only if $\bvar \notin \FV(\abs{\avar}{s})$ and $\subrel{s,[\avar:=\bvar],t}$.
\end{lemma}

\begin{proof}
First suppose $\bvar \notin \FV(s)$ and $\subrel{s,[\avar:=\bvar],t}$.  Then by definition,
$\subrel{\abs{\avar}{s}, [], \abs{\bvar}{t}}$.
By Lemma \ref{lem:substitutionrefl}, $\subrel{\abs{\avar}{s}, [], \abs{\avar}{s}}$, so
by Lemma \ref{lem:substitutionalpha}, $\abs{\avar}{s} =_\alpha \abs{\bvar}{t}$.

Next suppose $\abs{\avar}{s} =_\alpha \abs{\bvar}{t}$.  As $\avar \notin \FV(s)$, by Corollary
\ref{corr:alphafreevar}, $\bvar \notin \FV(\abs{\bvar}{t})$.
We can prove:
\begin{itemize}
\item if $\mu,\xi : \Vbound \mapsto \{0,\dots,k\}$ such that for all $\cvar_1,\cvar_2 \in
  \Vbound$: if $\mu(\cvar_1) = \mu(\cvar_2) > 0$ or $\xi(\cvar_1) = \xi(\cvar_2) > 0$, then
  $\cvar_1 = \cvar_2$,
\item $\gamma$ is a substitution in $\{ \cvar \in \Vbound \mid \mu(\cvar) > 0 \} \mapsto \{ \cvar \in \Vbound \mid \xi(\cvar) > 0 \}$,
\item for every $\cvar \in \FV(s)$ with $\mu(\cvar) = 0$ also $\xi(\cvar) = 0$,
\item for every $\cvar \in \FV(s)$ with $\mu(\cvar) > 0$ also $\xi(\gamma(\cvar)) = \mu(\cvar)$
\item and $s =_\alpha^{\mu,\xi,k+1} t$
\item then $\subrel{s,\gamma,t}$.
\end{itemize}
Using $\gamma = [\avar:=\bvar]$, $\mu = [\avar:=1]$, $\xi = [\bvar:=1]$, $\gamma = [\avar:=\bvar]$ and $k = 1$ we are done.

The proof of the above claim is done by a straightforward induction on $s$.
\hideproof{
    \begin{itemize}
    \item If $s = a(s_1,\dots,s_n)$ with $n > 0$ then $t = b(t_1,\dots,t_n)$ with $a =_\alpha^{\mu,\xi,k+1} b$
      and $s_i =_\alpha^{\mu,\xi,k+1} t_i$ for all $i$.
      By the induction hypothesis, $\subrel{a,\gamma,b}$ and $\subrel{s_i,\gamma,t_i}$ for all $i$, and
      therefore $\subrel{s,\gamma,t}$ by definition. %Lemma \ref{lem:appsubstitute}.
    \item If $s = \afun$ then $t = s$ and indeed $\subrel{s,\gamma,t}$.
    \item If $s = \avar$ with $\avar \notin \domain(\gamma)$, then $\mu(\avar) = 0$ (or $\avar \in \Vfree$)
      so $t = \avar$ as well, and indeed $\subrel{s,\gamma,t}$.
    \item If $s = \avar \in \domain(\gamma)$ then $t = \bvar$ such that $\mu(\avar) = \xi(\bvar) > 0$.
      By assumption, $\bvar$ is unique, and since $\xi(\gamma(\avar)) = \mu(\avar)$ we must have
      $t = \gamma(\avar)$, so $\subrel{s,\gamma,t}$.
    \item If $s = \abs{\avar}{s'}$ then $t = \abs{\bvar}{t'}$ with $s' =_\alpha^{\mu[\avar:=k+1],
      \xi[\bvar:=k+1],k+2} t'$.  Let $\gamma^{\avar:=\bvar} := [\avar := \bvar] \cup [\cvar :=
      \gamma(\cvar) \mid \cvar \in \domain(\gamma) \setminus \{\avar\}]$.
      Then by the induction hypothesis, $\subrel{s',\gamma^{\avar:=\bvar},t'}$.
      We immediately obtain $\subrel{s,\gamma,t}$.
    \item If $s = \meta{\avar}{s_1,\dots,s_k}$ then note that $\avar \notin \domain(\gamma)$.
      As $t = \meta{\avar}{t_1,\dots,t_k}$ with $s_i =_\alpha^{\mu,\xi,k+1} t_i$, we find
      $\subrel{s_i,\gamma,t_i}$ for all $i$ by the induction hypothesis; $\subrel{s,\gamma,t}$
      immediately follows.
    \end{itemize}
}
\end{proof}

\subsection{Combining substitutions}

The proof that $(s\gamma)\delta = s(\gamma\delta)$ is actually quite complex, both due to abstraction
and meta-application.  For this reason, we split it up into multiple smaller steps.

To start, we prove a version of this result limited to substitutions on $\Vbound$.

\begin{lemma}\label{lem:combinesubst:allbound}
Let $\domain(\gamma),\domain(\delta),\domain(\eta) \subseteq \Vbound$ with
and let $s$ be a term such that
$\subrel{\gamma(\avar),\delta,\eta(\avar)}$ for all $\avar \in \FV(s)$.
Let $t$ be such that $\subrel{s,\gamma,t}$.
Then there exists $u$ such that both $\subrel{t,\delta,u}$ and $\subrel{s,\eta,u}$.
\end{lemma}

If $\eta = \gamma\delta$ -- that is, $\domain(\eta) = \domain(\gamma) \cup \domain(\delta)$ and
$\subrel{\gamma(\avar),\delta,\eta(\avar)}$ for all $\avar$ in this domain -- then the
requirement on $\eta$ is certainly satisfied: it is satisfied for $\avar \in \domain(\eta)$ by
definition, and for $\avar$ not in this domain, $\subrel{\avar,\delta,\avar}$ holds.
We formulate it in this more general way because this will be convenient in the induction.

\begin{proof}
By induction on the size of $s$.
In the case of an abstraction $s = \abs{\avar}{s'}$ and $t = \abs{\bvar}{t'}$, we choose $\cvar$
outside both $\FV(\delta(w))$ for all $w \in \FV(t)$, and $\FV(\eta(w))$ for all $w \in \FV(s)$.
Since $\delta$ extended with $[\bvar:=\cvar]$ and $\eta$ extended with $[\avar:=\cvar]$ are both
substitutions with domain $\subseteq \Vbound$, we can apply the induction hypothesis.
The case of a meta-application $\meta{\avar}{s_1,\dots,s_k}$ is trivial with the induction
hypothesis since $\avar$ is not in the domain of $\gamma$ or $\delta$.
\hideproof{
    \begin{itemize}
    \item If $s = h(s_1,\dots,s_n)$ with $n > 0$, then there exist $q, t_1,\dots,t_n$ such that
      $\subrel{h,\gamma,q}$ and $\subrel{s_i,\gamma,t_i}$ for all $i$, and $t = q \cdot t_1 \cdots
      t_n$.  By the induction hypothesis, there exist $w$ such that both
      $\subrel{q,\delta,w}$ and $\subrel{h,\eta,w}$, and $u_1,\dots,u_n$ such that for all $i$
      both $\subrel{t_i,\delta,u_i}$ and $\subrel{s_i,\eta,u_i}$.
      Choose $u := w \cdot u_1 \cdots u_n$.
      Then $\subrel{t,\delta,u}$ by Lemma \ref{lem:appsubstitute}, and $\subrel{s,\eta,u}$ by
      definition.
    \item If $s = \afun$ then $t = \afun$; we choose $u := \afun$ as well.
    \item If $s = \avar$ then $t = \gamma(\avar)$ and $\subrel{t,\delta,\eta(\avar)}$ by assumption.
      We let $u := \eta(\avar)$.
    \item If $s = \abs{\avar}{s'}$ then $t = \abs{\bvar}{t'}$ with
      $\bvar \notin \FV(\gamma(w))$ for any $w \in \FV(s)$, and we have
      $\subrel{s',\gamma^{\avar:=\bvar},t'}$ where $\gamma^{\avar:=\bvar} = [\avar:=\bvar] \cup
      [w:=\gamma(w) \mid w \in \domain(\gamma) \setminus \{\avar\}]$.
      Let $\cvar$ be a variable of the same type as $\avar$, such that:
      \begin{itemize}
      \item $\cvar \notin \FV(\delta(w))$ for any $w \in \FV(t)$;
      \item $\cvar \notin \FV(\eta(w))$ for any $w \in \FV(s)$;
      \end{itemize}
      \ \\
      Since $\cvar$ is only required to be unequal to a finite number of variables, and $\Vbound$ has
      infinitely many elements of each type, such a variable can always be found.
    
      Now let $\delta^{\bvar:=\cvar} = [\bvar:=\cvar] \cup [w:=\delta(w) \mid w \in \domain(\delta)
      \setminus \{\bvar\}]$ and let $\eta^{\avar:=\cvar} = [\avar:=\cvar] \cup [w:=\eta(w) \mid w \in
      \domain(\eta) \setminus \{\avar\}]$.  Then clearly $\subrel{\gamma^{\avar:=\bvar}(w),\delta^{
      \bvar:=\cvar},\eta^{\avar:=\cvar}(w)}$ for all $w \in \FV(s')$:
      \begin{itemize}
      \item for $w = \avar$ we have $\subrel{\bvar,\delta^{\bvar:=\cvar},\cvar}$;
      \item and for all other $w$ we have $\subrel{\gamma(w),\delta^{\bvar:=\cvar},\eta(w)}$ by Lemma
        \ref{lem:substextend},because $\bvar \notin \FV(\gamma(w))$ when $w \in \FV(s) = \FV(s')
        \setminus \{\avar\}$ (by choice of $\bvar$).
      \end{itemize}
      \ \\
      Hence, we can apply the induction hypothesis on $\subrel{s',\gamma^{\avar:=\bvar},t'}$ to obtain
      $u'$ such that both $\subrel{t',\delta^{\bvar;=\cvar},u'}$ and $\subrel{s',\eta^{\avar:=\cvar},
      u'}$.  We choose $u := \abs{\cvar}{u'}$.  Due to the requirements on $\cvar'$ we have both
      $\subrel{t,\delta,u}$ and $\subrel{s,\eta,u}$.
    \item If $s = \meta{\avar}{s_1,\dots,s_k}$ then $\avar \in \Vmeta$, so $\avar \notin \domain(\gamma)
      \cup \domain(\delta)$.  Hence, $t = \meta{\avar}{t_1,\dots,t_k}$ for some $t_1,\dots,t_k$ such that
      $\subrel{s_i,\gamma,t_i}$ for all $i$.  By the induction hypothesis on each $s_i$, there exist
      $u_1,\dots,u_k$ such that $\subrel{t_i,\delta,u_i}$ and $\subrel{s_i,\eta,u_i}$ for all $i \in
      \{1,\dots,k\}$.  We are done choosing $u := \meta{\avar}{u_1,\dots,u_k}$.
      \qedhere
    \end{itemize}
}
\end{proof}

The next helper result is phrased in a very technical way so as to be usable in other proofs, but
it essentially expands on Lemma \ref{lem:combinesubst:allbound} by allowing the second substitution,
but not the first, to have non-binder variables in its domain.  It is not quite the same, however;
rather than showing that $(s\gamma)\delta = s(\gamma\delta)$ it states that, under certain
restrictions, $(s\gamma)\delta = s\delta_1(\gamma\delta)_2$, where $\delta_1$ is the
restriction of $\delta$ to variables which do not occur in $\domain(\gamma)$, and $(\gamma\delta)_2$
is the restriction of $\gamma\delta$ to variables in the domain of $\gamma$.
This for instance allows us to conclude that $a[\avar_1:=s_1,\dots,\avar_k:=s_k]\delta =
a[\avar_1:=s_1\delta,\dots,\avar_k:=s_k\delta]$ if $\FV(s) \cap (\domain(\delta) \setminus
\{\avar_1,\dots,\avar_k\}) = \emptyset$.

\begin{lemma}\label{lem:combinesubst:gammabound}
Let $\domain(\gamma),\domain(\eta) \subseteq \Vbound$,
$s$ a term, and $\chi$ a mapping from $\domain(\gamma)$ to $\domain(\eta)$, such that:
\begin{itemize}
\item $\subrel{s,\gamma,t}$
\item $\subrel{\gamma(\avar),\delta,\eta(\chi(\avar))}$ for all $\avar \in \domain(\gamma) \cap
  \FV(s)$
\item $\domain(\eta) \cap \FV(\delta(\avar)) = \emptyset$ for all $\avar \in \FV(s) \setminus
  \domain(\gamma)$
\item $\subrel{s,\delta^\chi,q}$, where $\delta^\chi$ is the substitution on domain
  $\domain(\gamma) \cup \domain(\delta)$ with
  $\delta^\chi(\avar) = \chi(\avar)$ if $\avar \in \domain(\gamma)$ and $\delta^\chi(\avar) =
  \delta(\avar)$ otherwise.
\end{itemize}
Then there exists $u$ such that both $\subrel{t,\delta,u}$ and $\subrel{q,\eta,u}$.
\end{lemma}

\begin{proof}
We prove the result by induction on the size of $s$.
The proof has quite a few intricate cases, and relies on the previous results.
We use Lemma \ref{lem:appsubstitute} for the case of $a$ being an application $h(s_1,\dots,s_n)$ with $n > 0$;
Lemma \ref{lem:substitutionrefl} for the case of a variable $\avar \notin \{\bvar_1,\dots,\avar_k\}$;
Lemma \ref{lem:substextend} and careful variable renaming for the case of an abstraction; and
Lemma \ref{lem:combinesubst:allbound} for the case of a meta-application.
\hideproof{
    \begin{itemize}
    \item If $s = h(s_1,\dots,s_n)$ with $n > 0$ then:
      \begin{itemize}
      \item $t = v_0 \cdot v_1 \cdots v_n$ with $\subrel{h,\gamma,v_0}$ and $\subrel{s_i,\gamma,v_i}$
        for $1 \leq i \leq n$, and
      \item $q = w_0 \cdot w_1 \cdots w_n$ with $\subrel{h,\delta^\chi,w_0}$ and
        $\subrel{s_i,\delta^\chi,w_i}$ for $1 \leq i \leq n$.
      \end{itemize}
      \ \\
      Hence, by the induction hypothesis, there exist $u_0,\dots,u_n$ with
      $\subrel{v_i,\delta,u_i}$ and $\subrel{w_i,\eta,u_i}$ for $0 \leq i \leq n$.
      Choose $u := u_0 \cdot u_1 \cdots u_n$. Then both
      $\subrel{t,\delta,u}$ and $\subrel{q,\eta,u}$ by Lemma \ref{lem:appsubstitute}.
    \item If $s = \afun$ then necessarily $t = q = \afun$ as well; we let $u := \afun$.
    \item If $s = \avar \in \domain(\gamma)$ then $t = \gamma(\avar)$ and $q = \delta^\chi(\avar) =
      \chi(\avar)$.  We choose $u := \eta(\chi(\avar))$.  Then by assumption $\subrel{t,\delta,u}$
      (since $\avar \in \FV(s) \cap \domain(\gamma)$) and by definition $\subrel{q,\eta,u}$.
    \item If $s = \avar \in \V \setminus \domain(\gamma)$, then $t = \avar$ and $q = \delta(\avar)$.
      We choose $u := q$.  Then clearly $\subrel{t,\delta,u}$.  Since $\domain(\eta) \cap \FV(q) =
      \emptyset$ we have $\subrel{q,\eta,u}$ by Lemma \ref{lem:substitutionrefl}.
    \item If $s = \abs{\avar}{s'}$ then $t = \abs{\bvar_1}{t'}$ and $q = \abs{\bvar_2}{q'}$ with:
      \begin{itemize}
      \item $\bvar_1 \notin \FV(\gamma(w))$ for any $w \in \FV(s) = \FV(s') \setminus \{\avar\}$
      \item $\bvar_2 \notin \FV(\delta^\chi(w))$ for any $w \in \FV(s) = \FV(s') \setminus \{\avar\}$
      \end{itemize}
      \ \\
      Let $\cvar$ be a variable of the same type as $\avar$, such that $\cvar \notin \FV(\delta(w))$
      for any $w \in \FV(t)$ and $\cvar \notin \FV(\eta(w))$ for any $w \in \FV(q)$.
      Since there are infinitely many binder variables of all types, we can always find such $\cvar$.
      Now let:
      \begin{itemize}
      \item $\gamma^{\avar:=\bvar_1} := [\avar:=\bvar_1] \cup [w:=\gamma(w) \mid w \in \domain(\gamma)
        \setminus \{\avar\}$
      \item $\delta^{\bvar_1:=\cvar} := [\bvar_1:=\cvar] \cup [w:=\delta(w) \mid w \in \domain(\delta)
        \setminus \{\bvar\}]$
      \item $\eta^{\bvar_2:=\cvar} := [\bvar_2:=\cvar] \cup [w:=\eta(w) \mid w \in \domain(\eta)
        \setminus \{\bvar\}]$
      \item $\chi^{\avar:=\bvar_2} := [\avar:=\bvar_2] \cup [w := \chi(w) \mid w \in \domain(\chi)
        \setminus \{\avar\}]$
      \end{itemize}
      \ \\
      We can apply the induction hypothesis, because:
      \begin{itemize}
      \item $\chi^{\avar:=\bvar_2}$ maps $\domain(\gamma) \cup \{\avar\} =
        \domain(\gamma^{\avar:=\bvar_1})$ to $\domain(\eta) \cup \{\bvar_2\} = \domain(\eta^{\bvar_2:=
        \cvar})$
      \item $\domain(\gamma^{\avar:=\bvar_1}),\domain(\eta^{\bvar_2:=\cvar}) \subseteq \Vbound$;
      \item $\subrel{s',\gamma^{\avar:=\bvar},t'}$ follows from $\subrel{s,\gamma,t}$
      \item $\subrel{\gamma^{\avar:=\bvar_1}(w),\delta^{\bvar_1:=\cvar},\eta^{\bvar_2:=\cvar}(\chi^{
        \avar:=\bvar_2}(w))}$ for all $w \in \FV(s') \cap \domain(\gamma^{\avar:=\bvar})$:
        \begin{itemize}
        \item if $w = \avar$ then this exactly says $\subrel{\bvar_1,\delta^{\bvar_1:=\cvar},\cvar}$,
          which clearly holds;
        \item otherwise, we must show $\subrel{\gamma(w),\delta^{\bvar_1:=\cvar},\eta^{\bvar_2:=\cvar}(
          \chi(w))}$; we have $\chi(w) \neq \bvar_2$ because otherwise $\bvar_2 \in \FV(\delta^{\chi(w)})$
          (as $w \in \FV(s') \setminus \{\avar\}$), and therefore $\eta^{\bvar_2:=\cvar}(\chi(w)) =
          \eta(\chi(w))$; and since $\bvar_1 \notin \FV(\gamma(w))$, Lemma \ref{lem:substextend} allows
          us to derive $\subrel{\gamma(w),\delta^{\bvar_1:=\cvar},\eta(\chi(w))}$ from
          $\subrel{\gamma(w),\delta,\eta(\chi(w))}$
        \end{itemize}
      \item $\domain(\eta^{\bvar_2:=\cvar}) \cap \FV(\delta^{\bvar_1:=\cvar}(w)) = \emptyset$ for all
        $w \in \FV(s') \setminus \domain(\gamma^{\avar:=\bvar_1})$:
        \begin{itemize}
        \item $w = \avar$ cannot hold, since $\avar \in \domain(\gamma^{\avar:=\bvar_1})$
        \item $w = \bvar_1$ cannot hold, since $\bvar_1 \notin \FV(\gamma(a))$ for any $a \in
          \FV(s') \setminus \{\avar\}$; so if $\bvar_1 \in \FV(s')$ then $\FV \in \domain(\gamma)
          \subseteq \domain(\gamma^{\avar:=\bvar_1})$
        \item otherwise, note that $w \notin \domain(\gamma^{\avar:=\bvar})$ implies $w \notin
          \domain(\gamma)$, so $\delta^{\bvar_1:=\cvar}(w) = \delta(w) =
          \delta^\chi(w)$; we both have $\bvar_2 \notin \FV(\delta^\chi(w))$ and $\domain(\eta) \cap
          \FV(\delta^\chi(w)) = \emptyset$, so $\domain(\eta^{\bvar_2:=\cvar}) \cap
          \FV(\delta^{\bvar_1:=\cvar}(w)) = \emptyset$ as well
        \end{itemize}
      \item $\subrel{s',\delta_1,q'}$, where $\delta_1$ is the substitution on domain
        $\domain(\delta^{\bvar_1:=\cvar}) \cup \domain(\gamma^{\avar:=\bvar_1})$ with $\delta_1(w) =
        \chi^{\avar:=\bvar_2}(w)$ if $w \in \domain(\gamma^{\avar:=\bvar_1})$ and $\delta_1(w) =
        \delta^{\bvar_1:=\cvar}(w)$ otherwise:
        \begin{itemize}
          \item from $\subrel{s,\delta^\chi,q}$ we obtain $\subrel{s',\delta_2,q'}$, where
            $\delta_2 = [\avar:=\bvar_2] \cup [w := \delta^\chi(w) \mid w \in \domain(\delta^\chi)
            \setminus \{\avar\}]$;
          \item for all $w \in \FV(s')$: $\delta_1(w) = \delta_2(w)$:
            \begin{itemize}
            \item if $w = \avar$ then $\delta_1(w) = \bvar_2 = \delta_2(w)$
              % NOTE: here we use that $\avar \in \domain(\gamma^{\avar:=\bvar_1}) even if
              % $\avar = \bvar_1$!  So this lemma only holds because we use an unusual definition
              % of domain.  For the common definition, it needs to be reformulated in a less
              % convenient way.
            \item if $w \in \domain(\gamma) \setminus \{\avar\}$ then $\delta_1(w) = \chi(w) =
              \delta^\chi(w) = \delta_2(w)$
            \item if $w \in \domain(\delta) \setminus (\domain(\gamma) \cup \{\avar\})$ then
              $\delta_1(w) = \delta^{\bvar_1:=\cvar}(w) = \delta(w)$ (**), $= \delta^\chi(w) =
              \delta_2(w)$, where (**) holds because $w \neq \bvar_1$:
              if $w = \bvar_1$, then $\bvar_1 \in \FV(\gamma(\bvar_1))$, contradicting the
              earlier assumption on $\bvar_1$
            \end{itemize}
          \item hence, by Lemma \ref{lem:substextend}, also $\subrel{s',\delta',q'}$
        \end{itemize}
      \end{itemize}
      \ \\
      Hence, by the induction hypothesis there exists $u'$ such that
      $\subrel{t', \delta^{\bvar_1:=\cvar}, u'}$ and $\subrel{q',\eta^{\bvar_2:=\cvar}, u'}$.
      We are done choosing $u := \abs{\cvar}{u}$.
    \item If $s = \meta{\avar}{s_1,\dots,s_k}$ then $\avar \in \Vmeta$ so $\avar \notin
      \domain(\gamma)$.  Hence, $t = \meta{\avar}{t_1,\dots,t_k}$ with $\subrel{s_i,\gamma,t_i}$ for
      $1 \leq i \leq k$.  Moreover, $\delta^\chi(\avar) = \delta(\avar)$.
    
      If $\avar \notin \domain(\delta)$, then $\avar \notin \domain(\delta^\chi(\avar))$ and
      $q = \meta{\avar}{q_1,\dots,q_k}$ where $\subrel{s_i,\delta^\chi,q_i}$ for $1 \leq i \leq k$.
      We apply the induction hypothesis to find $u_1,\dots,u_k$ such that
      $\subrel{t_i,\delta,u_i}$ and $\subrel{q_i,\eta,u_i}$ for $1 \leq i \leq k$.
      Since $\avar \notin \domain(\eta)$ either, we are done choosing $u := \meta{\avar}{u_1,\dots,
      u_k}$.
    
      Otherwise, write $\delta(\avar) = \abs{\bvar_1 \dots \bvar_k}{w}$.
      Then there exist $q_1,\dots,q_k$ such that $\subrel{w, [\bvar_1:=q_1,\dots,\bvar_k:=q_k],q}$
      and $\subrel{s_i,\delta^\chi,q_i}$ for $1 \leq i \leq k$.
      Again, we apply the induction hypothesis to find $u_1,\dots,u_k$ such that
      $\subrel{t_i,\delta,u_i}$ and $\subrel{q_i,\eta,u_i}$ for $1 \leq i \leq k$.
      Then we have:
      \begin{itemize}
      \item the substitutions $\gamma' := [\bvar_1:=q_1,\dots,\bvar_k:=q_k]$;
        $\delta' := \eta$;
        $\eta' := [\bvar_1:=u_1,\dots,\bvar_k:=u_k]$
        all have a domain $\subseteq \Vbound$;
      \item for all $\cvar \in \FV(w)$: $\subrel{\gamma'(\cvar),\delta',\eta'(\cvar)}$:
        if $\cvar = \bvar_i$ (with not $\bvar_i = \bvar_j$ for any $j > i$) then
        this says $\subrel{q_i,\eta,u_i}$; and otherwise, this says $\subrel{\cvar,\eta,
        \cvar}$ which holds because if $\cvar \in \FV(w) \setminus \{\bvar_1,\dots,
        \bvar_k\}$ then $\cvar \in \FV(\delta(\avar))$, so $\cvar \notin \domain(\eta)$ by
        assumption
      \item $\subrel{w,\gamma',q}$
      \end{itemize}
      \ \\
      Hence, by Lemma \ref{lem:combinesubst:allbound} we find $u$ such that boy
      $\subrel{q,\eta,u}$ and $\subrel{w,[\bvar_1:=u_1,\dots,\bvar_k:=u_k],u}$; the latter
      immediately gives $\subrel{t,\delta,u}$.
      \qedhere
    \end{itemize}
}
\end{proof}

With these preparations we are ready to show that $s(\gamma\delta) = (s\gamma)\delta$.
Using the $\subrel{}$ relation, this is formulated as follows:

\begin{lemma}\label{lem:combinesubst:main}
Let $s,t$ be terms, and $\gamma,\delta,\eta$ be substitutions, such that for all variables $x$ in
$\FV(s)$: $\subrel{\gamma(x),\delta,\eta(x)}$.  Then,
  if $\subrel{s,\gamma,t}$ then there exists $u$ such that both $\subrel{t,\delta,u}$ and
  $\subrel{s,\eta,u}$.
\end{lemma}

\begin{proof}
By induction on the definition of $\subrel{}$ (or, put differently, induction first on the presence
of $\Vmeta$ variables in $\domain(\gamma)$, second on the size of $s$).
The cases where $s$ is not a meta-application are exactly the same as in the proof of Lemma
\ref{lem:combinesubst:allbound}, so we only need to consider the cases $s =
\meta{\avar}{s_1,\dots,s_k}$.  These follows easily with the induction hypothesis and 
Lemma \ref{lem:combinesubst:gammabound}.
\hideproof{
    \begin{itemize}
    \item If $s = \meta{\avar}{s_1,\dots,s_k}$ and $\avar \notin \domain(\gamma)$, then
      $t = \meta{\avar}{t_1,\dots,t_k}$ with $\subrel{s_i,\gamma,t_i}$ for all $i$.  By the
      induction hypothesis we find $u_1,\dots,u_k$ such that both $\subrel{t_i,\delta,u_i}$
      and $\subrel{s_i,\eta,u_i}$ for all $i$.
      If also $\avar \notin \domain(\delta)$, then we are done choosing $u := \meta{\avar}{
      u_1,\dots,u_k}$.  Otherwise, write $\delta(\avar) = \abs{\bvar_1 \dots \bvar_k}{q}$.
      By Lemma \ref{lem:substdefined}, there exists $u$ such that
      $\subrel{q,[\bvar_1:=u_1,\dots,\bvar_k:=u_k],u}$.
      This property gives both $\subrel{t,\delta,u}$ and $\subrel{s,\eta,u}$ (since
      $\eta(\avar) = \delta(\avar)$).
    \item If $s = \meta{\avar}{s_1,\dots,s_k}$ and $\avar \in \domain(\gamma)$, we can write
      $\gamma(\avar) = \abs{\bvar_1 \dots \bvar_k}{a}$.
      There exist $t_1,\dots,t_k$ such that $\subrel{s_i,\gamma,t_i}$ for all $i$, and
      $\subrel{a,[\bvar_1:=t_1,\dots,\bvar_n:=t_k],t}$.
      By the induction hypothesis on each $s_i$, there exist $u_1,\dots,u_k$ such that both
      $\subrel{t_i,\delta,u_i}$ and $\subrel{s_i,\eta,u_i}$ for all $i \in \{1,\dots,k\}$.
    
      By Lemma \ref{lem:abssubst} and $\subrel{\gamma(\avar),\delta,\eta(\avar)}$, we can write
      $\eta(\avar) = \abs{\cvar_1 \cdots \cvar_k}{b}$ such that:
      \begin{itemize}
      \item each $\cvar_i \notin \FV(\delta(w))$ for any $w \in \FV(a) \setminus \{\bvar_1,\dots,\bvar_k\}$
      \item if $\cvar_i = \cvar_j$ for some $i < j$ then $\bvar_i \notin \FV(a)$ or $\bvar_i = \bvar_l$ for some $l > i$
      \item $\subrel{a,[\bvar_1:=\cvar_1,\dots,\bvar_k:=\cvar_k] \cup [w := \delta(w) \mid
        w \in \domain(\delta) \setminus \{\bvar_1,\dots,\bvar_k\}],b}$
      \end{itemize}
      \ \\
      We apply Lemma \ref{lem:combinesubst:gammabound} using:
        $s' := a$, $t' := t$, $q' := b$;
        $\gamma' := [\bvar_1:=t_1,\dots,\bvar_k:=t_k]$;
        $\delta' := \delta$;
        $\eta' := [\cvar_1:=u_1,\dots,\cvar_k:=u_k]$; and
        $\chi' := [\bvar_1:=\cvar_1,\dots,\bvar_k:=\cvar_k]$.
      This provides $u$ such that both $\subrel{t,\delta,u}$ and $\subrel{b,[\cvar_1:=u_1,\dots,\cvar_k:=u_k],u}$;
      this latter relation immediately gives $\subrel{s,\eta,u}$ as required.

      To see that we are allowed to apply Lemma \ref{lem:combinesubst:gammabound} (and thus complete the proof),
      we note:
      \begin{itemize}
      \item $\subrel{s',\gamma',t'}$ is given directly by $\subrel{a,[\bvar_1:=t_1,\dots,\bvar_n:=t_k],t}$;
      \item each variable in $\FV(a) \cap \domain(\gamma')$ is some $\bvar_i$ such that no $j > i$ exists
        with $\bvar_i = \bvar_j$; then also there exists no $j > i$ with $\cvar_i = \cvar_j$, so
        $\gamma'(\bvar_i) = t_i$ and $\eta'(\chi(\bvar_i)) = \eta'(\cvar_i) = u_i$, and since
        we chose $u_i$ so that $\subrel{t_i,\delta,u_i}$ we indeed have
        $\subrel{\gamma'(\bvar_i),\delta',\eta'(\chi'(\bvar_i))}$
      \item for $w \in \FV(a) \setminus \domain(\gamma')$, so $w \in \FV(a) \setminus \{\bvar_1,\dots,\bvar_k\}$,
        we have already observed $\cvar_i \notin \FV(\delta(w))$ for any $i$;
        hence, $\domain(\eta') \cap \FV(\delta(w)) = \emptyset$
      \item $\subrel{a,\delta^\chi,b}$ is given.
        \qedhere
      \end{itemize}
    \end{itemize}
}
\end{proof}

We conclude:

\begin{lemma}\label{lem:combinesubst}
Always $s(\gamma\delta) =_\alpha (s\gamma)\delta$.
\end{lemma}

\begin{proof}
Let $\eta = \gamma\delta$; that is, $\domain(\eta) = \domain(\gamma) \cup \domain(\delta)$ and
$\eta(x) = \gamma(x)\delta$ for all $x \in \domain(\eta)$.  Then clearly $\subrel{\gamma(x),
\delta,\eta(x)}$ for all $x$:
if $x \in \domain(\gamma) \cup \domain(\delta)$ this is assumed,
and otherwise $\subrel{x,\delta,x}$.
By Lemma \ref{lem:combinesubst:main} there exists $u$ such that $s\eta =_\alpha (s\gamma)\eta$,
so if $s\eta = t$ and $(s\gamma)\delta = t'$ then by Corollary \ref{cor:substitutionalpha}
also $t =_\alpha u =_\alpha t'$.
\end{proof}

\subsection{Well-definedness of reduction}

We need to prove that reduction is well-defined on equivalence classes of terms, since the
definition uses subterms, which are not the same under $\alpha$-equivalence.

\begin{lemma}\label{lem:reductionwelldefined}
If $s =_\alpha s'$ and $s \arr{\Rules} t$ then exists $t'$ with $t =_\alpha t'$ such that
$s' \arr{\Rules} t'$.
\end{lemma}

\begin{proof}
TODO
\end{proof}

\end{document}

%\section{Unconstrained first-order term rewriting}
%
%Although first-order (many-sorted) term rewriting systems can be seen as a kind of higher-order
%term rewriting system, we will present their definition separately first, and later explain how
%they can be viewed as part of the larger framework.
%
%\subsection{Terms} When considering \emph{first-order} term rewriting, we limit interest to $\F$
%with the following property: for every $(\afun : \atype) \in \F$ we have $\order(\atype) \leq 1$.
%First-order terms are those expressions $s$ such that $s : \asort$ can be derived for some
%\emph{base type} $\asort$ using the following clauses:
%\begin{itemize}
%\item if $(\afun : \atype_1 \arrtype \dots \arrtype \atype_n \arrtype \asort) \in \F$ and
%  $s_1 : \atype_1,\dots,s_n : \atype_n$ then $\afun(s_1,\dots,s_n) : \asort$;
%\item if $(\avar : \asort) \in \V$ then $\avar : \asort$.
%\end{itemize}
%We denote $\FOTerms(\F,\V)$ for the set of all first-order terms $s$.
%A first-order term of the form $\afun(s_1,\dots,s_n)$ is called a \emph{functional term} and
%$\afun$ is its root; a term $\avar$ is simply called a variable.
%If $s : \asort$ then we say that $\asort$ is the type of $s$; it is clear from the definitions
%above that each term has a unique type (which is a base type).
%
%The set $\FV(s)$ of \emph{variables} of a term $s$ is inductively defined as follows:
%\begin{itemize}
%\item $\FV(\afun(s_1,\dots,s_n)) = \FV(s_1) \cup \dots \cup \FV(s_n)$;
%\item $\FV(\avar) = \{ \avar \}$.
%\end{itemize}
%%That is, $\FV(s)$ contains all variables in $s$.
%
%\bigskip
%The \emph{subterm} relation $\subtermeq$ is defined as follows:
%\begin{itemize}
%\item $s \subtermeq s$ for all $s$;
%\item $s \subtermeq \afun(s_1,\dots,s_n)$ if $s \subtermeq s_i$ for some $i$.
%\end{itemize}
%If $s \subtermeq t$ we say that $s$ \emph{is a subterm of} $t$.
%
%\subsection{Substitution}
%
%A substitution is a function $\gamma$ that maps each variable $\avar \in \V$ to a term
%$\gamma(\avar)$ of the same type.  A substitution is applied to an arbitrary first-order term as
%follows:
%\begin{itemize}
%\item $\afun(s_1,\dots,s_n)\gamma = \afun(s_1\gamma,\dots,s_n\gamma)$;
%\item $\avar\gamma = \gamma(\avar)$.
%\end{itemize}
%
%The \emph{domain} $\domain(\gamma)$ of a substitution $\gamma$ is the set of all variables $x$
%such that $\gamma(x) \neq x$.
%We denote $[x_1:=s_1,\dots,x_n:=s_n]$ for the substitution $\gamma$ with $\gamma(x_i) = s_i$ for
%$1 \leq i \leq n$ and $\gamma(y) = y$ for $y \notin \{x_1,\dots,x_n\}$.
%For two substitutions $\gamma$ and $\delta$, we let $\gamma\delta$ denote the substitution that
%maps each variable $x$ to $\gamma(x)\delta$.
%
%\subsection{Positions}
%
%The \emph{positions} of a given first-order term are the paths to specific subterms, defined as
%follows:
%
%\begin{itemize}
%\item $\Positions(\afun(s_1,\dots,s_n)) = \{ \epsilon \} \cup \{ i \cdot p \mid 1 \leq i \leq n
%  \wedge p \in \Positions(s_i) \}$;
%\item $\Positions(\avar) = \{ \epsilon \}$.
%\end{itemize}
%Note that positions are associated to a term; thus, not every integer sequence is a position.
%
%For a term $s$ and a position $p \in \Positions(s)$, the \emph{subterm of $s$ at position $p$},
%denoted $s|_p$, is defined as follows:
%\begin{itemize}
%\item $s|_\epsilon = s$;
%\item $\afun(s_1,\dots,s_n)|_{i \cdot p} = s_i|_p$;
%\end{itemize}
%
%Note that $t \subtermeq s$ if and only if there is some position $p \in \Positions(s)$ with
%$t = s|_p$.
%If $s|_p$ has the same type as some term $t$, then $s[t]_p$ denotes $s$ with the subterm at position
%$p$ replaced by $t$.  Formally, $s[t]_p$ is obtained as follows:
%\begin{itemize}
%\item $s[t]_p = t$;
%\item $\afun(s_1,\dots,s_n)[t]_{i \cdot p} = \afun(s_1,\dots,s_{i-1},s_i[t]_p,s_{i+1},\dots,s_n)$.
%\end{itemize}
%
%\subsection{Rules and reduction}
%
%A rule is a pair $\ell \arrz r$ of two terms with the same type.
%For a given set of rules $\Rules$, the reduction relation $\arr{\Rules}$ is given by:
%\begin{itemize}
%\item if there exist $\ell \arrz r \in \Rules$ and $p \in \Positions(s)$ and substitution $\gamma$
%  such that $s|_p = \ell\gamma$, then $s \arr{\Rules} s[r\gamma]_p$.
%\end{itemize}
%
%\bigskip
%A \emph{first-order term rewriting system (TRS)} is an abstract rewriting system of the form
%$(\FOTerms(\F,\V),\arr{\Rules})$.
%
%In principle, we have defined a \emph{many-sorted} TRS here; a traditional unsorted TRS is obtained
%by limiting interest to the case $\Sorts = \{ \unitsort \}$.


