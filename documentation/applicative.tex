\documentclass{lmcs}
\pdfoutput=1

\usepackage{enumerate}
\usepackage[colorlinks=true]{hyperref}
\usepackage{amssymb}
\usepackage{xcolor,latexsym,amsmath,extarrows,alltt}
\usepackage{xspace}
\usepackage{booktabs}
\usepackage{mathtools}
\usepackage{enumitem}
\usepackage{stmaryrd}
\usepackage{microtype}

\theoremstyle{theorem}\newtheorem{theorem}{Theorem}
\theoremstyle{theorem}\newtheorem{lemma}[theorem]{Lemma}
\theoremstyle{theorem}\newtheorem{corollary}[theorem]{Corollary}
\theoremstyle{definition}\newtheorem{definition}[theorem]{Definition}
\theoremstyle{definition}\newtheorem{example}[theorem]{Example}

\newcommand{\N}{\mathbb{N}}
\newcommand{\F}{\mathcal{F}}
\newcommand{\V}{\mathcal{V}}
\newcommand{\Vfree}{\mathcal{V}_{\mathit{free}}}
\newcommand{\Vbound}{\mathcal{V}_{\mathit{bound}}}
\newcommand{\Sorts}{\mathcal{S}}
\newcommand{\Types}{\mathcal{Y}}
\newcommand{\Terms}{\mathcal{T}}
\newcommand{\ATerms}{\mathcal{T}_{\mathcal{A}}}
\newcommand{\FOTerms}{\mathcal{T}_{\mathcal{FO}}}
\newcommand{\Rules}{\mathcal{R}}
\newcommand{\FV}{\mathit{FV}}
\newcommand{\Positions}{\mathit{Positions}}
\newcommand{\Pairs}{\mathit{Pairs}}

\newcommand{\domain}{\mathtt{dom}}
\newcommand{\order}{\mathit{order}}

\newcommand{\asort}{\iota}
\newcommand{\bsort}{\kappa}
\newcommand{\atype}{\sigma}
\newcommand{\btype}{\tau}
\newcommand{\ctype}{\pi}
\newcommand{\dtype}{\alpha}
\newcommand{\identifier}[1]{\mathtt{#1}}
\newcommand{\afun}{\identifier{f}}
\newcommand{\bfun}{\identifier{g}}
\newcommand{\cfun}{\identifier{h}}
\newcommand{\avar}{x}
\newcommand{\bvar}{y}
\newcommand{\cvar}{z}
\newcommand{\Avar}{X}
\newcommand{\Bvar}{Y}
\newcommand{\Cvar}{Z}
\newcommand{\AFvar}{F}
\newcommand{\BFvar}{G}
\newcommand{\CFvar}{H}

\newcommand{\abs}[2]{\lambda #1.#2}

\newcommand{\arity}{\mathit{arity}}
\newcommand{\head}{\mathsf{head}}
\newcommand{\arrtype}{\rightarrow}
\newcommand{\arrz}{\Rightarrow}
\newcommand{\arr}[1]{\arrz_{#1}}
\newcommand{\arrr}[1]{\arr{#1}^*}
\newcommand{\subtermeq}{\unlhd}
\newcommand{\headsubtermeq}{\unlhd_{\bullet}}
\newcommand{\headsuptermeq}{\unrhd_{\bullet}}
\newcommand{\supterm}{\rhd}
\newcommand{\suptermeq}{\unrhd}

\newcommand{\symb}[1]{\mathtt{#1}}

\newcommand{\nul}{\symb{0}}
\newcommand{\one}{\symb{1}}
\newcommand{\nil}{\symb{nil}}
\newcommand{\cons}{\symb{cons}}
\newcommand{\strue}{\symb{true}}
\newcommand{\sfalse}{\symb{false}}
\newcommand{\suc}{\symb{s}}
\newcommand{\map}{\symb{map}}
\newcommand{\bool}{\symb{bool}}
\newcommand{\nat}{\symb{nat}}
\newcommand{\lijst}{\symb{list}}
\newcommand{\unitsort}{\mathtt{o}}

\newcommand{\cora}{\textsf{CORA}\xspace}

\newcommand{\secshort}{\S}
\newcommand{\myparagraph}[1]{\paragraph{\textbf{#1}}}

\setlength{\parindent}{0pt}
\setlength{\parskip}{\bigskipamount}
\setlist[itemize]{topsep=-\bigskipamount}

\newcommand{\mysubsection}[1]{\vspace{-12pt}\subsubsection{#1}}

\begin{document}

\title{COnstrained Rewriting Analyser: formalism}
\author{Cynthia Kop}
\address{Department of Software Science, Radboud University Nijmegen}
\email{C.Kop@cs.ru.nl}

\maketitle

\begin{abstract}
\cora\ is a tool meant to analyse constrained term rewriting systems, both
first-order and higher-order.  This document explains the underlying formalism.
\end{abstract}

\section{Types}

We fix a set $\Sorts$ of \emph{sorts} and define the set $\Types$ of \emph{types} inductively:
\begin{itemize}
\item all elements of $\Sorts$ are types (also called \emph{base types});
\item if $\atype,\btype \in \Types$ then $\atype \arrtype \btype$ is also a type (called an arrow
  type).
\end{itemize}
The arrow operator $\arrtype$ is right-associative, so all types can be denoted in a form
$\atype_1 \arrtype \dots \arrtype \atype_m \arrtype \asort$ with $\asort \in \Sorts$; we say the
\emph{arity} of this type is $m$, and the \emph{output sort} is $\asort$.

The \emph{order} of a type is recursively defined as follows:
\begin{itemize}
\item for $\asort \in \Sorts$: $\order(\asort) = 0$;
\item for arrow types: $\order(\atype \arrtype \btype) = \max(\order(\atype) + 1,\order(\btype))$.
\end{itemize}

\bigskip
Type equality is literal equality (i.e., $\atype_1 \arrtype \btype_1$ is equal to $\atype_2 \arrtype
\btype_2$ iff $\atype_1 = \atype_2$ and $\btype_1 = \btype_2$).

\subsection*{Remarks}

We do not impose limitations on the set of sorts.  In traditional, unsorted term rewriting, there
is only one sort (e.g., $\Sorts = \{ \unitsort \}$). However, we may also have a larger finite or
even infinite sort set.
In the future, we may consider a shallow form of polymorphic types, but for the moment we will limit
interest to these simple types.

\section{Applicative higher-order term rewriting systems (ATRSs)}

Let us start by explaining systems without constraints. Most of the notions will be directly
relevant to constrained systems as well.

\subsection{Terms}
Terms are \emph{well-typed} expressions built over given sets of \emph{function symbols} and
\emph{variables}. The full definition is presented below.

\mysubsection{Symbols and variables}

We fix a set $\F$ of \emph{function symbols}, also called the \emph{alphabet}; each function symbol
is a \emph{typed constant}. Notation: $\afun \in \F$ or $(\afun : \atype) \in \F$ if we wish to
explicitly refer to the type (but the type should be considered implicit in the symbol).
Function symbols will generally be referred to as $\afun,\bfun,\cfun$ or using more suggestive
notation.

We also fix a set $\V$ of \emph{variables}, which are typed constants in the same way.  $\V$ should
be disjoint from $\F$.
Variables will generally be referred to as $\avar,\bvar,\cvar,\Avar,\Bvar,\Cvar,\AFvar,\BFvar,
\CFvar$ or using more suggestive notation.

\mysubsection{Term formation}

Terms are those expressions $s$ such that $s : \atype$ can be derived for some $\atype \in \Types$
using the following clauses:
\begin{itemize}
\item if $(\afun : \atype_1 \arrtype \dots \arrtype \atype_n \arrtype \btype) \in \F$ and
  $s_1 : \atype_1,\dots,s_n : \atype_n$ then $\afun(s_1,\dots,s_n) : \btype$;
\item if $(\avar : \atype_1 \arrtype \dots \arrtype \atype_n \arrtype \btype) \in \V$ and
  $s_1 : \atype_1,\dots,s_n : \atype_n$ then $\avar(s_1,\dots,s_n) : \btype$.
\end{itemize}
A term of the form $\afun(s_1,\dots,s_n)$ is called a \emph{functional term} and $\afun$ is its
root.
A term of the form $\avar(s_1,\dots,s_n)$ is called a \emph{var term}, and $\avar$ is its variable.
If $s : \atype$ then we say that $\atype$ is the type of $s$; it is clear from the definitions
above that each term has a unique type.

Note that in the definition above, $n$ is not required to be maximal; for example, if
$\symb{greater} : \mathtt{int} \arrtype \mathtt{int} \arrtype \mathtt{bool}$, then each of
$\symb{greater}(),\symb{greater}(\avar)$ and $\symb{greater}(\avar,\bvar)$ are terms (with
distinct types). When no arguments are given (i.e., a term $\afun()$ or $\avar()$), we may omit
the brackets and just denote $\afun$ or $\avar$ for the term.  In this sense, the elements of $\F$
and $\V$ may be considered terms.
A term $\avar$ can simply be called a variable (but is also still a var term);
a term $\afun$ may be called a constant (but is also still a functional term).

\mysubsection{Sets of terms}

We denote $\ATerms(\F,\V)$ for the set of all terms $s$.

A term is \emph{first-order} if its type can be derived by the following clauses:
\begin{itemize}
\item if $(\afun : \atype_1 \arrtype \dots \arrtype \atype_n \arrtype \asort) \in \F$ with $\asort
  \in \Sorts$ and
  $s_1 : \atype_1,\dots,s_n : \atype_n$ then $\afun(s_1,\dots,s_n) : \asort$;
\item if $(\avar : \asort) \in \V$ and $\asort \in \Sorts$, then $\avar : \asort$.
\end{itemize}
The set of first-order terms is denoted $\FOTerms(\F,\V)$.

\subsection{Subterms and positions}

The \emph{positions} of a given term are the paths to specific subterms, defined as follows:

\begin{itemize}
\item $\Positions(\afun(s_1,\dots,s_n)) = \{ \epsilon \} \cup \{ i \cdot p \mid 1 \leq i
  \leq n \wedge p \in \mathit{Positions}(s_i) \}$;
\item $\Positions(\avar(s_1,\dots,s_n)) = \{ \epsilon \} \cup \{ i \cdot p \mid 1 \leq i
  \leq n \wedge p \in \Positions(s_i) \}$;
\end{itemize}
Note that positions are associated to a term; thus, not every sequence of natural numbers is a
position.

For a term $s$ and a position $p \in \Positions(s)$, the \emph{subterm of $s$ at position $p$},
denoted $s|_p$, is defined as follows:
\begin{itemize}
\item $s|_\epsilon = s$;
\item $\afun(s_1,\dots,s_n)|_{i \cdot p} = s_i|_p$;
\item $\avar(s_1,\dots,s_n)|_{i \cdot p} = s_i|_p$;
\end{itemize}

If $s|_p$ has the same type as some term $t$, then $s[t]_p$ denotes $s$ with the subterm at position
$p$ replaced by $t$.  Formally, $s[t]_p$ is obtained as follows:
\begin{itemize}
\item $s[t]_\epsilon = t$;
\item $\afun(s_1,\dots,s_n)[t]_{i \cdot p} = \afun(s_1,\dots,s_{i-1},s_i[t]_p,s_{i+1},\dots,s_n)$;
\item $\avar(s_1,\dots,s_n)[t]_{i \cdot p} = \avar(s_1,\dots,s_{i-1},s_i[t]_p,s_{i+1},\dots,s_n)$.
\end{itemize}
Thus, we can find and replace the subterm at a given position.

We say that \emph{$t$ is a subterm of $s$}, notation $t \subtermeq s$, if there is some position
$p \in \Positions(s)$ with $t = s|_p$.  This could equivalently be formulated as follows:

\begin{lemma}
$t \subtermeq s$ if and only if one of the following holds:
\begin{itemize}
\item $s = t$;
\item $s = \afun(s_1,\dots,s_n)$ or $s = \avar(s_1,\dots,s_n)$ and $t \subtermeq s_i$ for some $i$;
\end{itemize}
\end{lemma}

It should be noted that in contrast to most definitions of higher-order rewriting, we do \emph{not}
consider, for example, $\afun(x)$ to be a subterm of $\afun(x,y)$.  Instead, we define the
following: \emph{$t$ is a head-subterm of $s$}, notation $t \headsubtermeq s$ if one of the
following holds:
\begin{itemize}
\item $t \subtermeq s$;
\item $t = \afun(s_1,\dots,s_i)$ and there exist $s_{i+1},\dots,s_n$ such that
  $\afun(s_1,\dots,s_n) \subtermeq s$;
\item $t = \avar(s_1,\dots,s_i)$ and there exist $s_{i+1},\dots,s_n$ such that
  $\avar(s_1,\dots,s_n) \subtermeq s$;
\end{itemize}
So, the head-subterms of $s$ are both the subterms of $s$, and those terms that occur as the head
of a subterm of $s$.

Regarding different kinds of terms: the subterms and positions of a first-order term by these
definitions are exactly the subterms and positions as they are usually considered in first-order
term rewriting; however, head-subterms are generally not considered.  For applicative terms,
both subterms and head-subterms are usually referred to as just ``subterms''; we distinguish them
here because doing so is practical for analysis.

\subsection{Application}\label{subsec:application}
A term $s : \atype_1 \arrtype \dots \arrtype \atype_m \arrtype \btype$ can be applied to a sequence
$[t_1,\dots,t_m]$ of terms, provided $t_1 : \atype_1,\dots,t_m : \atype_m$, through the following
clauses:
\begin{itemize}
\item $\afun(s_1,\dots,s_n) \cdot [t_1,\dots,t_m] = \afun(s_1,\dots,s_n,t_1,\dots,t_m)$;
\item $x(s_1,\dots,s_n) \cdot [t_1,\dots,t_m] = x(s_1,\dots,s_n,t_1,\dots,t_m)$.
\end{itemize}
We write $s \cdot t$ as short-hand notation for $s \cdot [t]$.

We actually have the following result:
\begin{lemma}\label{lem:applicative_notation}
The set $\ATerms(\F,\V)$ is the smallest set such that:
\begin{itemize}
\item $\F \cup \V \subseteq \ATerms(\F,\V)$;
\item if $s,t \in \ATerms(\F,\V)$ and $s : \atype \arrtype \btype$ and $t : \atype$ then
  $s \cdot t \in \ATerms(\F,\V)$.
\end{itemize}
\end{lemma}

\begin{proof}
Easy by induction on term formation (both directions).
\end{proof}

Lemma~\ref{lem:applicative_notation} shows that our applicative terms are the same as applicative
terms constructed in the traditional way; however, for convenience we denote them in a functional
notation.

\subsection{Substitution}

A \emph{substitution} is a function $\gamma$ that maps each variable $\avar \in \V$ to a term
$\gamma(\avar)$ of the same type.
The \emph{domain} $\domain(\gamma)$ of a substitution $\gamma$ is the set of all variables $x$
such that $\gamma(x) \neq x$.
We denote $[x_1:=s_1,\dots,x_n:=s_n]$ for the substitution $\gamma$ with $\gamma(x_i) = s_i$ for
$1 \leq i \leq n$ and $\gamma(y) = y$ for $y \notin \{x_1,\dots,x_n\}$.

Applying a substitution $\gamma$ to a term $s$, notation $s\gamma$, yields a new term of the same
type:
\begin{itemize}
\item $\afun(s_1,\dots,s_n)\gamma = \afun(s_1\gamma,\dots,s_n\gamma)$;
\item $\avar(s_1,\dots,s_n)\gamma = \gamma(\avar) \cdot [s_1\gamma,\dots,s_n\gamma]$
  (using application, see section \ref{subsec:application}).
\end{itemize}

\subsection{Free variables}

The set of \emph{free variables} of a term is inductively defined as follows:
\begin{itemize}
\item $\FV(\afun(s_1,\dots,s_n)) = \FV(s_1) \cup \dots \cup \FV(s_n)$;
\item $\FV(\avar(s_1,\dots,s_n)) = \{ \avar \} \cup \FV(s_1) \cup \dots \cup \FV(s_n)$;
\end{itemize}
That is, $\FV(s)$ contains all variables in $s$.

\subsection{Rules and rewriting}

A rule $\rho$ is a pair $\ell \arrz r$ of two terms with the same type, where $\FV(r) \subseteq
\FV(\ell)$ and $\ell$ is not a variable.

For a given set of rules $\Rules$, the reduction relation $\arr{\Rules}$ is given by:
\begin{itemize}
\item $s \arrz t$ if there exist:
  \begin{itemize}
  \item $\ell \arrz r \in \Rules$;
  \item $p \in \Positions(s)$;
  \item terms $v_1,\dots,v_n$;
  \item substitution $\gamma$;\\
  \end{itemize}
  such that $s|_p = \ell\gamma \cdot [v_1,\dots,v_n]$ and $t = s[r\gamma\cdot [v_1,\dots,v_n]]_p$.
\end{itemize}
That is, $s \arr{\Rules} t$ by a rule $\ell \arrz r$ if $s \headsuptermeq \ell\gamma$ and $t$ is
$s$ with the occurrence of $\ell\gamma$ replaced by $r\gamma$.
This explicitly includes applications of rules at the head of a subterm.

\mysubsection{Pattern rules}
A term $s$ is a \emph{pattern} if for every subterm $t \subtermeq s$ we have: if $t$ is a var term
$x(s_1,\dots,s_n)$ then $n = 0$.  That is, the only var terms occurring in a pattern are variables.
Patterns will be relevant in rule formation for specific limitations of ATRSs.

A rule is a \emph{pattern rule} if the left-hand side $\ell$ is a pattern.

\mysubsection{First-order rules}
A rule $\ell \arrz r$ is first-order if both $\ell$ and $r$ are in $\FOTerms(\F,\V)$.

\subsection{ATRSs}

We now have all the ingredients to define an \emph{applicative term rewriting system (ATRS)}.

\mysubsection{Abstract Rewriting Systems}

An abstract rewriting system is a pair $(\mathcal{A},\arrz)$ where $\mathcal{A}$ is a set and
$\arrz$ a binary relation on that set.  Properties such as termination and confluence can be
expressed in terms of abstract rewriting systems.

\mysubsection{ATRSs}

An applicative term rewriting system (ATRS) is an abstract rewriting system of the form
$(\ATerms(\F,\V),\arr{\Rules})$.

An applicative pattern term rewriting system (APTRS) is an ATRS $(\ATerms(\F,\V),\arr{\Rules})$
where all elements of $\Rules$ are pattern rules.

\mysubsection{First-sorted TRSs}

A \emph{many-sorted term rewriting system} (MTRS) is an abstract rewriting system of the form
$(\FOTerms(\F,\V),\arr{\Rules})$ where for all $(\afun : \atype) \in \F$: $\order(\atype) \leq 1$,
and $\Rules$ is a set of first-order rules.

An \emph{unsorted first-order term rewriting system} (TRS) is a many-sorted term rewriting system
with $\Sorts = \{ \unitsort \}$.

\end{document}

